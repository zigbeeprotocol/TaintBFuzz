\appendix

\chapter{Standalone Source Release}
\label{chap:make-src-distrib}

Requirements and tips to create a source release of Frama-C
(i.e. \texttt{make src-distrib}).

\section{Requirements}

You will need the following programs:

\begin{itemize}
\item a custom (unreleased) version of \texttt{headache};
\item on Mac OS, binaries \texttt{gtar} and \texttt{gzip}.
\end{itemize}

Details about how to install them are presented below.

\section{Custom headache}

Clone the following repository (by default, every Frama-C developer has
access to it):

\texttt{git clone git@git.frama-c.com/dev/headache.git}

Then run:

\verb+./configure && make && make install+

Note that some extra opam packages may be needed, such as \texttt{camomile}.

This version of \texttt{headache} needs to be in your PATH for the next steps.

\section{Check your headers}

Before making a source distribution, make sure your headers are up-to-date,
by running:

\texttt{make check-headers}

Any plug-ins you are distributing along with Frama-C should have their
distributed files (all OCaml sources, plus those listed in
\verb+PLUGIN_DISTRIBUTED_EXTERNAL+) listed in file:

\verb+headers/header_spec.txt+

Their headers will then be checked by the above make target.

\textbf{Note:} if you fail to check your headers, the distribution script may
corrupt your files. You were warned.

\section{(Mac OS only) GNU tools}

Some GNU-specific tools are necessary to generate a source distribution. They
are:

\begin{itemize}
\item \texttt{gtar}: available via \texttt{brew install gnu-tar}
\item \texttt{gzip}: available via \texttt{brew install gzip}
\end{itemize}

Both commands must be available in your PATH.

\section{Making a distribution archive}

You can make an {\em closed-source} distribution (default),
or an {\em open-source} distribution.

\begin{itemize}
\item closed-source: headers contain the CEA closed-souce license; safe to
  distribute (receiver cannot re-distribute them publicly).
\item open-source: headers are LGPL; used for public releases, but not for
  intermediate versions (e.g. when uploading a specific version of Frama-C
  inside a virtual machine for a paper).
\end{itemize}

Use Makefile variable \verb+OPEN_SOURCE=yes+ to create an open-source release,
as in:

\verb+make src-distrib OPEN_SOURCE=yes+

Otherwise, by default a closed-source archive will be created.

\subsection{Reproducible builds}

Archives try to follow recommendations from \url{https://reproducible-builds.org/docs/archives/},
that is, sorted file order, no user metadata, deterministic timestamps
(one per day). Still, the version of \verb+autoconf+ used and other factors
may affect the generated archive.
