\chapter{The Build Stage}

The procedure for creating the source distribution.

\section{Prerequisites}

\begin{itemize}
\item All tools needed to compile Frama-C (that you should have anyways)
\item \texttt{bash} v4.0 or higher
\item \texttt{git-lfs}
\item GNU \texttt{parallel}
\item a \TeX distribution (e.g. \TeX{}live),
including notably the following utilities:
\begin{itemize}
\item \texttt{latexmk}
\item (recommended) \texttt{texfot}
\end{itemize}
\end{itemize}

\section{Final checks}

\begin{itemize}

\item Check plug-in dependencies in all \texttt{configure.*}, in case they have
  changed: if you don't know them, ask plug-in developers to verify them.

\todo{should be done when the plugin is modified}

\item Check the headers and corresponding licenses files for new files
  \todo{should be done when the plugin is modified}

%  (\texttt{make check-headers} is your friend, but it is done by CI anyway).


\item Check the contents of \texttt{INSTALL.md} \todo{Should always be up to date}
  \begin{itemize}
    \item Update the Frama-C version number in the `Reference configuration` section
    \item In particular, check and update a full known good configuration for
    opam packages, including alt-ergo, why3, coq, etc.
    Use \verb+bin/check-reference-configuration.sh+ to help check if the
    configuration is ok. You can also try running
    \verb+make wp-qualif+.
  \end{itemize}
\item Check the contents of \texttt{README.md} \todo{Should always be up to date}
\item Check the contents of \texttt{Changelog}
  \todo{Should always be up to date}
\item Carefully read the output of the configure to be sure that there are no
  buggy messages. Make sure you have required OPAM packages installed in your machine,
  e.g. alt-ergo.
\end{itemize}

\section{Update the Sources}\label{sec:update-sources}

There are many administrative steps, coordinated by the release manager.
\begin{enumerate}
\item [{\em Non-beta only}] If a new version of the ACSL manual is to be released, make sure you also
  have a clone of the \texttt{acsl} manual Github repository
  (\url{git@github.com:acsl-language/acsl.git}) in the \texttt{doc}
  directory
\item Update files \texttt{VERSION} and \texttt{VERSION\_CODENAME}
  (for beta releases, add suffix \texttt{\textasciitilde{}beta})
\item [{\em Non-beta only}] Update file \texttt{ALL\_VERSIONS}. \todo{This section must be put somewhere else in the release manual}
\item Update files
  \texttt{src/plugins/wp/Changelog} and
  \texttt{src/plugins/e-acsl/doc/Changelog},
  to add the header corresponding to the
  new version. % For the final release, use the script
  % \texttt{doc/changelog/generate} to check that the HTML page can be built,
  % and check its contents.
\item Update the Frama-C's authors list in files
  \texttt{src/plugins/gui/help\_manager.ml} and \texttt{opam/opam}.
  Traditionally, an author is added if they have contributed 100+ LoC in
  the release, as counted by:
  \begin{verbatim}
  git ls-files -z | \
  parallel -0 -n1 git blame --line-porcelain | \
  sed -n 's/^author //p' | sort -f | uniq -ic | sort -nr
  \end{verbatim}
  (source for the command: \url{https://gist.github.com/amitchhajer/4461043})
\item Recompile and test to check that all is fine.
\item Check the documentation as per Section~\ref{update_doc}
\item update the \texttt{opam} directory: update the version name and
  version number in the fields \texttt{version}, \texttt{doc} and
  \texttt{depends} in \texttt{opam/opam}. \textbf{NB:} for a beta version, be sure
  to use a tilde \textasciitilde{} between the version number and \texttt{beta},
  and not a dash \texttt{-}.
\item Create the last commit
\item Create the tag using \texttt{git tag \$(cat VERSION | sed -e "s/~/-/")} and push it (e.g. via \texttt{git push origin \$(cat VERSION | sed -e "s/~/-/")})
\end{enumerate}

\section{Update the Documentation}\label{update_doc}

\subsection{Manuals}

Most manuals depend on the \texttt{VERSION} file, so make sure to regenerate
them when this file changes.
Also, most of the manuals include an appendix with the list of
changes that have happened, divided by version of Frama-C. Make sure that the
first section has the correct release name.

\todo{Mention the other things that change from one release to the
  other, such as the ACSL version number}

\todo{The release manager should create fresh Changes sections in \texttt{master}. But When?: at fork time, or when the release is ready?}


The official \FramaC documents relative to the kernel and ACSL to be released
must be recompiled in order to take into account the \texttt{VERSION}
file:

\begin{center}
  \begin{tabular}{lll}
    \hline
    \textsf{User Manual} &
    \texttt{doc/userman} &
    \expertise{Julien} \\
    \textsf{Developer Manual} &
    \texttt{doc/developer} &
    \expertise{Julien} \\
    \textsf{ACSL Implementation}  &
    \texttt{doc/acsl} (from Github) &
    \expertise{Virgile} \\
    % \textsf{man page} &
    % \texttt{man/frama-c.1} &
    % \expertise{Virgile} \\
    \hline
  \end{tabular}
\end{center}

Documents relative to the plug-ins can be found at the same place. This list
isn't exhaustive:

\begin{center}
  \begin{tabular}{lll}
    \hline
    \textsf{RTE}  &
    \texttt{doc/rte} &
    \expertise{Julien} \\
    \textsf{Aoraï}  &
    \texttt{doc/aorai} &
    \expertise{Virgile} \\
    \textsf{Metrics}  &
    \texttt{doc/metrics} &
    \expertise{André} \\
    \textsf{Value Analysis}  &
    \texttt{doc/value} &
    \expertise{David} \\
    \textsf{WP}  &
    \texttt{src/plugins/wp/doc/manual/} &
    \expertise{Loïc} \\
  \end{tabular}
\end{center}

Clone the \texttt{acsl} manual
(\url{git@github.com:acsl-language/acsl.git}) in
\texttt{doc/acsl} in order to generate the ACSL reference and implementation
manuals.

% manuals (which uses git-annex) are now deprecated in favor of using the
% Github wiki
%Clone the \texttt{manuals} repository
%(\url{git@git.frama-c.com:frama-c/manuals.git}) in
%\texttt{doc/manuals} in order to update the manuals (required by the
%\texttt{install} target below).

In the \texttt{doc} directory, just run \texttt{make all} to compile
and install all manuals, together with some final coherence checks with
respect to the current Frama-C implementation (notably
for the developer manual and ACSL implementation manual).

% No longer use git annex (as recommended by VP)
%\paragraph{Getting the manuals on the web (ultimately)}
%
% In \texttt{doc/manuals} after installing all the manuals:
%\begin{shell}
%git annex add .
%git annex copy . --to origin
%git commit -am "manuals for release ...."
%git push
%\end{shell}

\subsection{API documentation}

To update the \textsf{API} documentation, you must update the
\texttt{@since} tags. This involves the following script (\texttt{sed -i} must
work for this purpose):
\begin{shell}
./bin/update_api_doc.sh <NEXT>
\end{shell}
\verb+<NEXT>+ is the long name of the final release, i.e.
\verb+`cat VERSION`-`VERSION_CODENAME`+
(without the \verb+beta+ suffix, if any).

Check that no \verb|+dev| suffix remains inside comments:

\begin{shell}
git grep +dev src
\end{shell}

Test by recompiling Frama-C and rebuilding the API documentation:
\begin{shell}
make -j byte gui-byte
make doc # or just doc-kernel
\end{shell}

\todo{What is this step:
check consistency of API documentation. (the plug-in development guide
  (\texttt{make check-devguide}) doesn't work anymore)}

\section{Build the Source Distribution}
\label{sec:build-source-distr}

This steps creates the tarball of Frama-C, the tarball of the API and
copy them to the website. It also copies the manuals.

\textbf{Note}: all standard plugins must be enabled to ensure they are distributed
in the \verb+.tar.gz+. Some plug-ins require optional dependencies
(e.g. markdown-report and \verb+ppx_deriving+), so you must ensure that,
when running \verb+./configure+, no plugins that should be distributed are missing.
Some of the checks are performed by the release build script, but not all of them.
Also, some plug-ins may have components depending on optional packages
(e.g. Server and \verb+zmq+), which are not always thoroughly tested. If such
plug-ins are only partially enabled, check that the non-compiled sources are
distributed in any case, or report it as a bug to their developers.

If you have never run \verb+make src-distrib+, you can try a standalone build
of the \verb+.tar.gz+ itself, before doing the rest (don't forget to use
\verb+OPEN_SOURCE=yes+).
Refer to Appendix \ref{chap:make-src-distrib} if you have problems when
doing so. In any case, the archive will be recreated by a script, so you can
skip this step.

You need to have installed \texttt{git-lfs}, otherwise the build
script will not correctly use lfs when putting large files in the
website git.

\expertise{release manager} Use the script
\texttt{build-src-distrib.sh} for this purpose (\texttt{bash version
  4} required).

The script will search for the following repositories:

\begin{itemize}
  \item \texttt{./doc/acsl} (\texttt{git@github.com:acsl-language/acsl.git})
  \item \texttt{./frama-c.wiki} (\texttt{git@git.frama-c.com:pub/frama-c.wiki.git})
  \item \texttt{./website} (\texttt{git@git.frama-c.com:pub/pub.frama-c.com.git})
\end{itemize}

If they are not available, they can be cloned by the script.

The following steps will be performed by the script:

\begin{itemize}
  \item compile manuals
  \item build source distribution
  \item build the API bundle
  \item build the documentation companions
  \item update locally the wiki pages
  \item create a new branch for the website
  \item run some sanity checks on the source distribution
  \item generate the \texttt{opam} file
\end{itemize}

In order to create "ready to deploy" wiki and website pages, a file \texttt{main\_changes.md}
must be provided. The expected format is:

\begin{lstlisting}
# Kernel

- item
- ...

# <Plugin-Name> (Prefer alphabetic order)

- item
- ...

# ...

\end{lstlisting}

The performed sanity checks are:

\begin{itemize}
  \item verification that no non-free files are distributed,
  \item verification that no non-free plugins are distributed,
  \item no \texttt{/home/user} path can be found in the distribution,
  \item the source code can be compiled and tests succeed.
\end{itemize}

Note that there are some interactive steps in the script, that ask for user
approval before going further.

Finally, ensure that locale \verb+en_US.UTF-8+ is available on your system,
as it is used by the script to ensure a proper encoding of the generated files.

Now, run the script:
\begin{shell}
./bin/build-src-distrib.sh
\end{shell}

The generated files are available in the \texttt{./distributed} repository.
After testing more extensively the distribution, the following actions should
be performed:

\begin{itemize}
  \item push stable branch on the public repository
  \item push stable tag on the public repository
  \item push the local Frama-C wiki branch to the public repository
  \item push the generated website branch
\end{itemize}

\section{Testing the Distribution}

Please check that the distribution (sources and API) is OK:
\begin{enumerate}
\item check for possible path/environment leaks in the \texttt{tar.gz}:
  grep for the path where your Frama-C git directory is, e.g.
  \texttt{/home/user/git/frama-c}. It should appear nowhere in the archive.
\item check that \texttt{./configure \&\& make -j \&\& sudo make install} works fine
  (or check e.g. \texttt{./configure --prefix=\$PWD/build \&\& make -j \&\& make install});
\item Alternatively, you can use \texttt{docker} to compile the archive against a
  precise configuration:
  \begin{itemize}
    \item \verb+cp distributed/frama-c-<VERSION>.tar.gz developer_tools/docker+
    \item \verb+cd developer_tools/docker+
    \item \verb+make Dockerfile.dev+
    \item \verb+docker build . -t framac/frama-c:dev --target frama-c-gui-slim \+\\
          \verb+  -f Dockerfile.dev --build-arg=from_archive=frama-c-<VERSION>.tar.gz+
    \end{itemize}
\item test the installed binaries (especially the GUI). (On Linux, OCI tests
  everything but the GUI); If you've taken the \texttt{docker} route, you might
  want to install the \href{https://github.com/mviereck/x11docker}{\texttt{x11docker}} script,
  in order to be able to launch
  \verb+x11docker framac/frama-c:dev frama-c-gui+
\item redo the two steps above on Windows/WSL \expertise{André}\expertise{Allan},
  macOS \expertise{André}\expertise{Loïc}\\\expertise{Thibaud};
  \begin{itemize}
  \item For instance, you can compile and test as follows:
  \item \verb+./configure --prefix="$PWD/build"+
  \item \verb+make+;
  \item \verb+make install+;
  \item \verb+rm -f ~/.why3.conf; why3 config detect+;
  \item \verb+build/bin/frama-c tests/idct/*.c -eva -wp -wp-rte+
  \item If you have a GUI, replace \verb+frama-c+ above with \verb+frama-c-gui+
    and click around, see if things work (e.g. no segmentation faults).
  \end{itemize}
\item have a look at the documentation of the API (untar, open
  \texttt{index.html} in a browser, click on a few links, etc).
  If there are minor bugs/typos, note them for later, but it's not
  worth delaying the release to fix them.
\item consider decompressing both the current and the previous release archives
  in different directories, then using \texttt{meld} to compare them. This
  allows spotting if some files were unexpectedly added, for instance.
\end{enumerate}


%%%Local Variables:
%%%TeX-master: "release"
%%%End:
