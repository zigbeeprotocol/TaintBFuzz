\chapter{The Branch Stage}

That is the procedure for forking the release from \texttt{master}.

\section{``Freezing'' master}

When the ``freeze'' period arrives (usually a few weeks before the beta
release), the release manager may decide to prevent direct merges to the
\texttt{master} branch, until the \texttt{stable} branch is created.
If so, then the branch should be protected such that Developers can no
longer push and merge to it, without asking for a Master to do it.

In Gitlab, this is done via: Frama-C project $\rightarrow$ Settings
$\rightarrow$ Repository $\rightarrow$ Protected Branches.

\textbf{The \texttt{master} branch must be protected \emph{at all times}.
  Parameters
  \texttt{Allowed to merge} and \texttt{Allowed to push} must be set to
  \texttt{Developers + Maintainers} during non-freeze times, and to
  \texttt{Maintainers} during freeze time.}

\section{Creating the branch}

Note that you must be member of the GitLab groups "frama-c", "dev" and have
access to each plugin of continuous integration.

Create the branch \texttt{stable/release}, where \texttt{release} is the
element name, using the \texttt{frama-ci} tool:
\begin{enumerate}
\item Install \texttt{frama-ci} tools:
\begin{shell}
opam pin add frama-ci-tools git@git.frama-c.com:frama-c/Frama-CI.git
\end{shell}
\item Create an API token for gitlab, in your gitlab profile settings.
\item Run the command
\begin{shell}
frama-ci-create-branch --token=\$TOKEN \
--name=stable/release --default-branch
\end{shell}
This command creates a branch \texttt{stable/release} for frama-c and for
each plugin tested by the CI — and configures the CI to use these branches
by default when testing fixes for the release.
These branches are directly created on the gitlab server.
\end{enumerate}
What can be committed in this branch must follow the release schedule,
and go through Merge-requests. Everything else should go in \texttt{master},
which can then be reset to standard-level protection (Developers + Maintainers
allowed to push/merge).

\section{Creating the milestones}

Create the milestone for the next release on \textsf{Gitlab},
in the Frama-C group
\expertise{François, Julien}.


\section{BTS}

Check public issue tracker at \url{https://git.frama-c.com/pub/frama-c/issues/} All issues should have been at least acknowledged: at least they should be assigned to someone, preferably tagged appropriately.

Send a message to the Frama-C channel on LSL's Mattermost.
Each developer should have a
look at their own assigned/monitored/reported issues. Depending on their severity,
the issues should be tagged with the current release, or with the next one.

\section{Changelog}

Create headers in the Changelog(s). This is to avoid some merge conflicts,
but also so that merge requests add information in the proper part of
the Changelog.

\begin{itemize}
\item Add the following in the Changelogs, in \texttt{stable/element}
\begin{verbatim}
 ####################################
 Open Source Release <nb> (<element>)
 ####################################
\end{verbatim}

This should go directly below
\begin{verbatim}
 ##################################
 Open Source Release <next-release>
 ##################################
\end{verbatim}

The changelogs to update are the global Frama-C changelog and each plugin
changelog in \texttt{src/plugins}.

\item Check the manuals if they contain a changelog, usually in files named
\texttt{changes.tex} or a variation of it. Add the following in
\texttt{stable/element}
\begin{verbatim}
\section*{<nb> (<element>)}
\end{verbatim}

This should go directly below
\begin{verbatim}
\section*{\framacversion}
\end{verbatim}

and this line should be commented out.

\begin{itemize}
  \item \emph{Take care to adapt the format of the version according to the
        content of the file}.
  \item \emph{Check for manual changelogs in \texttt{doc/}, but also in
        \texttt{src/plugins/**/doc}}.
\end{itemize}

\item Merge this branch in \texttt{master}
\end{itemize}

From now on, Changelog items corresponding to MR merged into \texttt{master}
must be placed between these two banners. This should minimize conflicts when
merging back \texttt{stable/element} into \texttt{master}.

\todo{Can we simplify this?}

\section{Copyright}

Check that the date in copyright headers is correct. If not then the script
\texttt{headers/updates-headers.sh} can be used to update them.

\begin{itemize}
  \item Update the dates in the license files in:
  \begin{itemize}
    \item \texttt{headers/close-source/*}
    \item \texttt{headers/open-source/*}
    \item \texttt{ivette/headers/close-source/*}
    \item \texttt{ivette/headers/open-source/*}
    \item \texttt{src/plugins/e-acsl/headers/close-source/*}
    \item \texttt{src/plugins/e-acsl/headers/open-source/*}
  \end{itemize}
  \item Update the headers with the following commands:
  \begin{itemize}
    \item \texttt{./headers/updates-headers.sh headers/header\_spec.txt headers/open-source}
    \item \texttt{./headers/updates-headers.sh ivette/headers/header\_spec.txt headers/open-source ivette}
    \item \texttt{./headers/updates-headers.sh src/plugins/e-acsl/headers/header\_spec.txt src/plugins/e-acsl/headers/open-source src/plugins/e-acsl}
  \end{itemize}
  \item Check if some copyright are left to update by \texttt{grep}-ing the date in the sources: \texttt{grep -r -e "-yyyy" .}
\end{itemize}

%%%Local Variables:
%%%TeX-master: "release"
%%%End:
