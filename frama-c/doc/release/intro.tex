\chapter{Introduction}

This manual is intended for driving the release process of \FramaC
distribution. It covers several topics: people involved in the release effort,
repository resources and web site maintenance. The last chapter provides the
ultimate procedure for releasing a new distribution of \FramaC.

\section{Roles}

Some individuals have particular expertises in some fields: they are indicated
like \expertise{this} whenever appropriated.

The members of \FramaC team involved in the release effort play different
roles, and we must distinguish between:

\begin{description}
\item[Developers:] including the kernel and plug-ins developers. They
  are responsible for all the source files in the repository, and the
  associated documentation.
%% \item[Binary Builders:] they are responsible for creating a binary distribution
%%   of the release for a specific architecture \expertise{none currently}.
\item[Web Site Maintainers:] they are responsible for updating the web site,
  during the release and for possible later updates \expertise{Florent ? Thibaud ?}.
\item[Release Manager:] they are responsible for the organisation of the
  release process
\end{description}

\section{Release Overview}

A \FramaC release consists of the following main steps:
\begin{enumerate}

\item \textbf{The branch stage.} A branch is created, in which the release
  will be prepared. It will ultimately become the released version.
  The development of unstable features may continue in master, while the
  branch is dedicated to the ongoing release.

\item \textbf{The build stage.} The source files are setup by the
  developers, and the bug tracking system is updated to reflect the
  distribution state. A source distribution is created and registered,
  with its updated documentation.

\item \textbf{The web stage.} A snap-shot of the complete web site is
  created, by merging the new release files with the old,
  already distributed ones. %% Some old data might be dropped away if
  %% necessary, although they are still committed somewhere in the
  %% repository.

\item \textbf{The go-online stage}. The web-site snapshot is put on
  the web. Time to notify the world!

\end{enumerate}



%%%Local Variables:
%%%TeX-master: "release"
%%%End:
