\chapter{The Web Stage}

Where all our efforts goes on the web.  There are two very different
tasks for \FramaC to go online.  Authoring the web site pages is the
responsibility of the developers and the release manager. Publishing
the web site can only be performed by authorized people, who may not
be the release manager.

\section{Requirements}

The website \texttt{README.md} provides a quick guide to locally setup
the website.

\section{Generate website}

Before pushing the branch to the website, it can be deployed locally but
this is not absolutely necessary.

Push the branch generated by the release script on the website repository.

You can have a look at the generated pages on \texttt{pub.frama-c.com} (after
the \texttt{deploy} target of the website's continuous integration has successfully run.
This can take a few minutes if the server is loaded).
Note however that the download links won't work as they are available only
once the branch is merged (\texttt{download} links are handled by Git-LFS).

Check the following pages:

\begin{itemize}
  \item \texttt{index.html} must display:
  \begin{itemize}
    \item the new event,
    \item a link to the (beta) release at the bottom.
  \end{itemize}
  \item \texttt{/fc-versions/<version\_name>.html}:
  \begin{itemize}
    \item check Changelog link,
    \item check manual links (reminder: the links are dead at this moment), it must contain \texttt{NN.N-Version}
    \item check ACSL version.
  \end{itemize}
  \item \texttt{/html/changelog.html\#Version-NN.N}
  \item \texttt{/html/acsl.html}: check ACSL versions list
  \item \texttt{rss.xml}: check last event
\end{itemize}

For a beta version, the installation pages for:
\begin{itemize}
\item \texttt{/html/installations/beta\_version\_name.html} should indicate the beta status
\item \texttt{/html/installations/current\_version\_name.html} should not indicate anything
\item \texttt{/html/installations/previous\_version\_name.html} should indicate older version
\end{itemize}

For a final version, the installation pages for:
\begin{itemize}
\item \texttt{/html/installations/version\_name.html} should not indicate anything
\item \texttt{/html/installations/previous\_version\_name.html} should indicate older version
\end{itemize}

On GitLab, the following files must appear as \textbf{new} in \texttt{download}.
Note that for beta versions, if you have not included all manuals, not all of
them will appear:

\begin{itemize}
  \item \texttt{acsl-X.XX.pdf}
  \item \texttt{acsl-implementation-NN.N-CODENAME.pdf}
  \item \texttt{plugin-development-guide-NN.N-CODENAME.pdf}
  \item \texttt{user-manual-NN.N-CODENAME.pdf}
  \item \texttt{<plugin>-manual-NN.N-CODENAME.pdf}\\
        for Aorai, EVA, Metrics, RTE and WP
  \item \texttt{e-acsl/e-acsl-X.XX.pdf}
  \item \texttt{e-acsl/e-acsl-implementation-NN.N-CODENAME.pdf}
  \item \texttt{e-acsl/e-acsl-manual-NN.N-CODENAME.pdf}
  \item \texttt{aorai-example-NN.N-CODENAME.tgz}
  \item \texttt{frama-c-NN.N-CODENAME.tar.gz}
  \item \texttt{frama-c-api-NN.N-CODENAME.tar.gz}
  \item \texttt{hello-NN.N-CODENAME.tar.gz}
\end{itemize}

For final release \textbf{ONLY}, the following files must appear as \textbf{modified} in \texttt{download}:

\begin{itemize}
  \item \texttt{acsl.pdf}
  \item \texttt{frama-c-acsl-implementation.pdf}
  \item \texttt{frama-c-plugin-development-guide.pdf}
  \item \texttt{frama-c-user-manual.pdf}
  \item \texttt{frama-c-<plugin>-manual.pdf}\\
        for Aorai, EVA, Metrics, RTE, \textbf{Value Analysis} and WP
  \item \texttt{e-acsl/e-acsl.pdf}
  \item \texttt{e-acsl/e-acsl-implementation.pdf}
  \item \texttt{e-acsl/e-acsl-manual.pdf}
  \item \texttt{frama-c-aorai-example.tgz}
  \item \texttt{hello.tar.gz}
\end{itemize}

\section{Announcements}

\begin{itemize}
\item Send an e-mail to \texttt{frama-c-discuss} announcing the release.
\item Tweet the release, pointing to the Downloads page.
\item Ideally, a blog post should arrive in a few days, with some interesting
  new features.
\end{itemize}

\section{Opam package}

You'll need a GitHub account to create a pull request on the official opam repository,
\texttt{ocaml/opam-repository.git}.

\begin{itemize}
\item Clone \texttt{opam-repository}: \texttt{git clone git@github.com:ocaml/opam-repository.git}
\item Make sure you are on \texttt{master} and your branch is up-to-date
\item Create a new directory: \\
  \texttt{packages/frama-c/frama-c.<version>}
\item Copy the file \texttt{./distributed/opam} to the opam repository clone in: \\
 \texttt{packages/frama-c/frama-c.<version>/opam}
\item You can provide \verb|sha256| and/or \verb|sha512| checksums at the end of this file if ou wish.
\item (optional) Check locally that everything is fine:
\begin{verbatim}
opam switch create local <some-ocaml-compiler-version>
opam repository add local <path-to-repository-clone>
opam repository set-repos local
opam install frama-c
\end{verbatim}
(of course, if you already create a local switch before and it uses your
local version of the repository, you just have to switch to it).

\textbf{Note:} uncommitted changes are ignored by \texttt{opam repository};
you have to locally commit them before \texttt{opam update} will take them into
account.

\item Create a branch with any name you want (e.g. frama-c.<version>) and push it to your remote Github
\item Create a pull request to opam-repository. If all tests pass,
  someone from opam should merge it for you.
  \textbf{Note:} some opam tests may fail due to external circumstances; if the
  error log makes no sense to you, wait to see if someone will contact you
  about it, or ask directly on the merge request.
\end{itemize}

\section{Updating pub/frama-c}

You'll need to be member of the \texttt{pub/frama-c} project to be able to
commit to it.

\begin{itemize}
\item add the \texttt{pub/frama-c} remote to your Git clone;
\item make sure the last commit is tagged (either a release candidate, or a
  final release);
\item push the stable/\texttt{<codename>} branch to the \texttt{pub} remote.
\end{itemize}

({\em Non-beta only}) After pushing the tag to Gitlab, go to the Releases page
and create a release for the tag.

You should also push the wiki changes. Note that all files listed as \textbf{new}
in the website section should appear in the wiki but:

\begin{itemize}
  \item \texttt{frama-c-api-NN.N-CODENAME.tar.gz}
  \item \texttt{hello-NN.N-CODENAME.tar.gz}
\end{itemize}

and E-ACSL files are not in a sub-folder \texttt{e-acsl} on the wiki.

\section{Other repositories to update}

Check if other Frama-C (and related) repositories need to be updated:

\begin{itemize}
\item \texttt{acsl-language/acsl} (if last minute patches were applied)
\item \texttt{pub/open-source-case-studies}
\item \texttt{pub/sate-6}
\item other \texttt{pub} repositories related to Frama-C...
\end{itemize}

\section{Preparing the Next Release}

Just update the \texttt{VERSION} file in \texttt{master}, by adding
\texttt{"+dev"}. Do not add any newline at the end of the
\texttt{VERSION} file.

\section{Docker image preparation}

\textbf{Note:} you need access to the \texttt{framac} Docker Hub account to be able to upload the image.

Make sure that \texttt{devel\_tools/docker/Makefile} is up-to-date:

\begin{itemize}
\item changes to \texttt{reference-configuration.md} must have been ported
  to the corresponding \texttt{Dockerfile.dev} entry inside the
  \texttt{Makefile}.
\end{itemize}

Copy the \texttt{.tar.gz} archive to the \texttt{devel\_tools/docker} directory.

Run \texttt{make release}. It should decompress the archive, build and test
the Frama-C Docker image.

If the local tests do not work, check that the OCaml version and package
dependencies are all up-to-date.

If the image is built succesfully, you can also try building the GUI image
(\texttt{make release-gui}) and the stripped image
(\texttt{make release-stripped}).

If you want to upload these images to the Docker Hub, you can re-tag them, e.g.

\begin{lstlisting}
docker tag framac/frama-c:release framac/frama-c:<VERSION>
\end{lstlisting}

Where \texttt{<VERSION>} is the release number, possibly with a suffix, but
{\em without} characters such as \texttt{+}. For instance, you can use
\texttt{23.1-beta} for a beta release.

Then upload the renamed image(s) with:

\begin{lstlisting}
docker push framac/frama-c:<VERSION>
docker push framac/frama-c-gui:<VERSION>
docker push framac/frama-c:<VERSION>-stripped
\end{lstlisting}
You will need to have setup your \texttt{framac} Docker Hub account for this to work.

%%% Local Variables:
%%% mode: latex
%%% TeX-master: "release"
%%% End:
