\chapter{Utilisation}\label{sec-man}

Dans ce chapitre essaye de présenter quelques bases qui permettent de tester
la boîte à outil de \slicing à partir de l'interpréteur ou
d'un fichier de commande.
Pour l'utilisation de l'interface graphique,
 se reporter à la documentation correspondante.
Il est également possible d'exécuter certaines actions élémentaires
en ligne de commande. Voir les options proposées à l'aide de~:
\begin{center}
\verb!bin/toplevel.top --help!
\end{center}

\section{Utilisation interactive}

Lorsqu'on lance l'outil pour faire du slicing, il faut lui fournir le ou les
fichiers à analyser, et au minimum l'option \verbtt{-deps}
qui permet d'effectuer l'analyse de
dépendances (utiliser \verbtt{--help} pour voir les autres options)~:
\begin{center}
\verb!bin/toplevel.top -deps fichier.c!
\end{center}

On se retrouve alors avec un prompt \verb!#!
qui signale que l'on est sous un interpréteur \caml.\\

Pour quitter l'interpréteur, il faut utiliser la commande~:
\begin{center}
\verb!#quit;;!
\end{center}

(le \verb!#! n'est pas le prompt, il fait parti de la commande).
On peut aussi utiliser \verbtt{Ctrl-D}.

\begin{astuce}\label{astuce-ledit}
Comme tout interpréteur \caml, il n'a pas de capacité d'éditer la ligne de
commande (pour corriger une faute de frappe par exemple), ni d'historique (pour
rappeler une commande précédente). Il est donc fortement conseillé, même si ce
n'est pas indispensable, d'installer
\indextxt{\verbtt{ledit}}, que l'on peut obtenir à l'adresse~: \\
\url{ftp://ftp.inria.fr/INRIA/Projects/cristal/Daniel.de\_Rauglaudre/Tools/}.
Il suffit ensuite de faire précéder le nom de la commande par \verbtt{ledit}, tout
simplement.
Faire \verbtt{man ledit} pour avoir plus d'information sur les possibilités de cet
outil.
\end{astuce}

Sous l'interpréteur, on peut utiliser n'importe quelle instruction \caml,
et toutes les commandes spécifiques à l'outil.

Pour ceux qui ne sont pas familier avec \caml,
pour lancer une commande, il faut taper son nom suivie des arguments,
ou de parenthèses vide s'il n'y a pas d'arguments, et terminer par deux
points-virgules.

L'interpréteur affiche le type de retour de la commande
(\verbtt{unit} correspond au type vide),
suivi éventuellement de la valeur retournée.
Dans la plupart des cas, la première information n'intéresse pas l'utilisateur,
mais cet affichage ne peut pas être désactivé.
Il peut parfois être utile pour retrouver la signature des commandes<
par exemple~:

\centerline{\tt module S = Db.Slicing;;}

affiche le type du nouveau module S, et permet donc de voir la liste des
commandes du module \verbtt{Db.Slicing}.\\

L'organisation de \ppc regroupe toutes les commandes dans le module \verbtt{Db}.
Le sous-module \verbtt{Db.Slicing} contient les commandes de \slicing.
Le nom d'une commande doit normalement être préfixé par le nom du
module a qui elle appartient.
Pour pouvoir utiliser les noms sans
préfixe, il faut ouvrir le module à l'aide de~:

\centerline{\tt open Db;;}

On peut aussi choisir de renommer le module pour avoir un préfixe plus court~:

\centerline{\tt module S = Db.Slicing;;}

Des commandes provenant d'autres modules sont également utiles
dans le cadre de l'utilisation du \slicing.
\verbtt{Kui}, en particulier,
fournit des fonctions qui permettent de naviguer dans les
informations générées par les autres analyseurs.

\section{Fichier de commandes}

Les commandes peuvent être tapées directement, mais on peut également charger
des fichiers contenant les commandes grâce à la commande \verb!#use "cmds.ml";;!.
Le contenu du fichier \verbtt{cmds.ml} est alors interprété comme si on l'avait
tapé, et on reprend la main sous l'interpréteur à l'issue de son exécution.\\

Si l'on veut simplement lancer un fichier de commandes,
on peut utiliser la redirection Unix.
L'interpréteur se terminera alors à la rencontre du EOF (fin de fichier).

\begin{astuce}\label{astuce-jnl-ledit}
En utilisant \verbtt{ledit} (voir l'astuce \vpageref{astuce-ledit})
on obtient également la possibilité de mémoriser
l'historique, et d'avoir donc une journalisation des commandes lancées (très
pratique comme base pour écrire un script, ou pour rejouer une séquence).
Pour cela, il suffit de taper~:

\centerline{{\tt ledit -h cmds.ml bin/toplevel.top} ... arguments...}
\end{astuce}

\section{Exemples}

Quelques exemples d'utilisation peuvent être trouvées dans le répertoire de
test. Les deux fichiers suivant fournissent quelques fonctions qui facilitent
l'utilisation~:
\begin{itemize}
  \item \verbtt{libSelect.ml} propose une interface simplifiée pour des sélections
    simples,
  \item \verbtt{libAnim.ml} permet de générer des représentations graphiques du
    projet à différentes étapes de sa construction afin de visualiser
    les relations entre les spécialisations.
\end{itemize}
