
\chapter{Détail des actions} \label{app-actions}

Ce chapitre présente une fiche signalétique pour chaque action du calcul
interprocédurale présentées au chapitre \ref{sec-interproc}.
Elle correspondent aux commandes telles qu'elles ont été spécifiées,
et il peut y avoir quelques différences par rapport à ce qui a été implémenté,
mais il s'agit de différences mineures (principalement les noms).
Pour plus de détails, le lecteur est invité à consulter la document du code
qui sera toujours la plus à jour.\\

\newcommand{\proto}[2]{
  \section{#1}\labact{#1}{2}
  \centerline{\textit{#1}(#2)}
  \bb
  }
\newcommand{\precond}{{\bf Précondition}}
\newcommand{\param}{{\bf Paramètres}}
\newcommand{\creable}{{\bf Création par l'utilisateur}}
\newcommand{\modifiable}{{\bf Modifiable par l'utilisateur}}
\newcommand{\generable}{{\bf Génération automatique}}
\newcommand{\application}{{\bf Application}}
\newcommand{\genere}{{\bf Génération}}
\newcommand{\modifie}{{\bf Modifications}}

Les indications suivantes sont données pour chaque action~:
\begin{itemize}
  \item \param~: indique le sens des paramètres de l'action,
  \item \precond~: indique les conditions éventuelles de création et/ou
    d'application,
  \item \creable~: indique si l'utilisateur peut la créer,
  \item \modifiable~: indique si l'utilisateur peut la modifier lorsqu'elle est
    dans la liste d'attente,
  \item \generable~: indique si cette action peut être générée par l'outil,
    et précise dans quels cas,
  \item \application~: détaille ce qu'il se passe quand cette action est
    appliquée,
  \item \genere~: donne la liste des actions pouvant être générées,
  \item \modifie~: indique si une spécialisation peut être crée ou modifiée.
\end{itemize}

\proto{NewSlice}{$f_0$}

\begin{itemize}
  \item \param~: $f_0$ est la fonction source pour laquelle on souhaite créer
    une nouvelle spécialisation.
  \item \precond~: néant.
  \item \creable~: oui.
  \item \modifiable~: oui (suppression).
  \item \generable~: oui, par \actChooseCall.
  \item \application~: 
    crée une nouvelle spécialisation $f_i$, initialement sans aucun marquage
    (tout invisible).
  \item \genere~: néant.
  \item \modifie~: création d'une nouvelle $f_i$.
\end{itemize}

\proto{AddUserMark}{$f_i, (e,m)$}

\begin{itemize}
  \item \param~: $f_i$ est la fonction dont on veut modifier le marquage,
    $e$ un élément à marquer, $m$ la marque à lui ajouter.
    Différentes version de cette action peut être proposée pour faciliter la
    désignation de $e$ et le choix de la marque $m$.
  \item \precond~: $f_i$ doit exister et $e$ désigner un élément valide de $f$.
  \item \creable~: oui.
  \item \modifiable~: oui (suppression).
  \item \generable~: non.
  \item \application~: ajoute $m$ à la marque $m1$ de $e$ dans $f_i$
    et propage dans les dépendances.
  \item \genere~: $\actExamineCalls$.
  \item \modifie~: modification de $f_i$.
\end{itemize}

\proto{ExamineCalls}{$f_i$}

\begin{itemize}
  \item \param~: $f_i$ est la fonction dont le marquage a été modifié
    et pour laquelle il faut vérifier les appels.
  \item \precond~: $f_i$ doit exister.
  \item \creable~: non.
  \item \modifiable~: non.
  \item \generable~: oui.
  \item \application~: 
    vérifie la cohérence du marquage de chaque appel de fonction $c$
    de \call(f) et également des éventuelles appels à $f_i$.
  \item \genere~: 
    $\actChooseCall$ et/ou $\actMissingOutputs$ si le marquage de
    certains appels de fonction a été modifié, et $\actMissingInputs$ si
    $f_i$ est appelée et que les entrées de certains appels sont insuffisamment
    marquées.
  \item \modifie~: rien.
\end{itemize}


\proto{ChooseCall}{$c, f_i$}

\begin{itemize}
  \item \param~: appel $c$ dans la fonction spécialisée $f_i$.
  \item \precond~: néant.
  \item \creable~: non.
  \item \modifiable~: oui, elle peut être remplacée par un $\actChangeCall(c,
    f_i, g_j)$ (voir les conditions de création de cette action).
  \item \generable~: oui, quand le marquage de $f_i$ est modifié et que 
    l'appel $c$ devient visible, cette action est générée pour lui
    attribuer une fonction à appeler.
  \item \application~: détermine la fonction à appeler en tenant compte du mode
    de fonctionnement.
  \item \genere~: \actChangeCall{} et éventuellement \actNewSlice{} et
    \actAddOutputMarks{}.
  \item \modifie~: rien.
\end{itemize}

\proto{ChangeCall}{$c, f_i, g_j$}

\begin{itemize}
  \item \param~: on considère l'appel $c$ de $f_i$, $g_j$ est la fonction à
    appeler.
  \item \precond~: $\Call(f_0)(c)=g_0$ et la signature de sortie de $g_j$
    doit être compatible avec $\sigc(c, f_i)$.
  \item \creable~: oui.
  \item \modifiable~: oui, mais seulement s'il l'a initialement créée.
  \item \generable~: oui, par un \actChooseCall.
  \item \application~: $\Call(f_i)(c)=g_j$ et les marques des entrées de $g_j$
    sont propagées dans $f_i$ ($\actModifCallInputs(c, f_i)$).
  \item \genere~: l'action \actModifCallInputs{} est directement
    appliquée, mais elle peut générer d'autres actions (voir
    \refact{ModifCallInputs}{2}).
  \item \modifie~: $f_i$.
\end{itemize}


\proto{MissingInputs}{$c, f_i$}

\begin{itemize}
  \item \param~: 
    appel $c$ dans la fonction spécialisée $f_i$,
  \item \precond~: la fonction appelée nécessite plus d'entrées que n'en calcule
    $f_i$.
  \item \creable~: 
    non,
  \item \modifiable~: 
    oui, elle peut être remplacée par un $\actChangeCall(c, f_i, g_j)$ (voir les
    conditions de création de cette action).
  \item \generable~: 
    oui, lorsque la fonction $\Call(f_i)(c) = g_j$ a été modifiée, et qu'elle
    nécessite le marquage d'entrées qui ne le sont pas $\sigc(c,f_i)$.
  \item \application~: équivalente à \actModifCallInputs{} ou \actChooseCall{}
    en fonction du mode de fonctionnement.
  \item \genere~: dépend de l'application (voir ci-dessus).
  \item \modifie~: dépend de l'application (voir ci-dessus).
\end{itemize}


\proto{ModifCallInputs}{$c, f_i$}

\begin{itemize}
  \item \param~: appel $c$ dans la fonction spécialisée $f_i$.
  \item \precond~: $\Call(f_i)(c) = g_j$
  \item \creable~: non.
  \item \modifiable~: non.
  \item \generable~: oui, par l'application de \actMissingInputs{} dans certains
    modes.
  \item \application~: propage les marques des entrées de $g_j$ au niveau de
    l'appel $c$ de $f_i$ et et dans le dépendances.
  \item \genere~: 
    $\actChooseCall$ et/ou $\actMissingOutputs$ si le marquage de
    certains appels de fonction a été modifié, et $\actMissingInputs$ si
    $f_i$ est appelée et que les entrées de certains appels sont insuffisamment
    marquées.
  \item \modifie~: $f_i$.
\end{itemize}

\proto{MissingOutputs}{$c, f_i$}

\begin{itemize}
  \item \param~: appel $c$ dans la fonction spécialisée $f_i$
  \item \precond~: $\Call(f_i)(c) = g_j$
  \item \creable~: non.
  \item \modifiable~: oui, elle peut être remplacée par un 
    $\actChangeCall(c, f_i, g_k)$ 
    (voir les conditions de création de cette action).
  \item \generable~: oui, lorsque le marquage de  $c$ dans $f_i$ est modifié,
    que l'appel est attribué à $g_j$, et que les sorties de $g_j$
    sont insuffisantes.
  \item \application~: dépend du mode~:
    \begin{itemize}
      \item \actAddOutputMarks{}
      \item ou \actChooseCall{}
    \end{itemize}
  \item \genere~: dépend de l'application (voir ci-dessus).
  \item \modifie~: dépend de l'application (voir ci-dessus).
\end{itemize}

\proto{AddOutputMarks}{$f_i, \outsigf$}

\begin{itemize}
  \item \param~: $f_i$ est la fonction dont on veut modifier le marquage,
     $\outsigf$ indique les marques qui doivent être ajouté aux sorties.
  \item \precond~: $f_i$ doit exister et $\outsigf$ correspondre à une signature
    des sorties de $f$.
  \item \creable~: non.
  \item \modifiable~: non.
  \item \generable~: oui, par l'application de \actMissingOutputs.
  \item \application~: 
    les marques de $\outsigf$ sont ajoutées aux marques $m2$ des
    sorties de $f_i$ et propagées.
  \item \genere~: 
    \actChooseCall{} et/ou \actMissingOutputs{} si le marquage de
    certains appels de fonction a été modifié, et \actMissingInputs{} si
    $f_i$ est appelée et que les entrées de certains appels sont insuffisamment
    marquées.
  \item \modifie~: modification de $f_i$.
\end{itemize}

\proto{CopySlice}{$f_i$}

\begin{itemize}
  \item \param~: $f_i$ est la fonction spécialisée à copier.
  \item \precond~: $f_i$ doit exister.
  \item \creable~: oui.
  \item \modifiable~: oui (suppression),
  \item \generable~: non.
  \item \application~: 
    crée une nouvelle spécialisation $f_j$ de $f$ dont le marquage et les appels
    sont identiques à ceux de $f_i$.  Attention, à l'issue de cette action,
    $f_j$ n'est pas appelée.
  \item \genere~: non.
  \item \modifie~: création d'une nouvelle $f_j$.
\end{itemize}

\proto{Combine}{$f_i, f_j$}

\begin{itemize}
  \item \param~: $f_i$ et $f_j$ sont les fonctions spécialisées à combiner.
  \item \precond~: $f_i$ et $f_j$ doivent exister.
  \item \creable~: oui.
  \item \modifiable~: oui (suppression),
  \item \generable~: non.
  \item \application~: calcule une nouvelle spécialisation $f_k$
  \item \genere~: $\actChooseCall$
  \item \modifie~: création et calcul de $f_k$.
\end{itemize}

\proto{DeleteSlice}{$f_i$}

\begin{itemize}
  \item \param~: $f_i$ est la fonction spécialisée à supprimer.
  \item \precond~: $f_i$ ne doit pas être appelée.
  \item \creable~: oui.
  \item \modifiable~: oui (suppression).
  \item \generable~: non.
  \item \application~: $f_i$ est supprimée.
  \item \genere~: néant.
  \item \modifie~: $f_i$ est supprimée.
\end{itemize}
