\begin{frame}[fragile]{Uses of Objects}
\begin{block}{A small comparison}
%TODO: trouver comment avoir des hline avec du multirow...
\begin{tabular}{|c|c|c|}
& Traditional OO languages & \ocaml \\
Encapsulation & \multirow{2}{*}{Objects} & Modules \\
Late binding & & Objects \\
\end{tabular}
\end{block}
\begin{block}{Objects in \ocaml}
\begin{itemize}
\item used only where one an extensible behavior is explicitly desired.
\item modules and functors often more suitable.
\item Two usages in \framac: AST visitor and lablgtk-based GUI
\end{itemize}
\end{block}
\end{frame}

%TODO: faire des focus sur les différents éléments...
\begin{frame}[fragile]{How to define a class}
\lstset{language=Ocaml}
\tikzset{incode/.style={baseline=-.5ex,remember picture,overlay,
                        minimum height=0.05ex}}
\tikzset{outline/.style={shape=ellipse,draw=Gold1,fit={#1}}}
\tikzset{comment/.style={xshift=15mm,yshift=10mm,shape=ellipse callout,
                         anchor=south west,callout absolute pointer={#1},
                         at=(current page.south west),
                         draw=Goldenrod1,fill=LightGoldenrod1!60,
                         callout pointer arc=4}}
\begin{overlayarea}{\textwidth}{\textheight}
\begin{ocamlcode}
class my_visitor ?\tikz[incode]\coordinate(prmx);?x y?\tikz[incode]\coordinate(prmy);?:
?\tikz[incode]\coordinate(classtypb);?Visitor.frama_c_visitor?\tikz[incode]\coordinate(classtype);? =
?\tikz[incode]\coordinate(locvarb);?let local_var = f x y in?\tikz[incode]\coordinate(locvare);?
object(?\tikz[incode]\coordinate(selfb);?self?\tikz[incode]\coordinate(selfe);?)
  ?\tikz[incode]\coordinate(inhb);?inherit Visitor.frama_c_inplace?\tikz[incode]\coordinate(inhe);?
  ?\tikz[incode]\coordinate(valb);?val v1?\tikz[incode]\coordinate(vale);? = Stack.create ()
  val ?\tikz[incode]\coordinate(mutb);?mutable?\tikz[incode]\coordinate(mute);? v2 = 0
  ?\tikz[incode]\coordinate(methb);?method vvrbl?\tikz[incode]\coordinate(methe);? vi = ...
      ?\tikz[incode]\coordinate(callb);\tikz[incode]\coordinate(self2b);?self?\tikz[incode]\coordinate(self2e);?#internal_method?\tikz[incode]\coordinate(calle);? v2
  method ?\tikz[incode]\coordinate(privb);?private?\tikz[incode]\coordinate(prive);? internal_method x = ...
end
\end{ocamlcode}
\end{overlayarea}
\begin{overlayarea}{0pt}{0pt}
\onslide<2>{%
\begin{tikzpicture}[overlay, remember picture]
\node[outline=(prmx) (prmy)] (prm) {};
\node[comment={(prm.south west)}] {Classes can take parameters};
\end{tikzpicture}
}
\onslide<3>{%
\begin{tikzpicture}[overlay, remember picture]
\node[outline={(classtypb) (classtype)}] (prm) {};
\node[comment={(prm.south)}]
     {Constrain the interface (\lstinline{class type})};
\end{tikzpicture}
}
\onslide<4>{%
\begin{tikzpicture}[overlay, remember picture]
\node[outline={(inhb) (inhe)}] (prm) {};
\node[comment={(prm.south)}] {Inheritance};
\end{tikzpicture}
}
\onslide<5>{%
\begin{tikzpicture}[overlay, remember picture]
\node[outline={(selfb) (selfe)}] (prm) {};
\node[outline={(self2b) (self2e)}] {};
\node[comment={(prm.south west)}] {Naming current object};
\end{tikzpicture}
}
\onslide<6>{%
\begin{tikzpicture}[overlay, remember picture]
\node[outline={(callb) (calle)}] (prm) {};
\node[comment={(prm.south)}] {Calling a method};
\end{tikzpicture}
}
\onslide<7>{%
\begin{tikzpicture}[overlay, remember picture]
\node[outline={(methb) (methe)}] (prm) {};
\node[comment={(prm.south)}] {Normal (public) method};
\end{tikzpicture}
}
\onslide<8>{%
\begin{tikzpicture}[overlay, remember picture]
\node[outline={(privb) (prive)}] (prm) {};
\node[comment={(prm.south)}] {Private method};
\end{tikzpicture}
}
\onslide<9>{%
\begin{tikzpicture}[overlay, remember picture]
\node[outline={(valb) (vale)}] (prm) {};
\node[comment={(prm.south)}] {Instance variable};
\end{tikzpicture}
}
\onslide<10>{%
\begin{tikzpicture}[overlay, remember picture]
\node[outline={(mutb) (mute)}] (prm) {};
\node[comment={(prm.south)}] {Mutable instance variable};
\end{tikzpicture}
}
\onslide<11>{%
\begin{tikzpicture}[overlay, remember picture]
\node[outline={(locvarb) (locvare)}] (prm) {};
\node[comment={(prm.south)}] {Local variables};
\end{tikzpicture}
}
\end{overlayarea}
\end{frame}

\begin{frame}[fragile]{Typing}
\begin{itemize}
\item Each \texttt{class} implicitly defines a \texttt{class type}
\item Type is mainly the list of public methods with their type
\item Explicit definition: \lstinline{class type my_class_type = object ... end}
\item \alert{Structural} subtyping, not directly related to inheritance
  \begin{itemize}
  \item class $A$ is a subtype of $B$ if it has at least the same methods
  \item and the type of method $m$ in $A$ is a subtype of the one in $B$
  \item formalized duck-typing
  \end{itemize}
\end{itemize}
\end{frame}

\begin{frame}[fragile]{Definition of class members}
\begin{tabular}{m{0.17\textwidth}|%
                >{\centering}m{0.14\textwidth}|%
                >{\centering}m{0.14\textwidth}|%
                >{\centering}m{0.14\textwidth}|%
                >{\centering}m{0.17\textwidth}|}
\cline{2-5}
& Method & Private method & Instance variable & Local\newline binding
\tabularnewline
\cline{1-5}
\multicolumn{1}{|m{0.17\textwidth}|}
{Available outside of object} & 
   \yes & \no & \no & \no
\tabularnewline
\cline{1-5}
\multicolumn{1}{|m{0.17\textwidth}|}{Available in inherited classes} & 
   \yes & \yes & \yes & \no
\tabularnewline
\cline{1-5}
\multicolumn{1}{|m{0.17\textwidth}|}{Late\newline binding} & 
   \yes & \yes & \no & \no
\tabularnewline
\cline{1-5}
\end{tabular}
\end{frame}

\begin{frame}[fragile]{How to use objects}
\lstset{language=[Objective]Caml}
\begin{description}
\item[creation] \lstinline{let obj = new my_visitor x y}
\item[coercion] \lstinline{(obj :> Visitor.frama_c_visitor)}
\item[direct definition] \lstinline{let obj = object ... end}
\end{description}
\end{frame}

% Local Variables:
% TeX-master: "slides.tex"
% ispell-local-dictionary: "english"
% End:
