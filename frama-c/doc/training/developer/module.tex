\begin{frame}{Module}
  \framesubtitle{Overview}

  \begin{itemize}
  \item small typed functional language by itself
  \item based on the core language
  \item namespace
  \item encapsulation
  \item generic programming
  \end{itemize}

\end{frame}

\begin{frame}[fragile]{Module}
  \framesubtitle{Structure}

  \begin{ocamlcode}
(* implementation of rationals *)
struct
  type t = int * int
  let pgcd n m = ...
  let make n d = 
    let p = pgcd n d in
    n / p, d / p
  let integer n = n, 1
  let add (n1, d1) (n2, d2) = 
    make (n1 * d2 + n2 * d1) (d1 * d2)
  ...
end
  \end{ocamlcode}

\end{frame}

\begin{frame}[fragile]{Module}
  \framesubtitle{Names, submodule and access}

\begin{ocamlcode}
(* modules can be named *)
module Rational = 
  ... (* code of the previous slide *)

(* submodules are possible *)
module M1 = struct 
  module M2 = struct 
    module M3 = struct let x = ... end 
  end 
end

(* access through the dot notation *)
let r_one: Rational.t = Rational.integer 1
let x = M1.M2.M3.x
\end{ocamlcode}
\end{frame}

\begin{frame}[fragile]{Module}
  \framesubtitle{Typing}

  \begin{itemize}
  \item \ocaml infers a module type for each module
  \item types of structure are signatures
  \end{itemize}

  \begin{ocamlcode}
(* inferred type for module Rational *)
sig
  type t = int * int
  val pgcd: int -> int -> int
  val make: int -> int -> int * int
  val integer: int -> int * int
  val add: int * int -> int * int -> int * int
  ...
end
  \end{ocamlcode}
\end{frame}

\begin{frame}[fragile]{Module}
  \framesubtitle{Explicit Signature}

  \begin{ocamlcode}
module type Rational = sig
  type t
  val make: int -> int -> t
  val integer: int -> t
  val add: t -> t -> t
  ...
end
module Rational: Rational
  \end{ocamlcode}

  \begin{itemize}
  \item abstract types
  \item hide implementation details through subtyping
  \item encapsulation: easy to change the implementation without
    changing its interface
  \item unnamed signature
  \end{itemize}

\end{frame}

%% \begin{frame}[fragile]{Module}
%%   \framesubtitle{Compilation Unit}

%%   \begin{itemize}
%%   \item .ml files are particular structures
%%   \item .mli files are particular signatures
%%   \end{itemize}

%%   \begin{ocamlcode}
%% module L = List
%% module C = Cil
%% module type T = Cil_types
%%   \end{ocamlcode}
%% \end{frame}

\begin{frame}[fragile]{Module}
  \framesubtitle{Opening and inclusion}

  \begin{ocamlcode}
open Rational
let r_one = zero
open Cil_types

module My_list = struct
  include List
  let singleton x = [ x ]
  let tl _ = failwith "should never be used"
end
  \end{ocamlcode}

  \begin{itemize}
  \item `\lstinline+open+' provides a direct access to a structure's namespace
    (or signature)
  \item usually bad to have to many `\lstinline+open+' at the same time
  \item `\lstinline+include+' allows to extends/redefine a structure or a
    signature
  \end{itemize}

\end{frame}

\begin{frame}[fragile]{Module}
  \framesubtitle{Functor definition}
  
\begin{ocamlcode}
module type Ring = sig
  type t
  val zero: t         val one: t
  val add: t -> t -> t  val opp: t -> t
  val mult: t -> t -> t
end
module Polynomial(R: Ring) = struct
  type ring = R.t    type t = R.t array
  let zero = [| R.zero |]
  let monomial c n =
    let p = Array.create (n + 1) R.zero in
    p.(n) <- c; p
   ...
end
\end{ocamlcode}
\end{frame}

\begin{frame}[fragile]{Module}
  \framesubtitle{Functor use}

\begin{ocamlcode}
module IntegerPolynomial =
  Polynomial
    (struct
      type t = int
      let zero = 0
      let one = 1
      let add = ( + )
      let mult = ( * )
      let opp n = - n
    end)

module RationalPolynomial = 
  Polynomial(Rational)
\end{ocamlcode}
\end{frame}

\begin{frame}[fragile]{Module}
  \framesubtitle{Functor typing}

\begin{ocamlcode}
module Polynomial(R: Ring) = sig
  type ring = R.t    type t
  val zero: t      
  val monomial: R.t -> int -> t
  ...
end
  
module type Polynomial: sig
  type ring     type t
  val zero: t   val monomial: R.t -> int -> t
end
module Polynomial(R: Ring):
  Polynomial with type ring = R.t
\end{ocamlcode}

\end{frame}
