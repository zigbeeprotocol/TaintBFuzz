\documentclass{beamer}
\usepackage[utf8]{inputenc}
\usepackage{graphicx}
\usepackage{alltt}
\usepackage{amsmath,amssymb,theorem}

\title{Browsing your code dependencies}

\usetheme{frama-c}

\definecolor{vert}{rgb}{0.10588, 0.60392, 0.023529}	% #4e9a06
\newcommand{\orange}[1]{{\textcolor{frama-c-1}{#1}}}
\newcommand{\orangepale}[1]{{\textcolor{frama-c-2}{#1}}}
\newcommand{\vvert}[1]{{\textcolor{vert}{#1}}}
\newcommand{\bleu}[1]{{\textcolor{blue}{#1}}}
\newcommand{\gris}[1]{{\textcolor{gray}{#1}}}

\newcommand{\goup}{\vspace{-5pt}}

\newcommand{\launch}[2]{%
\only<beamer>{%
  \leavevmode%
  \pdfstartlink user {%
    /Subtype /Link
    /A <</S /Launch
    /F (#1)
       >>
       }%
  #2%
  \pdfendlink%
}
\only<handout>{#2}
}

\newcommand{\code}[1]{\bleu{\texttt{#1}}}
\newcommand{\comment}[1]{\gris{(#1)}}

\newcommand{\continuing}{\framesubtitle{(continuing)}}
\newcommand{\intro}[1]{\begin{center}\vvert{#1}\end{center}}
\newenvironment{sect}[1]{\orange{#1}\begin{itemize}}{\end{itemize}}
\newenvironment{features}{\begin{sect}{Features}}{\end{sect}}
\newenvironment{warnings}{\begin{sect}{Warnings}}{\end{sect}}
\newenvironment{whatitisgoodfor}{\begin{sect}{What is it good for}}{\end{sect}}
\newenvironment{howtouse}{\begin{sect}{How to use}}{\end{sect}}
\newenvironment{exs}{\begin{sect}{Examples}}{\end{sect}}
\newenvironment{precise}{\begin{sect}{More precisely}}{\end{sect}}

\begin{document}

%%%%%%%%%%%%%%%%%%%%%%%%%%%%%%%%%%%%%%%%%%%%%%%%%%%%%%%%%%%%%%%%%%%%%%%%%%%%%%%
% title

\begin{frame}
  \begin{center}
    \orange{\LARGE Frama-C Training Session\\[0.5em]
      \Large Browsing your code dependencies\\[1.2em] 
    \vvert{\large Julien Signoles}\\[0.5em]
    \vvert{\Large CEA LIST}\\[0.5em]
    \vvert{\large Software Reliability Labs}}
  \end{center}

  \vspace{1.2em}
  ~\hfill
  \includegraphics[width=0.15\textwidth]{cealist.pdf}
  \hfill~
\end{frame}

%%%%%%%%%%%%%%%%%%%%%%%%%%%%%%%%%%%%%%%%%%%%%%%%%%%%%%%%%%%%%%%%%%%%%%%%%%%%%%%
% general introduction

\begin{frame}{What are we going to do}

  \intro{How to better understand a C code within \orangepale{Frama-C} \\
    by extracting semantic information from this code}

\begin{sect}{For what purpose}
\item helping to start verification of an unknown code
\item helping to understand results of heavier analyses
\item helping heavier analyses to give better results
\item helping audit activities 
\item helping reverse-engineering activities 
\end{sect}\smallskip

\begin{sect}{In what way}
\item using a \orangepale{battery of Frama-C plug-ins}, either syntactic or
  semantic
\end{sect}\medskip

\end{frame}

%%%%%%%%%%%%%%%%%%%%%%%%%%%%%%%%%%%%%%%%%%%%%%%%%%%%%%%%%%%%%%%%%%%%%%%%%%%%%%%

\begin{frame}{Syntactic analyzers}

\goup
\intro{Only deduce information from a direct use of the AST}
\goup

\begin{warnings}
\item those here-presented use the normalised program, not the original one
\item does not use advanced semantical information (for instance, the value of
  a variable at some statement)
\item in particular, \orangepale{does not handle pointers}
\item some may provide incorrect results in some cases
\end{warnings}\smallskip

\begin{sect}{Syntactic analyzers within Frama-C}
\item analysing code using program syntax only is \orangepale{not the
  main goal of Frama-C}
\item only few syntactic analyzers in Frama-C
\end{sect}

\end{frame}

%%%%%%%%%%%%%%%%%%%%%%%%%%%%%%%%%%%%%%%%%%%%%%%%%%%%%%%%%%%%%%%%%%%%%%%%%%%%%%%

\begin{frame}{Syntactic analyzers}
\framesubtitle{what they (do not) provide}

\begin{sect}{What their are good for}
\item getting information quickly
\end{sect}\medskip

\begin{sect}{What their are \orangepale{\emph{not}} good for}
\item providing a big amount of useful information
\item providing confidence if they may provide incorrect results
\end{sect}

\end{frame}

%%%%%%%%%%%%%%%%%%%%%%%%%%%%%%%%%%%%%%%%%%%%%%%%%%%%%%%%%%%%%%%%%%%%%%%%%%%%%%%

\begin{frame}{Semantic analyzers}

  \intro{In this session, all semantic analyzers are based on abtract interpretation and the value
    analysis plug-in}

\begin{features}
\item \orangepale{theoretically sound}: always provide correct results, as long
  as there are no soudness implementation bugs
\item \orangepale{handle pointers correctly}
\end{features}\medskip

\begin{sect}{Semantic analyzers within Frama-C}
\item most Frama-C plug-in are semantic analyzers
\end{sect}

\end{frame}

%%%%%%%%%%%%%%%%%%%%%%%%%%%%%%%%%%%%%%%%%%%%%%%%%%%%%%%%%%%%%%%%%%%%%%%%%%%%%%%

\begin{frame}{Semantic analyzers}
\framesubtitle{warnings}

\begin{warnings}
\item \orangepale{run the value analysis first}
\item may take a long time
\item over-approximate the results
\item all the ways to improve the efficiency/precision of the
  value analysis apply
\item all the limitations of the value analysis also apply
\item all the alarms emitted by the value analysis should be carefully
  examined
\end{warnings}

\end{frame}

%%%%%%%%%%%%%%%%%%%%%%%%%%%%%%%%%%%%%%%%%%%%%%%%%%%%%%%%%%%%%%%%%%%%%%%%%%%%%%%
% part I: lightweight analyzers

\begin{frame}{Outline}
\begin{enumerate}
\item \vvert{Lightweight analyzers}
  \begin{itemize}
  \item Metrics           \hfill \emph{syntactic}
  \item Callgraphs        \hfill \emph{both}
  \item Constant foldings \hfill \emph{both}
  \item Occurrence        \hfill \emph{semantic}
  \end{itemize}\medskip
%
\item \gris{Dependencies and effects}
  \begin{itemize}
  \item \gris{Functional dependencies and effects  \hfill \emph{semantic}}
  \item \gris{Imperative effects       \hfill \emph{semantic}}
  \item \gris{Operational effect       \hfill \emph{semantic}}
  \item \gris{Data scoping             \hfill \emph{semantic}}
  \end{itemize}\medskip
%
\item \gris{Reducing code to analyse}
  \begin{itemize}
  \item \gris{Slicing   \hfill \emph{semantic}}
  \item \gris{Sparecode \hfill \emph{semantic}}
  \item \gris{Impact    \hfill \emph{semantic}}
  \end{itemize}
\end{enumerate}
\end{frame}

%%%%%%%%%%%%%%%%%%%%%%%%%%%%%%%%%%%%%%%%%%%%%%%%%%%%%%%%%%%%%%%%%%%%%%%%%%%%%%%

\begin{frame}{Lightweight analyzers}

  They are either:
  \begin{itemize}
  \item \vvert{syntactic analyzers}; or
  \item \vvert{semantic analyzers remaining quite precise} even if the value
    analysis does not give so precise results
  \end{itemize}

\end{frame}

%%%%%%%%%%%%%%%%%%%%%%%%%%%%%%%%%%%%%%%%%%%%%%%%%%%%%%%%%%%%%%%%%%%%%%%%%%%%%%%
% metrics

\begin{frame}{Metrics}

\framesubtitle{syntactic code metrics}

\goup
\intro{Give some syntactic metrics about the analyzed code.}
\goup

\begin{features}
\item defined and undefined functions
\item number of calls to each function
\item potential entry points (the never-called functions)
\item number of loc
\item number of conditionals, assignments, loops, calls, gotos, pointer access
\end{features}\smallskip

\begin{warnings}
\item measures are done on the normalised code, not on the original one
\item does not take function pointers into account
\end{warnings}

\end{frame}

%%%%%%%%%%%%%%%%%%%%%%%%%%%%%%%%%%%%%%%%%%%%%%%%%%%%%%%%%%%%%%%%%%%%%%%%%%%%%%%

\begin{frame}{Metrics}
\continuing

\begin{whatitisgoodfor}
\item helping to measure how difficult the analyses will be
\item helping to identify whether some file is missing
\item helping to identify which functions have to be stubbed or specified
\item helping to identify entry points of the analyzed code
\end{whatitisgoodfor}\medskip

\begin{howtouse}
\item \code{-metrics} dumps metrics on stdout
\item \code{-metrics-dump <f>} dumps metrics on file \code{f}
\item also (partially) available from the GUI
\end{howtouse}


\end{frame}

%%%%%%%%%%%%%%%%%%%%%%%%%%%%%%%%%%%%%%%%%%%%%%%%%%%%%%%%%%%%%%%%%%%%%%%%%%%%%%%
% callgraph

\begin{frame}{Syntactic callgraph}

\goup
\intro{Indicate the callers of each function}
\goup

\begin{features}
\item representation as \orangepale{graphs into dot files}
\item notion of \orangepale{service}, a group of related functions which seems
  to provide common functionalities
\end{features}\smallskip

\begin{sect}{Warning}
\item does not take function pointers into account
\end{sect}\smallskip

\begin{whatitisgoodfor}
\item helping to identify entry points of the analyzed code
\item helping to discover services provided by an application
\item grasping the code architecture
\end{whatitisgoodfor}

\end{frame}

%%%%%%%%%%%%%%%%%%%%%%%%%%%%%%%%%%%%%%%%%%%%%%%%%%%%%%%%%%%%%%%%%%%%%%%%%%%%%%%

\begin{frame}{Syntactic callgraph}
\continuing

\begin{howtouse}
\item \code{-cg <f>} dumps callgraph in dot file \code{f}
\item \code{-cg-init-func <f>} adds function \code{f} as a root service
\item from the GUI: menu \orangepale{View}, then \orangepale{Show Call Graph}
  \comment{still experimental}
\end{howtouse}

\end{frame}


%%%%%%%%%%%%%%%%%%%%%%%%%%%%%%%%%%%%%%%%%%%%%%%%%%%%%%%%%%%%%%%%%%%%%%%%%%%%%%%

\begin{frame}{Semantic callgraph}


  \intro{Same as the syntactic callgraph... \\
    But using the program semantics}

\begin{features}
\item correctly deal with function pointers
\end{features}\medskip

\begin{warnings}
\item run the value analysis first: may take a long time
\end{warnings}\medskip

\begin{whatitisgoodfor}
\item computing the callgraph for codes with function pointers
\end{whatitisgoodfor}
\end{frame}

%%%%%%%%%%%%%%%%%%%%%%%%%%%%%%%%%%%%%%%%%%%%%%%%%%%%%%%%%%%%%%%%%%%%%%%%%%%%%%%

\begin{frame}{Semantic callgraph}
\continuing

\begin{howtouse}
\item \code{-scg-dump} dumps the callgraph to stdout into dot format
\item \code{-cg-init-func <f>} uses function \code{f} as a root service
\item not available from the GUI
\end{howtouse}\medskip

\begin{warnings}
\item currently not the same interface as the syntactic callgraph
  \comment{will be fixed soon}
\item currently not exactly the same notion of service as the syntactic
  callgraph \comment{will be fixed soon}
\end{warnings}


\end{frame}

%%%%%%%%%%%%%%%%%%%%%%%%%%%%%%%%%%%%%%%%%%%%%%%%%%%%%%%%%%%%%%%%%%%%%%%%%%%%%%%
% users

\begin{frame}{Users}
\framesubtitle{function callees}

\goup
\intro{Same as the semantic callgraph... \\
  But not represented as a graph}
\goup

\begin{sect}{Feature}
\item display the callees of each functions
\end{sect}\smallskip

\begin{sect}{Warning}
\item no service computed
\end{sect}\smallskip

\begin{whatitisgoodfor}
\item extracting information with some external automatic tools (like grep)
\end{whatitisgoodfor}\smallskip

\begin{howtouse}
\item \code{-users} dumps the function callees on stdout
\end{howtouse}

\end{frame}

%%%%%%%%%%%%%%%%%%%%%%%%%%%%%%%%%%%%%%%%%%%%%%%%%%%%%%%%%%%%%%%%%%%%%%%%%%%%%%%
% constant folding

\begin{frame}{Syntactic constant folding}

\goup
\intro{Fold all constant expressions in the code before analysis}
\goup

\begin{sect}{Feature}
\item replace constant expressions by their results
\end{sect}\smallskip

\begin{sect}{Warning}
\item local propagation only: do not propagate the assignment of a constant to
  a left-value in the propram
\end{sect}\smallskip

\begin{whatitisgoodfor}
\item quickly simplifying programs with lots of constant expressions
\item using analysis puzzled by big constant expressions
\end{whatitisgoodfor}\smallskip

\begin{howtouse}
\item \code{-constfold} performs this analysis before all others
\end{howtouse}

\end{frame}

%%%%%%%%%%%%%%%%%%%%%%%%%%%%%%%%%%%%%%%%%%%%%%%%%%%%%%%%%%%%%%%%%%%%%%%%%%%%%%%

\begin{frame}{Semantic constant folding}

\intro{Propagate constant expressions in the whole program}

\begin{precise}
  \item generate a new program where expressions of the input program which are
    established as constant by the value analysis are
    \begin{itemize}
    \item replaced by their value
    \item propagated through the whole program
    \end{itemize}
\end{precise}\smallskip

\begin{features}
\item the output program is a compilable C code
\item it has the same behaviour as the original one
\item handle constant integers and pointers, even function pointers
\end{features}

\end{frame}

%%%%%%%%%%%%%%%%%%%%%%%%%%%%%%%%%%%%%%%%%%%%%%%%%%%%%%%%%%%%%%%%%%%%%%%%%%%%%%%

\begin{frame}{Semantic constant folding}
\continuing

\begin{sect}{Warning}
\item does not handle floating-point values yet
\end{sect}\smallskip

\begin{whatitisgoodfor}
\item simplifying programs with lots of constant values
\item using analysis puzzled by constant expressions
%\item watching benefits of constant propagation for some analysis
\end{whatitisgoodfor}\smallskip

\begin{howtouse}
\item \code{-semantic-const-folding} propagates constants and pretty print
  the new source code
\item \code{-semantic-const-fold <f1>, ..., <fn>} propagates constants only
  into functions \code{f1}, ..., \code{fn}
\item \code{-cast-from-constant} replaces expressions by constants even when
  doing so requires a pointer cast
\end{howtouse}

\end{frame}

%%%%%%%%%%%%%%%%%%%%%%%%%%%%%%%%%%%%%%%%%%%%%%%%%%%%%%%%%%%%%%%%%%%%%%%%%%%%%%%
% occurrence

\begin{frame}{Occurrence}
\framesubtitle{where variables are used}

\goup
\intro{Show the uses of a variable in a program}
\goup

\begin{precise}
\item highlight the left-values that may access a part of the location
  denoted by the selected variable
\end{precise}\smallskip

\begin{features}
\item take aliasing into account
\item also show uses of a C variable in logic annotations
\item mainly a graphical plug-in
\end{features}\smallskip

\begin{warnings}
\item quite difficult to use in batch mode
\item does not handle logic variable yet
\end{warnings}

\end{frame}

%%%%%%%%%%%%%%%%%%%%%%%%%%%%%%%%%%%%%%%%%%%%%%%%%%%%%%%%%%%%%%%%%%%%%%%%%%%%%%%

\begin{frame}{Occurrence}
\continuing

\begin{whatitisgoodfor}
\item understanding a quite mysterious piece of code
\item discovering some unknown aliases of the program
\end{whatitisgoodfor}\medskip

\begin{howtouse}
\item \code{-occurrence} dumps the occurrences of each variable on
  stdout
\item from the GUI: left panel and contextual menu
\end{howtouse}

\end{frame}

%%%%%%%%%%%%%%%%%%%%%%%%%%%%%%%%%%%%%%%%%%%%%%%%%%%%%%%%%%%%%%%%%%%%%%%%%%%%%%%
% part II

\begin{frame}{Outline}
\begin{enumerate}
\item \gris{Lightweight analyzers}
  \begin{itemize}
  \item \gris{Metrics           \hfill \emph{syntactic}}
  \item \gris{Callgraphs        \hfill \emph{both}}
  \item \gris{Constant foldings \hfill \emph{both}}
  \item \gris{Occurrence        \hfill \emph{semantic}}
  \end{itemize}\medskip
%
\item \vvert{Dependencies and effects}
  \begin{itemize}
  \item Functional dependencies and effects  \hfill \emph{semantic}
  \item Imperative effects       \hfill \emph{semantic}
  \item Operational effects      \hfill \emph{semantic}
  \item Data scoping             \hfill \emph{semantic}
  \end{itemize}\medskip
%
\item \gris{Reducing code to analyse}
  \begin{itemize}
  \item \gris{Slicing   \hfill \emph{semantic}}
  \item \gris{Sparecode \hfill \emph{semantic}}
  \item \gris{Impact    \hfill \emph{semantic}}
  \end{itemize}
\end{enumerate}
\end{frame}

%%%%%%%%%%%%%%%%%%%%%%%%%%%%%%%%%%%%%%%%%%%%%%%%%%%%%%%%%%%%%%%%%%%%%%%%%%%%%%%

\begin{frame}{Dependencies and effects}

\begin{features}
\item several notions of input/output for functions
\item several kinds of dependencies
\end{features}

\end{frame}

%%%%%%%%%%%%%%%%%%%%%%%%%%%%%%%%%%%%%%%%%%%%%%%%%%%%%%%%%%%%%%%%%%%%%%%%%%%%%%%
% functional dependencies

\begin{frame}{Functional dependencies}

\goup
\intro{Dependencies between inputs and outputs of functions}
\goup

\begin{sect}{Definitions}
\item \orangepale{functional output} of a function \code{f}: left-value that
  may be modified in \code{f} when \code{f} terminates
\item \orangepale{functional input} of a function \code{f}: left-value which
  may impact the output value of a functional output of \code{f}
\end{sect}\smallskip

\begin{features}
\item functional outputs and inputs 
\item dependencies between outputs and inputs
\item indicate whether the analyzer knows that an output is always modified
  (when the function terminates)
\item ignore local variables \comment{from the next release}
\end{features}

\end{frame}

%%%%%%%%%%%%%%%%%%%%%%%%%%%%%%%%%%%%%%%%%%%%%%%%%%%%%%%%%%%%%%%%%%%%%%%%%%%%%%%

\begin{frame}{Functional dependencies}
\continuing

\begin{howtouse}
\item mainly a batch plug-in
\item \code{-deps} displays the functional dependencies for each
  function
\item \code{-calldeps} displays the functional dependencies by callsite: if a
  function is called several times, results are not merged
\end{howtouse}\medskip

\begin{whatitisgoodfor}
\item providing dataflow specifications of functions
\item helping to understand relations between inputs and outputs of each
  function
\item improving precision of other analyser through \code{-calldeps}
\end{whatitisgoodfor}

\end{frame}

%%%%%%%%%%%%%%%%%%%%%%%%%%%%%%%%%%%%%%%%%%%%%%%%%%%%%%%%%%%%%%%%%%%%%%%%%%%%%%%
% operational dependencies

\begin{frame}{Inout}
\framesubtitle{imperative and operational effects}

\intro{What is read, what is written, \\
  what is read before being written}

\begin{sect}{Definitions}
\item \orangepale{imperative input} of a function \code{f}: left-value that may
  be read in \code{f}
\item \orangepale{imperative output} of a function \code{f}: left-value that
  may be written in \code{f}
\item \orangepale{operational input} of a function \code{f}: left-value that
  is read without having been previously written to, when \code{f}
  terminates
\end{sect}

\end{frame}

%%%%%%%%%%%%%%%%%%%%%%%%%%%%%%%%%%%%%%%%%%%%%%%%%%%%%%%%%%%%%%%%%%%%%%%%%%%%%%%

\begin{frame}{Inout}
\continuing

\begin{features}
\item imperative inputs and outputs
\item operational inputs
%\item ignore local variable
\end{features}

\begin{warnings}
\item mainly a batch plug-in
\item operational inputs are still experimental: the specification may
  change
\item operational outputs exist but are not yet documented
\end{warnings}\medskip

\end{frame}

%%%%%%%%%%%%%%%%%%%%%%%%%%%%%%%%%%%%%%%%%%%%%%%%%%%%%%%%%%%%%%%%%%%%%%%%%%%%%%%

\begin{frame}{Inout}
\framesubtitle{command line options}

\begin{howtouse}
\item \code{-input} displays the imperative inputs of each
  function; locals and function parameters are not displayed
\item \code{-input\_with\_formals} same as \code{-input}, but displaying
  function parameters
\item \code{-out} displays the imperative outputs of each function
\item \code{-inout} displays the operational inputs of each function
\end{howtouse}

\end{frame}

%%%%%%%%%%%%%%%%%%%%%%%%%%%%%%%%%%%%%%%%%%%%%%%%%%%%%%%%%%%%%%%%%%%%%%%%%%%%%%%
% sparecode

\begin{frame}{Scope}
\framesubtitle{data scoping}

\intro{Dependencies of a given left-value $l$ at a given program point $L$}

\begin{features}
\item \orangepale{show defs}: statements that may define the value of $l$ at
  $L$
\item \orangepale{zones}: statements that may contribute to define the value of
  $l$ at $L$
\item \orangepale{data scope}: statements where $l$ is guaranteed to have the
  same value as at $L$
\end{features}\medskip

\begin{sect}{Warning}
\item still experimental
\end{sect}

\end{frame}

%%%%%%%%%%%%%%%%%%%%%%%%%%%%%%%%%%%%%%%%%%%%%%%%%%%%%%%%%%%%%%%%%%%%%%%%%%%%%%%

\begin{frame}{Scope}
\continuing

\begin{whatitisgoodfor}
\item locally better understand what the program does
  \begin{itemize}
  \item relations between left-values
  \item where the current value of a left-value comes from
  \item scope of definition of a left-value
  \end{itemize}
\end{whatitisgoodfor}\medskip

\begin{howtouse}
\item only available from the GUI: sub-menu \orangepale{Dependencies} of the
  contextual menu with three entries (\orangepale{Show defs},
  \orangepale{Zones}, \orangepale{DataScope})
\end{howtouse}

\end{frame}

%%%%%%%%%%%%%%%%%%%%%%%%%%%%%%%%%%%%%%%%%%%%%%%%%%%%%%%%%%%%%%%%%%%%%%%%%%%%%%%
% part III

\begin{frame}{Outline}
\begin{enumerate}
\item \gris{Lightweight analyzers}
  \begin{itemize}
  \item \gris{Metrics           \hfill \emph{syntactic}}
  \item \gris{Callgraphs        \hfill \emph{both}}
  \item \gris{Constant foldings \hfill \emph{both}}
  \item \gris{Occurrence        \hfill \emph{semantic}}
  \end{itemize}\medskip
%
\item \gris{Dependencies and effects}
  \begin{itemize}
  \item \gris{Functional dependencies and effects \hfill \emph{semantic}}
  \item \gris{Imperative effects       \hfill \emph{semantic}}
  \item \gris{Operational effects      \hfill \emph{semantic}}
  \item \gris{Data scoping             \hfill \emph{semantic}}
  \end{itemize}\medskip
%
\item \vvert{Reducing code to analyse}
  \begin{itemize}
  \item Slicing   \hfill \emph{semantic}
  \item Sparecode \hfill \emph{semantic}
  \item Impact    \hfill \emph{semantic}
  \end{itemize}
\end{enumerate}
\end{frame}

%%%%%%%%%%%%%%%%%%%%%%%%%%%%%%%%%%%%%%%%%%%%%%%%%%%%%%%%%%%%%%%%%%%%%%%%%%%%%%%

\begin{frame}{Reducing code to analysed}

\begin{features}
\item generate a new program in a new project
\item the new program is compilable
\item the new program is usually shorter
\item the new program is usually easier to analyze
\end{features}\medskip

\begin{sect}{Warning}
\item usually is not always...
\end{sect}

\end{frame}

%%%%%%%%%%%%%%%%%%%%%%%%%%%%%%%%%%%%%%%%%%%%%%%%%%%%%%%%%%%%%%%%%%%%%%%%%%%%%%%
% slicing

\begin{frame}{Slicing}
  \framesubtitle{program specialization}

  \intro{Specialize the program according to some user-provided criteria}

\begin{features}
\item generate a new program in a new project
\item the new program is compilable
\item the new program and the analysed one have the same behaviour according to
  the slicing criterion
\end{features}

\end{frame}

%%%%%%%%%%%%%%%%%%%%%%%%%%%%%%%%%%%%%%%%%%%%%%%%%%%%%%%%%%%%%%%%%%%%%%%%%%%%%%%

\begin{frame}{Slicing}
\framesubtitle{available criteria}

\goup
\intro{What are the available criteria?}

\begin{sect}{Criteria for code observation}
% \item preserving behaviour of function \orangepale{calls}
% \item preserving behaviour of function \orangepale{returns}
\item preserving effects of \orangepale{statements}
\item preserving the \orangepale{read/write accesses} of/to left-values
\end{sect}\medskip

\begin{sect}{Criteria for proving properties}
\item preserving behaviour of \orangepale{assertions}
\item preserving behaviour of \orangepale{loop invariants}
\item preserving behaviour of \orangepale{loop variants}
\item preserving behaviour of \orangepale{threats} (emitted by the value
  analysis)
\end{sect}

\end{frame}

%%%%%%%%%%%%%%%%%%%%%%%%%%%%%%%%%%%%%%%%%%%%%%%%%%%%%%%%%%%%%%%%%%%%%%%%%%%%%%%

\begin{frame}{Slicing}
\framesubtitle{selecting criteria}

\begin{sect}{Pragmas}
\item \code{/*@ slice pragma ctrl; */} preserves the reachability of this
  control-flow point
\item \code{/*@ slice pragma expr e; */} preserves the value of the ACSL
  expression \code{e} at this control-flow point
\item \code{/*@ slice pragma stmt; */} preserves the effects of the next
  statement
\end{sect}\medskip

\begin{howtouse}
\item from command line options
\item from the GUI: left panel and contextual menu
\end{howtouse}

\end{frame}

%%%%%%%%%%%%%%%%%%%%%%%%%%%%%%%%%%%%%%%%%%%%%%%%%%%%%%%%%%%%%%%%%%%%%%%%%%%%%%%

\begin{frame}{Slicing}
\framesubtitle{command line options}

\goup
\intro{Each option preserves the semantics of the input program according to a
  specific criterion}

\begin{sect}{Options of the form \code{-slice-$criterion$ <f1>, ..., <fn>}}
\item \code{-slice-calls}: \orangepale{calls} to these functions
\item \code{-slice-return}: the \orangepale{return} of the these functions
\item \code{-slice-pragma}: \orangepale{slicing pragmas} in theses functions
\item \code{-slice-assert}: \orangepale{assertions} of these functions
\item \code{-slice-loop-inv}: \orangepale{loop invariants} in these functions
\item \code{-slice-loop-var}: \orangepale{loop variants} in these functions
\item \code{-slice-threat}: \orangepale{threats} in these functions
\end{sect}

\end{frame}

%%%%%%%%%%%%%%%%%%%%%%%%%%%%%%%%%%%%%%%%%%%%%%%%%%%%%%%%%%%%%%%%%%%%%%%%%%%%%%%

\begin{frame}{Slicing}
\framesubtitle{command line options (continuing)}

\begin{sect}{Options of the form \code{-slice-$criterion$ <v1>, ..., <vn>}}
\item \code{-slice-value} values of these left-values at the end of the entry
  point
\item \code{-slice-rd} \orangepale{read access} to these left-values
\item \code{-slice-wr} \orangepale{write access} to these left-values
\end{sect}\medskip

\begin{sect}{Warning}
\item addresses of the left-values are evaluated at the beginning of the entry
  point
\end{sect}

% \begin{sect}{Example}
% \item \code{frama-c -slice-return main -slice-wr foo, bar}
%   generates a new compilable program preserving
%   \begin{itemize}
%   \item the behaviour of the \texttt{return} of function \texttt{main}
%   \item the behaviour of each assignment to \texttt{foo} and \texttt{bar}
%   \end{itemize}
% \end{sect}

\end{frame}

%%%%%%%%%%%%%%%%%%%%%%%%%%%%%%%%%%%%%%%%%%%%%%%%%%%%%%%%%%%%%%%%%%%%%%%%%%%%%%%

\begin{frame}{Slicing}
\framesubtitle{customisation}

\begin{sect}{Custom options}
\item \code{-slicing-level <n>} specifies how to slice the callees
  \begin{itemize}
  \item \code{0}: never slice the called functions
  \item \code{1}: slice the callees but preserves all their functional outputs
  \item \code{2}: slice the callees but create at most 1 slice by function
  \item \code{3}: most precise slices; create as many slices as necessary
  \end{itemize}
  Default level is \code{2}
\item \code{-no-slice-undef-functions} does not slice the prototype of
  undefined functions (default)
\item \code{-slice-undef-functions} slices the prototype of undefined functions
\item \code{-slice-print} pretty prints the sliced code
\end{sect}

\begin{sect}{Warning}
\item the higher the slicing level is, the slower the slicing is
\end{sect}

\end{frame}

%%%%%%%%%%%%%%%%%%%%%%%%%%%%%%%%%%%%%%%%%%%%%%%%%%%%%%%%%%%%%%%%%%%%%%%%%%%%%%%

\begin{frame}{Slicing}

\begin{whatitisgoodfor}
\item helping to extract the signifiant parts of a program according to your
  own criteria
\item helping to understand where a behavior comes from
\item helping analyses to give better results
\item helping audit activities 
\end{whatitisgoodfor}

\end{frame}

%%%%%%%%%%%%%%%%%%%%%%%%%%%%%%%%%%%%%%%%%%%%%%%%%%%%%%%%%%%%%%%%%%%%%%%%%%%%%%%
% sparecode

\begin{frame}{Sparecode}
\framesubtitle{code cleaner}

\intro{Remove useless code of the program}

\begin{features}
\item generate a new program in a new project
\item the new program is compilable
\item the values assigned to the output variables of the main function are
  preserved in the new program
\item slicing pragmas may be used to keep some statements
  \begin{itemize}
  \item \code{/*@ slice pragma ctrl; */}
  \item \code{/*@ slice pragma expr e; */}
  \item \code{/*@ slice pragma stmt; */}
  \end{itemize}
\end{features}

\end{frame}

%%%%%%%%%%%%%%%%%%%%%%%%%%%%%%%%%%%%%%%%%%%%%%%%%%%%%%%%%%%%%%%%%%%%%%%%%%%%%%%

\begin{frame}{Sparecode}
\continuing

\begin{warnings}
\item still experimental
\item partial support of ACSL: only the annotations inside function bodies
  (\emph{e.g.} assertions) are processed at the moment; all the others are
  ignored and do not appear in the new program
\end{warnings}\medskip

\begin{whatitisgoodfor}
\item help to discover what is useless in a program
\item may improve the results of others analyzers which are puzzled by some
  useless code
\end{whatitisgoodfor}

\end{frame}

%%%%%%%%%%%%%%%%%%%%%%%%%%%%%%%%%%%%%%%%%%%%%%%%%%%%%%%%%%%%%%%%%%%%%%%%%%%%%%%

\begin{frame}{Sparecode}
\framesubtitle{command line options}

\begin{howtouse}
\item \code{-sparecode-analysis} removes statements and functions that are not
  useful to compute the result of the program
\item \code{-sparecode-rm-unused-globals} removes unused types and global variables
\item \code{-sparecode-no-annot} may remove some useless code even if it
  changes the validity of some ACSL properties
\end{howtouse}

\end{frame}

%%%%%%%%%%%%%%%%%%%%%%%%%%%%%%%%%%%%%%%%%%%%%%%%%%%%%%%%%%%%%%%%%%%%%%%%%%%%%%%
% impact

\begin{frame}{Impact}

\goup
\intro{What could be discovered \\
if the side effect of a statement would be revealed}
\goup

\begin{precise}
\item a statement \code{s} is impacted by a statement \code{s'} iff modifying
  the effect of \code{s'} by another \emph{possible} one may modify the effect
  of \code{s}
\item an effect is \emph{possible} iff there is an execution of the program
  that generates this effect. For instance, the possible effects of
  \code{z=x+y;} in \code{x=c?0:1; y=c?0:1; z=x+y} are \code{z} becomes equal to
  \code{0} or \code{2}.
\end{precise}\smallskip

\begin{sect}{Warning}
\item still experimental
\end{sect}

\end{frame}

%%%%%%%%%%%%%%%%%%%%%%%%%%%%%%%%%%%%%%%%%%%%%%%%%%%%%%%%%%%%%%%%%%%%%%%%%%%%%%%

\begin{frame}{Impact}
\continuing

\begin{whatitisgoodfor}
\item helping to understand what a statement is useful for
\item helping to apprehend code changes
\item helping audit activities, in particular security audits
\end{whatitisgoodfor}\medskip

\begin{howtouse}
\item \code{-impact-pragma <f1>, ..., <fn>} computes the impact from the pragmas
  in functions \code{f1}, ..., \code{fn}. Only the following pragma is yet
  usable.
  
  \code{/*@ impact pragma stmt; */}
\item \code{-impact-print} dumps the result of the analysis on stdout
%\item \code{-impact-slicing} slices
\item from the GUI: left panel and contextual menu
\end{howtouse}

\end{frame}

%%%%%%%%%%%%%%%%%%%%%%%%%%%%%%%%%%%%%%%%%%%%%%%%%%%%%%%%%%%%%%%%%%%%%%%%%%%%%%%
% Conclusion

\begin{frame}{Conclusion}

  \intro{Battery of Frama-C plug-ins presented}

\begin{sect}{For what purpose}
\item helping to start verification of an unknown code
\item helping to understand results of heavier analyses
\item helping heavier analyses to give better results
\item helping audit activities 
\item helping reverse-engineering activities 
\end{sect}

\intro{Browse your code dependencies more easily!}

\end{frame}

\end{document}
