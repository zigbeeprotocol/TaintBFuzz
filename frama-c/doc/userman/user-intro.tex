\chapter{Introduction}
\label{user-intro}

This is \FramaC's user manual. 
\FramaC is an open-source platform dedicated to
the analysis of source code written in the \tool{C} programming
language. The \FramaC platform gathers several analysis techniques into
a single collaborative framework. 

This manual gives an overview of \FramaC for newcomers, 
and serves as a reference for expert users.
It only describes those platform features that are common to all analyzers. 
Thus it \emph{does not cover} the use of the analyzers provided in the \FramaC
distribution (Eva, WP, E-ACSL, \ldots). Each of these analyses has its
own specific documentation~\cite{wp,value,eacsl}. Furthermore, research
papers~\cite{sefm12,fac15} give a synthetic view of the platform, its main and
composite analyses, and some of its industrial achievements, while the
development of new analyzers is described in the Plug-in Development
Guide~\cite{plugin-dev-guide}.

\section{About this document}

Appendix \ref{chap:changes} references all the changes made to this document
between successive \FramaC releases.

In the index, page numbers written in bold italics (e.g.  
\textcolor{red}{\textit{\textbf{1}}}) reference the defining sections for the
corresponding entries while other numbers (e.g. \textcolor{red}{1}) are
less important references. 

\begin{important}
The most important paragraphs are displayed inside gray boxes like this one.
A plug-in developer \textbf{must} follow them very carefully.
\end{important}

\section{Outline}

The remainder of this manual is organized in several chapters. 

\begin{description}
\item[Chapter~\ref{user-overview}] provides a general overview of the platform.
\item[Chapter~\ref{user-start}] describes the basic elements for starting the
  tool, in terms of installation and commands.
\item[Chapter~\ref{user-plugins}] explains the basics of plug-in categories,
  installation, and usage.
\item[Chapter~\ref{user-sources}] presents the options of the source code
  pre-processor.
\item[Chapter~\ref{user-analysis}] gives some general options for parameterizing
  analyzers.
\item[Chapter~\ref{user-properties}] touches on the topic of code properties,
  and their validation by the platform.
\item[Chapter~\ref{user-services}] introduces the general services offered by
  the platform.
\item[Chapter~\ref{user-gui}] gives a detailed description of the graphical
  user interface of \FramaC.
\item[Chapter~\ref{user-report}] describes the \texttt{Report} plug-in, used
  for textual consolidation and export of warnings, errors and properties.
\item[Chapter~\ref{user-variadic}] presents the \texttt{Variadic} plug-in,
  used to help other plug-ins handle code containing variadic functions, such
  as \texttt{printf} and \texttt{scanf}.
\item[Chapter~\ref{user-analysis-scripts}] details several scripts used to
  help setup and run analyses on large code bases.
\item[Chapter~\ref{user-compliance}] contains information related to compliance
  to standards and coding guidelines (ISO C, CERT, CWEs, etc).
\item[Chapter~\ref{user-errors}] explains how to report errors \via \FramaC's
  public Gitlab repository.
\end{description}
