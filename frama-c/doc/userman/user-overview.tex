\chapter{Overview}
\label{user-overview}

\section{What is \FramaC?}

\FramaC is a platform dedicated to the analysis of source code written in
\tool{C}. The \FramaC platform gathers several analysis techniques into
a single collaborative extensible framework. The collaborative approach of
\FramaC allows analyzers to build upon the results already computed by
other analyzers in the framework. Thanks to this approach, \FramaC can provide
a number of sophisticated tools such as a concurrency safety analysis
(Mthread~\cite{mthread}),
an enforcer of secure information flow (SecureFlow~\cite{secureflow1,secureflow2}),
or a set of tools for various test coverage criteria (LTest~\cite{ltest}), among many others.

\section{\FramaC as a Code Analysis Tool}

\emph{Static analysis} of source code is the science of computing synthetic
information about the source code without executing it.

To most programmers, static analysis means measuring the source code with
respect to various metrics (examples are the number of comments per line of
code and the depth of nested control structures). This kind of syntactic
analysis can be implemented in \FramaC but it is not the focus of the project.

Others may be familiar with heuristic bug-finding tools. These tools take more
of an in-depth look at the source code and try to pinpoint dangerous
constructions and likely bugs (locations in the code where an error might
happen at run-time). These heuristic tools do not find all such bugs, and
sometimes they alert the user for constructions which are in fact not bugs.

\FramaC is closer to these heuristic tools than it is to software metrics
tools, but it has two important differences with them: it aims at being
correct, that is, never to remain silent for a location in the source code
where an error can happen at run-time. And it allows its user to manipulate
\emph{functional specifications}, and to \emph{prove} that the source code
satisfies these specifications.

\FramaC is not the only correct code analyzer out there, but analyzers of the
\emph{correct} family are less widely known and used. Software metrics tools do
not guarantee anything about the behavior of the program, only about the way it
is written. Heuristic bug-finding tools can be very useful, but because they do
not find all bugs, they can not be used to prove the absence of bugs in a
program. \FramaC on the other hand can guarantee that there are no bugs in a
program ("no bugs" meaning either no possibility of a run-time error, or even
no deviation from the functional specification the program is supposed to
adhere to). This of course requires more work from the user than heuristic
bug-finding tools usually do, but some of the analyses provided by \FramaC
require comparatively little intervention from the user, and the collaborative
approach proposed in \FramaC allows the user to get results about complex
semantic properties.

\FramaC also provides some {\em dynamic analysis}, via plug-ins such as
\tool{E-ACSL}~\cite{eacsl}, which performs {\em runtime verification}.
It is often used as a complement to static analysis: when some properties
cannot be proven statically, \tool{E-ACSL} instruments the code so that,
during execution, such properties are verified and {\em enforced}:
violations lead to alerts or immediate termination, preventing silent program
corruption or malicious infiltration, and helping pinpoint the exact cause
of a problem, instead of an indirect consequence much later.

For both static and dynamic analyses, \FramaC focuses on the {\em source code}.
Plug-ins can help translate higher-level properties and specifications, or
even provide front-ends to other languages; but, in the end, the strength of
the platform is centered around the code and its properties.

\subsection{\FramaC as a Lightweight Semantic-Extractor Tool}

\FramaC analyzers, by offering the possibility to extract semantic information
from \C code, can help better understand a program source.

The \C language has been in use for a long time, and numerous programs today
make use of \C routines. This ubiquity is due to historical reasons, and to the
fact that \C is well adapted for a significant number of applications (\eg
embedded code). However, the \C language exposes many notoriously awkward
constructs.
Many \FramaC plug-ins are able to reveal what the analyzed C code actually
does. Equipped with \FramaC, you can for instance:
\begin{itemize}
\item observe sets of possible values for the variables of the program at each
  point of the execution;
\item slice the original program into simplified ones;
\item navigate the dataflow of the program, from definition to use or from use
  to definition.
\end{itemize}

\subsection{\FramaC for Formal Verification of Critical Software}

\FramaC can verify that an implementation complies with a related set of formal
specifications.

Specifications are written in a dedicated language, \acsl (\emph{ANSI/ISO \C
  Specification Language})~\cite{acsl}. The specifications can be partial,
concentrating on one aspect of the analyzed program at a time.

The most structured sections of your existing design documents can also be
considered as formal specifications. For instance, the list of global variables
that a function is supposed to read or write to is a formal
specification. \FramaC can compute this information automatically from the
source code of the function, allowing you to verify that the code satisfies
this part of the design document, faster and with less risks than by code
review.

%% \section{\FramaC as a Dynamic Analysis Tool}

%% \emph{Dynamic analysis} of source code is the science of computing synthetic
%% information about the source code by executing it. Such analysis include ---but
%% are not limited to--- testing and monitoring.

\section{\FramaC as a Tool for \C programs}

The \C source code analyzed by \FramaC is assumed to follow the \tool{C99} ISO
standard\index{C99 ISO standard@\tool{C99} ISO standard}\footnote{Some parts of
the \tool{C11} standard, as well as some common GCC extensions,
are also supported.}. \C comments may
contain \acsl annotations~\cite{acsl} used as specifications to be interpreted
by \FramaC. The subset of \acsl currently interpreted in \FramaC is described
in~\cite{acsl-implem}.

Each analyzer may define the subsets of \C and \acsl that it understands, as
well as introduce specific limitations and hypotheses. Please refer to each
plug-in's documentation.

\section{\FramaC as an Extensible Platform}

\FramaC is organized into a modular architecture (comparable to that of the
\tool{Eclipse} IDE): each analyzer comes in the form of a
\emph{plug-in} and is connected to the platform itself, or \emph{kernel}, which
provides common functionalities and collaborative data structures.

Several ready-to-use analyses are included in the \FramaC distribution. This
manual covers the set of features common to all plug-ins, plus some plug-ins
which are used by several others, such as the graphical user interface
(\GUI) and reporting tools (\Report).
It does not cover use of the plug-ins that come in the \FramaC
distribution: value analysis (\Value), functional verification (\WP),
runtime verification (\tool{E-ACSL}), \etc.
Each of these analyses has its own specific
documentation~\cite{value,wp,eacsl}.

Additional plug-ins can be provided by third-party developers and installed
separately from the kernel. \FramaC is thus not limited to the set of analyses
initially installed. For instance, it may be extended with the \tool{Frama-Clang}
plug-in~\cite{framaclang}, which provides an experimental front-end for C++ code;
or \tool{MetAcsl}~\cite{metacsl}, which allows specifying higher-level
meta-properties; among others.

\section{\FramaC as a Collaborative Platform}

\FramaC's analyzers collaborate with each other.
Each plug-in may interact with other plug-ins of his choosing.
The kernel centralizes information and conducts the
analysis. This makes for robustness in the development of \FramaC while
allowing a wide functionality spectrum.
For instance, the \Slicing plug-in uses the results
of the \Eva (value analysis) plug-in and of the \From (functional dependencies)
plug-in.

Analyzers may also exchange information
through \acsl annotations~\cite{acsl}. A plug-in that needs to
make an assumption about the behavior of the program may express
this assumption as an \acsl property. Because \acsl is the
{\em lingua franca} of all plug-ins, another plug-in can later
be used to establish the property.

With \FramaC, it is possible to take advantage of the
complementarity of existing analysis approaches.
It is possible to apply the most sophisticated techniques
only on those parts of the analyzed program that require them.
The low-level constructs can for instance effectively be hidden
from them by high-level specifications, verified by other,
adapted plug-ins.
Note that the sound collaboration of plug-ins on different parts of
a same program that require different modelizations of C is
a work in progress. At this time, a safe restriction
for using plug-in collaboration is to limit the analyzed
program and annotations to those \C and \acsl constructs
that are understood by all involved plug-ins.

\section{\FramaC as a Development Platform}

\FramaC may be used for developing new analyses.
The collaborative and extensible approach of \FramaC allows powerful
plug-ins to be written with relatively little effort.

There are a number of reasons for a user of \FramaC to be also interested in
writing their own plug-in:
\begin{itemize}
\item a custom plug-in can emit very specific queries for the existing
  plug-ins, and in this way obtain information which is not easily available
  through the normal user interface;
\item a custom plug-in has more flexibility for finely tuning the behavior of the
  existing analyses;
\item some analyses may offer specific opportunities for extension.
\end{itemize}

If you are a researcher in the field of program analysis,
using \FramaC as a testbed for your ideas is a choice to consider.
You may benefit from the ready-made parser for C programs with \acsl
annotations. The results of existing analyses may simplify the problems
that are orthogonal to those you want to consider (in particular, the
value analysis provides sets of possible targets of every pointer
in the analyzed C program).
And lastly, being available as a \FramaC plug-in increases
your work's visibility among existing industrial users of \FramaC.
The development of new plug-ins is described in the
Plug-in Development Guide~\cite{plugin-dev-guide}.

\section{\FramaC as an Educational Platform}

\FramaC is being used as part of courses on formal verification,
program specification, testing, static analysis, and abstract interpretation,
with audiences ranging from Master's students to active professionals, in
institutions world-wide.  \FramaC is part of the curriculum at several
universities in France, England, Germany, Portugal, Russia and in the US; at
schools such as Ecole Polytechnique, ENSIIE, ENSMA, or ENSI Bourges; and as
part of continuing education units at CNAM, or at Fraunhofer FOKUS.

If you are a teacher in the extended field of software safety, using \FramaC as
a support for your course and lab work is a choice to consider. You may benefit
from a clean, focused interface, a choice of techniques to illustrate, and a
in-tool pedagogical presentation of their abstract values at all program
points. A number of course materials are also available on the web, or upon
simple inquiry to the \FramaC team.
