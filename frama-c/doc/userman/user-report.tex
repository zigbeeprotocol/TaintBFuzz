%% --------------------------------------------------------------------------
%% --- Report Plugin
%% --------------------------------------------------------------------------

\chapter{Reports}
\label{user-report}

An execution of \FramaC outputs many messages, warnings and various property statuses in textual form.
The Graphical User Interface (see Chapter~\ref{user-gui}) is a very good place to visualize all these results,
but there are no synthetic results and integration with development environments can be difficult.

The \textttdef{report} plug-in, provided by default with the \FramaC platform, is designed for this purpose.
It provides the following features, which we detail in turn:
\begin{itemize}
\item Printing a summary of property consolidated statuses;
\item Exporting a CSV file of property consolidated statuses;
\item Filtering and classifying warnings, errors and property consolidated statuses, based on user-defined rules in JSON format;
\item Output the above classification in JSON format;
\item Make Frama-C exit with a non-null status code on some classified warning or error.
\end{itemize}

\FramaC has recently earned the ability to output data in the SARIF%
\footnote{Static Analysis Results Interchange Format,
\url{https://www.oasis-open.org/committees/tc_home.php?wg_abbrev=sarif}} format.
This is performed by the \textsf{Markdown Report} plug-in, described in
section~\ref{sarif}.


%% --------------------------------------------------------------------------
%% ---  Textual Report
%% --------------------------------------------------------------------------

\section{Reporting on Property Statuses}

The following options of the \texttt{report} plug-in are available for reporting on consolidated property statuses:

\begin{description}

\item[\tt -report] Displays a summary of property statuses in textual form.
  The output is structured by functions and behaviors. The details of which plug-ins participate
  into the consolidation of each property status is also provided.
  This report is designed to be human-readable, although it can be verbose, subject to changes, and
  not suitable for integration with other tools.

\item[\tt -report-print-properties] Also print the properties definition (in ACSL form) within the
  report.

\item[\tt -report-(no)-proven] If not set, filter out the proven properties from the report.
\item[\tt -report-(no)-specialized] If not set, filter out the properties that are auxiliary instances of other properties,
  like the preconditions of a function at its call sites.
\item[\tt -report-untried] If set, also include in the report the properties that have not been tried.

\end{description}

%% --------------------------------------------------------------------------
%% --- CSV Report
%% --------------------------------------------------------------------------

\section{Exporting to CSV}

The consolidated property status database can be exported to a \texttt{CSV} file, eg. for an easy import into Excel.
To use this feature, simply use the following option of the \texttt{report} plug-in:

\begin{description}
\item[\tt -report-csv <file>.csv] Output the report in CSV format into the specified file.
\end{description}

Notice than it is \emph{not} necessary to set \texttt{-report} option to use this feature.
It is highly recommended to use this option in combination with the following other
standard \FramaC options:

\begin{logs}
> frama-c ... -then -no-unicode -report-csv <file>.csv
\end{logs}

The format of the output CSV file is a collection of property consolidated statuses, with one property per line.
Each line consists of a collection of TAB (ascii \verb+0x0A+) separated columns.
The first line of the report contains the title of the columns, as follows:

\begin{logs}
> head -1 <file>.csv
directory	file	line	function	property kind	status	property
\end{logs}

\section{Classification}

We denote by \textit{event} any warning, error and (finally consolidated) property status emitted by any plug-in
during the execution of \FramaC. We introduce the notion of \textit{event classification} as follows:
\begin{itemize}
\item An event can be assigned a \textit{class}, identified by a name;
\item The event is associated to a location (\textit{file,line}) when available;
\item It can be reformulated with a \textit{title} and an extended \textit{description};
\item An event may trigger an \textit{action} when it is detected.
\end{itemize}

The classification of events is defined by the user through \emph{classification rules}
which are provided to the \texttt{report} plug-in via configuration files in JSON format.
A typical invocation of \FramaC with classification has the following structure:

\begin{logs}
> frama-c -report-rules <file>.json ...other plugins... -then -report-classify
\end{logs}

The collection of events starts once the classification rules have been loaded.
Finally, a classification report is build. There are various options to tune the classification process
and the reporting output. See section~\ref{report-classification-options} for details.

\textbf{Remark:} it is possible to emit different classification reports successively
from the command line. At each \verb+-report-classify+, the pool of collected events is flushed
and will not be included in subsequent reports.

\subsection{Action}

Classified events can trigger one of the following actions:
\begin{description}
\item[\tt SKIP] the event is filtered out and not included in the final report;
\item[\tt INFO] the event is kept for user information;
\item[\tt REVIEW] the event shall be carefully read by the user, it contains
  verifications to be performed by hand to guarantee the soundness of the provided results;
\item[\tt ERROR] all the results shall be considered wrong due to improper use of the tool.
\end{description}

By default, \FramaC warnings shall trigger a \texttt{REVIEW} action and errors an \texttt{ERROR}
one. However, it is possible to modify actions with classification rules.

\subsection{Rules}

Each classification rule is a JSON record following the structure of Figure~\ref{report-classification-rule}.
A file of classification rules shall contain one rule or an array of rules. Several files can be loaded.
The first rule that applies to an event takes priority over subsequent ones.

An individual rule consists of one mandatory \emph{pattern} field, and other optional fields. All the details
are provided in the figure.

\begin{figure}[htbp]
\begin{verbatim}
{
   // Optional Fields

   "classid": "<identifier>" ; // Default is "unclassified"
   "plugin": "<identifier>" ;  // Default is "kernel"
   "category": "<category>" ; // Default is "*" for all categories
   "title": "<free text>" ;    // Default is a short name of the event
   "descr": "<free text>" ;    // Default is the entire text of the event
   "action": "<SKIP|INFO|REVIEW|ERROR>" ; Default is 'REVIEW'

   // Mandatory Pattern Field (unless a category is specified)

   [ "error"        // Applies to error messages
   | "warning"      // Applies to warning messages
   | "unknown"      // Properties with « Unknown » status
   | "untried"      // Properties with « Not Tried » status
   | "invalid"      // Properties with « Invalid » status
   | "unproved"     // Properties with any status other than « Valid »
   ] : "<regexp>" ;
}
\end{verbatim}
\caption{Classification Rule JSON Format}
\label{report-classification-rule}
\end{figure}

Plug-ins are identified by their short name, typically \verb+"rte"+ for the \verb+Rtegen+ plug-in,
see \verb+frama-c -plugins+ for details.

Category filters apply to active warning categories\footnote{Warning categories
  are described in section~\ref{sec:feedback-options}.}
or warning categories promoted to errors.
Use option \verb+frama-c -[plugin]-warn-key help+ to display the list of available warning categories for a given plugin.
Category filters use the same syntax and meaning than warning control options: category \verb+a:b+ applies to all messages
with category \verb+a:b[:...]+, but \emph{not} to messages with category \verb+a+ or \verb+c[:...]+.

When using a category filter, the pattern field can be omitted to match all warning messages of this category.

\subsection{Regular Expressions}

\textsf{OCaml} regular expressions are accepted for \verb+<regexp>+ pattern fields.

Regular expressions are used to determine when a rule applies to an event. The rule matches a warning or error of the specified
plug-in if the regular expression matches a \emph{prefix} of the event message (excluding location and header).
For property rules, the regular expression must match a \emph{prefix} of the canonical property name,
which have the following structure:

\begin{verbatim}
<Function><Behavior><Category><Names>
\end{verbatim}

Each part of the canonical property name is optional and separated by a `\_` character.

\subsection{Reformulation}

The optional fields \verb+title+ and \verb+descr+ of a classification rule allow for a \emph{reformulation} of the event.
Reformulations are plain text enriched by references to sub-parts of the matching regular expression of the event.
Hence, \verb+\*+ stands for the entire event message, \verb+\0+ is the matched prefix, \verb+\1..\9+ refers to the corresponding
sub-group of the \textsf{OCaml} regular expression. You can use \verb+\n+ to insert a new-line character and \verb+\<c>+ to escape
character \verb+<c>+.

\subsection{JSON Output Format}

The classification results can be exported to a single file in JSON format. It consists of an array of classified events,
each one following the format given in Figure~\ref{report-classified-event}.

\begin{figure}[htbp]
\begin{verbatim}
{
  "classid": "<ident>" ;
  "action": "<INFO|REVIEW|ERROR>" ;
  "title": "<free text>" ;
  "descr": "<free text>" ;
  [ "file": "<path>" ; "line": "<number>" ; ]
}
\end{verbatim}
\caption{Classified Event JSON Format}
\label{report-classified-event}
\end{figure}

\subsection{Classification Options}
\label{report-classification-options}

\begin{description}
\item[\tt -report-classify] Report classification of all properties, errors and
  warnings (opposite option is -report-no-classify)
\item[\tt -report-exit] Exit on error (set by default, opposite option is
  -report-no-exit)
\item[\tt -report-output <*.json>]  Output -report-classify in JSON format
\item[\tt -report-output-errors <file>]  Output number of errors to \verb+<file>+
\item[\tt -report-output-reviews <file>]  Output number of reviews to \verb+<file>+
\item[\tt -report-output-unclassified <file>]  Output number of unclassified to \verb+<file>+
\item[\tt -report-absolute-path] Force absolute path in JSON output. Normal behavior is to
  output relative filepath for files that are relative to the current working directory.
\item[\tt -report-rules <*.json,...>]  Configure the rules to apply for
  classification, and start monitoring.
\item[\tt -report-status] Classify also property statuses (set by default, opposite
  option is -report-no-status)
\item[\tt -report-stderr] Output detailed textual classification on stderr
  (opposite option is -report-no-stderr)
\item[\tt -report-stdout] Force detailed textual classification on stdout
  (opposite option is -report-no-stdout)
\item[\tt -report-unclassified-error <action>]
  Action to be taken on unclassified
  errors (default is: \verb+'ERROR'+)
\item[\tt -report-unclassified-invalid <action>]
  Action to be taken on invalid
  properties (default is: \verb+'ERROR'+)
\item[\tt -report-unclassified-unknown <action>]
  Action to be taken on unknown
  properties (default is: \verb+'REVIEW'+)
\item[\tt -report-unclassified-untried <action>]
  Action to be taken on untried
  properties (default is: \verb+'SKIP'+)
\item[\tt -report-unclassified-warning <action>]
  Action to be taken on unclassified
  warnings (default is: \verb+'REVIEW'+)
\end{description}

\section{SARIF Output via the Markdown Report Plug-in}\label{sarif}

SARIF is a JSON-based standard output format for static analysis results.
It is currently supported by \FramaC in its version 2.1.0.
Some IDEs, such as Visual Code, contain plug-ins which allow importing SARIF
reports. A few Microsoft tools and libraries related to SARIF are available
at \url{https://sarifweb.azurewebsites.net/}. Microsoft also publishes
command-line tools for SARIF, made available via NPM packages and .Net Core
applications.

\subsection{Prerequisites}

SARIF output is produced by the \textsf{Markdown Report} (MdR) plug-in.
The plug-in is distributed with \FramaC, but it depends on optional
features, namely the \texttt{opam} package \texttt{ppx\_deriving\_yojson},
so it may not be available in every \FramaC installation.

When installing \FramaC via opam, include the optional dependency
\texttt{ppx\_deriving\_yojson} to ensure MdR will be available.
Note that MdR has other features and output formats, which are not described
here.

\subsection{Generating a SARIF Report}

To enable generation of a SARIF report, use option \texttt{-mdr-gen sarif}.
It will produce a JSON file (by default, \texttt{report.sarif}) with an entry
for each ACSL property.

You can change the output file name with option \texttt{-mdr-out <f>}.

Note that there are no filtering options in the SARIF output; it is up to the
tools importing the file to decide which information to sort, filter, and
display.

For more details about \textsf{Markdown Report}, use option \texttt{-mdr-help.}
