%; whizzy-master "developpeur.tex"

\chapter{Reference Manual}\label{chap:refman}

\begin{target}
Developers who would like to have a deep understanding of \framac.
\end{target}

This chapter is a reference manual for \framac developers. It provides details
completing the previous chapters.

%%%%%%%%%%%%%%%%%%%%%%%%%%%%%%%%%%%%%%%%%%%%%%%%%%%%%%%%%%%%%%%%%%%%%%%%%%%%%%%

\section{Configure.in} \label{refman:configure}\codeidxdef{configure.in}

Figure~\ref{fig:configure-sections} presents the different parts of
\texttt{configure.in} in the order in which they are introduced in the file. The
second column of the table indicates whether the given part might need
to be modified by a kernel-integrated plug-in
developer\index{Plug-in!Kernel-integrated}. More details are provided below.
\begin{figure}[htbp]
\begin{center}
\begin{tabular}{|c|c|c|c|}
\hline
\textbf{Id} & \textbf{Name} & \textbf{Mod.} & \textbf{Reference} \\
\hline \hline
1 & Configuration of \make & no & \\
2 & Configuration of \caml & no & \\
3 & Configuration of mandatory tools/libraries & no & \\
4 & Configuration of non-mandatory tools/libraries & no & \\
5 & Platform configuration & no & \\
6 & Wished \framac plug-in & \textbf{yes} &  Sections~\ref{conf:add}
and~\ref{conf:dep-lib}\\
7 & Configuration of plug-in tools/libraries & \textbf{yes} & Section~\ref{conf:lib} \\
8 & Checking plug-in dependencies & \textbf{yes} & Section~\ref{conf:dep-plug} \\
9 & Makefile creation & \textbf{yes} & Section~\ref{conf:add} \\
10 & Summary & \textbf{yes} & Section~\ref{conf:add} \\
\hline
\end{tabular}
\end{center}
\caption{Sections of \texttt{configure.in}.}\label{fig:configure-sections}
\end{figure}
\begin{enumerate}
\item \textbf{Configuration of \make} checks whether the version of \make is
  correct. Some useful option is \texttt{--enable-verbosemake} (resp.
  \texttt{--disable-verbosemake}) which set (resp. unset) the verbose mode for
  make. In verbose mode, full make commands are displayed on the user console:
  it is useful for debugging the makefile. In non-verbose mode, only command
  shortcuts are displayed for user readability.
\item \textbf{Configuration of \caml} checks whether the version of \caml is
  correct.
\item \textbf{Configuration of other mandatory tools/libraries} checks
  whether all the external mandatory tools and libraries required by the
  \framac kernel are
  present.\index{Library!Configuration}\index{Tool!Configuration}
\item \textbf{Configuration of other non-mandatory tools/libraries} checks
  which external non-mandatory tools and libraries used by the \framac kernel
  are present.\index{Library!Configuration}\index{Tool!Configuration}
\item \textbf{Platform Configuration} sets the necessary platform
  characteristics (operating system, specific features of \gcc, \emph{etc}) for
  compiling \framac.\index{Platform}
\item \textbf{Wished \framac Plug-ins} sets which \framac plug-ins the user wants
  to compile.\index{Plug-in!Wished}
\item \textbf{Configuration of plug-in tools/libraries} checks the
  availability of external tools and libraries required by plug-ins for
  compilation and
  execution.\index{Library!Configuration}\index{Tool!Configuration}
\item \textbf{Checking Plug-in Dependencies} sets which plug-ins have to be
  disabled (at least partially) because they depend on others plug-ins which are
  not available (at least partially).\index{Plug-in!Dependency}
\item \textbf{Makefile Creation} creates \texttt{Makefile.config} from
  \texttt{Makefile.config.in}\codeidx{Makefile.config} including information
  provided by this configuration.
\item \textbf{Summary} displays summary of each plug-in availability.
\end{enumerate}

%%%%%%%%%%%%%%%%%%%%%%%%%%%%%%%%%%%%%%%%%%%%%%%%%%%%%%%%%%%%%%%%%%%%%%%%%%%%%%%

\section{Makefiles}\label{refman:make}\codeidxdef{Makefile}

In this section, we detail the organization of the different Makefiles existing
in \framac. First Section~\ref{make:overview} presents a general overview. Next
Section~\ref{make:sections} details the different sections of
\texttt{Makefile.config.in}, \texttt{Makefile.common},
\texttt{Makefile.generic}, \texttt{Makefile.generating} and
\texttt{Makefile}. Next Section~\ref{make:plugin} introduces the variables
customizing \texttt{Makefile.dynamic}. Finally
Section~\ref{make:dynamic} shows specific details of \texttt{Makefile.dynamic}.

\subsection{Overview}\label{make:overview}

\framac uses different Makefiles (plus the plug-in specific ones). They are:
\begin{itemize}
\item \texttt{Makefile}\codeidxdef{Makefile}: the general Makefile of
  \framac;
\item \texttt{Makefile.generating}\codeidxdef{Makefile.generating}:
  it contains the complex rules that generate files. It is not
  directly in the general Makefile in order to reduce the dependencies
  of these rules to \texttt{Makefile.generating};
\item \texttt{Makefile.config.in}\codeidxdef{Makefile.config.in}: the Makefile
  configuring some general variables (especially the ones coming from
  \texttt{configure});
\item \texttt{Makefile.common}\codeidxdef{Makefile.common}: the Makefile
  providing some other general variables;
\item \texttt{Makefile.generic}\codeidxdef{Makefile.generic}: the Makefile
  providing generic rules for compiling source files
\item
  \texttt{.Makefile.plugin.generated}\codeidxdef{.Makefile.plugin.generated}:
  the Makefile introducing specific stuff for plug-in compilation which is
  generated from
  \texttt{Makefile.plugin.template}\codeidxdef{Makefile.plugin.template};
\item \texttt{Makefile.dynamic}\codeidxdef{Makefile.dynamic}: the Makefile
  usable by plug-in specific Makefiles.
\item \texttt{Makefile.dynamic\_config}\codeidxdef{Makefile.dynamic\_config}:
  this Makefile is automatically generated either from
  \texttt{Makefile.dynamic\_config.internal}%
  \codeidxdef{Makefile.dynamic\_config.internal} or
  \texttt{Makefile.dynamic\_config.external}%
  \codeidxdef{Makefile.dynamic\_config.external}. It sets variables which
  automatically configure \texttt{Makefile.dynamic}.
\item \texttt{Makefile.clean}\codeidxdef{Makefile.clean} contains specific
  targets for cleaning which are separated from \texttt{Makefile} for
  performance reasons.
\item \texttt{.Makefile.user}\codeidxdef{.Makefile.user} is a per-user
  Makefile that can be used to override some variables. If it is not present,
  the default values of \texttt{Makefile} variables will be used.
\end{itemize}
\texttt{Makefile}\codeidxdef{Makefile} and
\texttt{.Makefile.user}\codeidxdef{.Makefile.user} are part of the root
directory of the \framac distribution while
the other ones are part of directory \texttt{share}. Each Makefile either
includes or is included into at least another one. Figure~\ref{fig:makefiles}
shows these relationships.
\begin{figure}[htbp]
\codeidx{Makefile}
\codeidx{Makefile.generating}
\codeidx{Makefile.config.in}
\codeidx{Makefile.plugin.generated}
\codeidx{Makefile.dynamic}
\begin{tikzpicture}[
plugin/.style={fill=Orange,draw=Orange},
generated/.style={fill=palered,draw=palered},
kernel/.style={fill=darkgreen,draw=darkgreen},
other/.style={fill=palered,draw=darkgreen},
opt/.style={fill=palegreen,draw=palegreen},
include/.style={-Latex,very thick,fill=none},
generation/.style={-Latex,very thick,dashed,fill=none},
]
\begin{scope}[every node/.style={rounded corners,node font=\small}]
  {[start chain=going right, node distance=\bigpaddelta]
    \node[on chain,kernel,draw=black] (config)
    { Makefile.config.in };
    \node[on chain,kernel,draw=black] (dyn-internal)
    { Makefile.dynamic\_config.internal};
    \node[on chain,kernel,draw=black] (dyn-external)
    { Makefile.dynamic\_config.external};
  }
  \coordinate[yshift=-\largepadding]
  (common-basept) at ($(config.south west)!0.5!(dyn-internal.south east)$);
  {[start chain=kernel-chain going below, node distance=\bigpaddelta]
    \node[on chain,kernel,draw=black,anchor=north] at (common-basept)
    (common) {Makefile.common};
    \node[on chain,kernel,draw=black] (generic) {Makefile.generic};
  }

  \node[generated,draw=black,right=\largepadding]
  at ($(kernel-chain-begin.north east)!0.5!(kernel-chain-end.south east)$)
  (dyn) { Makefile.dynamic\_config };

  {[start chain=main going right,node distance=\padding]
    \node[on chain,kernel,draw=black,
    below left=\largepadding and \padding]
    at (kernel-chain-end.south west) (generating) { Makefile.generating };
    \node[on chain,kernel,draw=black] (makefile) { Makefile };
    \node[on chain] {\ldots};
    \node[on chain,generated,draw=black]
    (generated) {.Makefile.plugin.generated };
    \node[on chain,kernel,draw=black,node distance=\bigpadding]
    (dynamic) {Makefile.dynamic};
  }
  {[start chain=opt going left,node distance=\bigpaddelta]
    \node[on chain,kernel,draw=black,anchor=north east,yshift=-\bigpadding]
    at (generated.south east) (template) {Makefile.plugin.template};
    \node[on chain,opt,draw=black] (clean) {Makefile.clean};
    \node[on chain,opt,draw=black] (user) {.Makefile.user};
  }
  {[start chain=specific going right,node distance=\bigpaddelta]
    \node[on chain,plugin,draw=black,anchor=north east,xshift=-\largepadding,yshift=-\largepadding]
    at (opt-begin.south east) {Makefile for plug-in 1 };
    \node[on chain] {\ldots};
    \node[on chain,plugin,draw=black] {Makefile for plugin $n$};
  }
\end{scope}
\matrix[anchor=north east,yshift=-\bigpadding,matrix of nodes,
every node/.style={node font=\small,right}] at (specific-end.south east) {
\textbf{Caption:} \\
\node at(0,0) (m1) {$m_1$};
\node at (\largepadding,0) (m2) {$m_2$};
\draw[include] (m1) -- (m2);
&
Makefile $m_1$ is included in Makefile $m_2$
\\
\node at(0,0) (gm1) {$m_1$};
\node at (\largepadding,0) (gm2) {$m_2$};
\draw[generation] (gm1) -- (gm2);
&
 Makefile $m_2$ is generated from Makefile $m_1$ \\
|[plugin,draw=none]| orange boxes & Plug-in Makefiles \\
|[generated,draw=none]| red boxes & Generated Makefiles \\
|[kernel,draw=none]| green boxes & Other kernel Makefiles \\
|[opt,draw=none]| light green boxes & Optional kernel Makefiles \\
};
\draw[kernel,include] (config) -- (common);
\draw[kernel,include] (dyn-internal) to[out=-60,in=20] (node cs:name=makefile,angle=60);
\draw[generated,generation] (dyn-internal) -- (dyn);
\draw [generated,generation] (dyn-external) -- (dyn);
\draw[kernel,include] (common.west) to[out=200,in=135] (node cs:name=makefile,angle=170);
\draw[kernel,include] (generic.west) to[out=200,in=120]
 (node cs:name=makefile,angle=130);
\draw[kernel,include] (common) -- (node cs:name=dynamic,angle=50);
\draw[kernel,include] (generic) -- (node cs:name=dynamic,angle=120);
\draw[kernel,include] (dyn) -- (node cs:name=dynamic,angle=20);
\draw[kernel,include] (generating) -- (makefile);
\draw[plugin,include,bend left] (generated.south west) to (makefile.south east);
\draw[plugin,include,bend right] (generated.north west) to (makefile.north east);
\draw[generated,generation] (template) -- (generated);
\draw[kernel,include] (generated) -- (dynamic);
\draw[plugin,include,bend left] (dynamic) to (node cs:name=specific-begin,angle=10);
\draw[plugin,include] (dynamic) -- (specific-end);
\draw[opt,include] (user) -- (makefile);
\draw[opt,include] (clean) -- (makefile);
\end{tikzpicture}
\caption{Relationship between the Makefiles}\label{fig:makefiles}
\end{figure}
\texttt{Makefile} and \texttt{Makefile.dynamic} are independent: the first one
is used to compile the \framac kernel while the second one is used to compile
the \framac plug-ins. Their common variables are defined in
\texttt{Makefile.common} (which includes
\texttt{Makefile.config.in}). They also include \texttt{Makefile.generic},
that defines default compilation rules for various kinds of source files.
\texttt{Makefile.plugin.template} defines generic rules
and variables for compiling plug-ins. It is used to generate
\texttt{.Makefile.plugin.generated} for each plugin.
\texttt{.Makefile.plugin.generated} is included either
by \texttt{Makefile} for
kernel-specific plug-ins integrated in \framac Makefile and by
\texttt{Makefile.dynamic} for plug-ins with their own Makefiles.
\texttt{.Makefile.user}\codeidxdef{.Makefile.user} is included by
\texttt{Makefile} when the former exists. It is only used when compiling
\framac itself, and has no effect for external plugins.

\subsection{Sections of \texttt{Makefile},
  \texttt{Makefile.generating},
  \texttt{Makefile.config.in}, 
  \texttt{Makefile.common} 
  and \texttt{Makefile.generic}}
\label{make:sections}
\codeidxdef{Makefile}
\codeidxdef{Makefile.generating}
\codeidxdef{Makefile.config.in}
\codeidxdef{Makefile.common}
\codeidxdef{Makefile.generic}

Figure~\ref{fig:make-sections} presents the different parts of
\texttt{Makefile.config.in}, \texttt{Makefile.common},
\texttt{Makefile.generic}, \texttt{Makefile.generating} and \texttt{Makefile} in
the order that they are introduced in these files. The third row of the tabular
says whether the given part may be modified by a kernel-integrated plug-in
developer\index{Plug-in!Kernel-integrated}. More details are provided below.
\begin{figure}[htbp]
\begin{center}
\begin{tabular}{|c|c|c|c|c|}
\hline
\textbf{Id} & \textbf{Name} & \textbf{File} & \textbf{Mod.} &
\textbf{Reference} \\
\hline \hline
1 & Working directories & Makefile.config.in & no & \\
2 & Installation paths & Makefile.config.in & no & \\
3 & Ocaml stuff & Makefile.config.in & no & \\
4 & Libraries & Makefile.config.in & no & \\
5 & Miscellaneous commands & Makefile.config.in & no & \\
6 & Miscellaneous variables & Makefile.config.in & no & \\
7 & Variables for plug-ins & Makefile.config.in & no & \\
\hline
1 (bis) & Working directories & Makefile.common & no & \\
8 & Flags & Makefile.common & no & \\
9 & Verbosing & Makefile.common & no & \\
10 & Shell commands & Makefile.common & no & \\
11 & Command pretty printing & Makefile.common & no & \\
12 & Tests & Makefile.common & no & \\
\hline
13 & Generic rules & Makefile.generic & no & \\
\hline
14 & Source files generation & Makefile.generating & no & \\
\hline
15 & Global plug-in variables & Makefile & no & \\
16 & Additional global variables & Makefile & no & \\
17 & Main targets & Makefile & no & \\
18 & Coverage & Makefile & no & \\
19 & Ocamlgraph & Makefile & no & \\
20 & \framac Kernel & Makefile & no & \\
21 & Plug-in sections & Makefile & \textbf{yes} & Section~\ref{adv:make} \\
22 & Generic variables & Makefile & no & \\
23 & Toplevel & Makefile & no & \\
24 & GUI & Makefile & no & \\
25 & Standalone obfuscator & Makefile & no & \\
26 & Tests & Makefile & no & \\
27 & Emacs tags & Makefile & no & \\
28 & Documentation & Makefile & no & \\
29 & Installation & Makefile & \textbf{yes} & \todoshort \\
30 & File headers: license policy & Makefile & \textbf{yes}
 & % Section~\ref{adv:copyright}
\\
31 & Makefile rebuilding & Makefile & no & \\
32 & Cleaning & Makefile & no & \\
33 & Depend & Makefile & no & \\
34 & \ptests & Makefile & no & \\
35 & Source distribution & Makefile & no & \\
\hline
\end{tabular}
\end{center}
\caption{Sections of \texttt{Makefile.config.in}, \texttt{Makefile.common} and
 \texttt{Makefile}.}\label{fig:make-sections}
\end{figure}
\begin{enumerate}
\item \textbf{Working directories} (split between
  \texttt{Makefile.config.in} and \texttt{Makefile.common} defines the main
  directories of \framac. In particular, it declares the variable
  \texttt{FRAMAC\_SRC\_DIRS}\codeidxdef{FRAMAC\_SRC\_DIRS} which should be extended
  by a plug-in developer if he uses files which do not belong to the plug-in
  directory (that is if variable
  \texttt{PLUGIN\_TYPES\_CMO}\codeidx{PLUGIN\_TYPES\_CMO} is set, see
  Section~\ref{make:plugin}).
\item \textbf{Installation paths} defines where \framac has to be installed.
\item \textbf{Ocaml stuff} defines the \ocaml compilers and specific related
  flags.
\item \textbf{Libraries} defines variables for libraries required by \framac.
\item \textbf{Miscellaneous commands} defines some additional commands.
\item \textbf{Miscellaneous variables} defines some additional variables.
\item \textbf{Variables for plug-ins} defines some variables used by plug-ins
  distributed within \framac (and using the \texttt{configure} of \framac).
\item \textbf{Flags} defines some variables setting compilation flags.
\item \textbf{Verbosing} sets how \texttt{make} prints the command. In
  particular, it defines the variable \texttt{VERBOSEMAKE}\codeidx{VERBOSEMAKE}
  which must be set \texttt{yes} in order to see the full make commands in the
  user console. The typical use is
\begin{shell}
$ make VERBOSEMAKE=yes
\end{shell}
\item \textbf{Shell commands} sets all the shell commands eventually executed
  while calling \texttt{make}.
\item \textbf{Command pretty printing} sets all the commands to be used for
  pretty printing.
\begin{example}
Consider the following target \texttt{foo} in a plug-in specific Makefile.
\codeidx{PRINT\_CP}\codeidx{CP}
\begin{makefilecode}
foo: bar
	$(PRINT_CP) $@
        $(CP) $< $@
\end{makefilecode}
Executing
\begin{shell}
$ make foo
\end{shell}
prints
\begin{shell}
Copying to  foo
\end{shell}
while executing\codeidx{VERBOSEMAKE}
\begin{shell}
$ make foo VERBOSEMAKE=yes
\end{shell}
prints
\begin{shell}
cp -f bar foo
\end{shell}
If one of the two commands is missing for the target \texttt{foo}, either
\texttt{make foo} or \texttt{make foo VERBOSEMAKE=yes} will not work as
expected.
\end{example}
\item \textbf{Tests} defines a generic template for testing plug-ins.
\item \textbf{Generic rules} contains rules in order to automatically produces
  different kinds of files (\emph{e.g.} \texttt{.cm[iox]} from \texttt{.ml} or
  \texttt{.mli} for \ocaml files)
\item \textbf{Source files generation} contains rules for generating files
  that depend on the configuration and on which the main \texttt{Makefile}
  depends on. They are put in an auxiliary \texttt{Makefile.generating} to
  avoid unnecessary rebuilds.
\item \textbf{Global plug-in variables} declares some plug-in specific variables
  used throughout the makefile.
\item \textbf{Additional global variables} declares some other variables used
  throughout the makefile.
\item \textbf{Main targets} provides the main rules of the makefile. The most
  important ones are \texttt{top}, \texttt{byte} and \texttt{opt} which
  respectively build the \framac interactive, bytecode and native toplevels.
\item \textbf{Coverage} defines how compile the eponymous library.
\item \textbf{Ocamlgraph} defines how compile the eponymous library.
\item \textbf{\framac Kernel} provides variables and rules for the \framac
  kernel\index{Kernel}. Each part is described in specific sub-sections.
\item ~\\[-6mm]
  \begin{important}
    After Section ``Kernel'', there are several sections corresponding to
    \textbf{plug-ins}\index{Plug-in!Compilation} (see
    Section~\ref{make:plugin}). This is the part that a plug-in developer has to
    modify in order to add compilation directives for its plug-in.
  \end{important}
\item \textbf{Generic variables} provides variables containing files to be
  linked in different contexts.
\item \textbf{Toplevel} provides rules for building the files of the form
  \texttt{bin/toplevel.*}.
\item \textbf{GUI} provides rules for building the files of the form
  \texttt{bin/viewer.*}
\item \textbf{Standalone obfuscator} provides rules for building the \framac
  obfuscator.
\item\label{make:sec:tests} 
  \textbf{Tests} provides rules to execute tests\index{Test}.
  \verb|make tests| takes care of generating the appropriate environment and 
  launching \texttt{ptests} (see Section~\ref{adv:ptests}) for all
  test suites of the kernel and enabled plugins. It is possible
  to pass options to \texttt{ptests} through the 
  \texttt{PTESTS\_OPTS}\codeidxdef{PTESTS\_OPTS} environment variable.
\item \textbf{Emacs tags} provides rules which generate \emacs tags (useful for
  a quick search of \caml definitions).\index{Tags}
\item \textbf{Documentation} provides rules generating \framac source
  documentation (see Section~\ref{adv:documentation}).\index{Documentation}
\item \textbf{Installation} provides rules for installing different parts of
  \framac.
\item \textbf{File headers: license policy} provides variables and rules to
  manage the \framac license policy.
  \index{Header}\index{License}
\item \textbf{Makefile rebuilding} provides rules in order to automatically
  rebuild \texttt{Makefile} and \texttt{configure} when required.
\item \textbf{Cleaning} provides rules in order to remove files generated by
  makefile rules.
\item \textbf{Depend} provides rules which compute \framac source dependencies.
\item \textbf{Ptests} provides rules in order to build \ptests (see
  Section~\ref{adv:ptests}).\index{Ptests}
\item \textbf{Source distribution} provides rules usable for distributing
  \framac.
\end{enumerate}

\subsection{Variables of \texttt{Makefile.dynamic}}
\label{make:plugin}
\codeidxdef{Makefile.dynamic}

Figures~\ref{fig:make-parameters} and~\ref{fig:special-parameters}
presents all the variables that can be set before including
\texttt{Makefile.dynamic} (see Section~\ref{adv:dynamic-make}).
Details are provided below.
\begin{figure}[htbp]
\begin{center}
\begin{tabularx}{\textwidth}{|c|c|X|}
\hline
\textbf{Kind} & \textbf{Name} & \textbf{Specification} \\
\hline \multirow{5}{27mm}{\centering{Plug-in Declaration}}
& \texttt{PLUGIN\_NAME} & Module name of the plug-in \\
& \texttt{PLUGIN\_DIR} & Directory containing plug-in source files \\
& \texttt{PLUGIN\_ENABLE} & Whether to compile the plug-in \\
& \texttt{PLUGIN\_HAS\_META} & Provided \texttt{META} (default: \texttt{no})\\
\hline \multirow{4}{27mm}{\centering{Packaging\\(for \texttt{META})}}
& \texttt{PLUGIN\_DESCRIPTION} & Short description (defaults to name) \\
& \texttt{PLUGIN\_VERSION} & Version number (defaults to \framac's) \\
& \texttt{PLUGIN\_DEPENDENCIES} & Dependencies to other plug-ins\\
& \texttt{PLUGIN\_REQUIRES} & Dependencies to \texttt{ocamlfind} packages\\
\hline \multirow{4}{27mm}{\centering{Object Files}}
& \texttt{PLUGIN\_CMO} & Object files (without \texttt{.cmo}) \\
& \texttt{PLUGIN\_CMI} & Standalone interfaces (without \texttt{.cmi}) \\
& \texttt{PLUGIN\_GUI\_CMO} & Additional objects files for the GUI \\
& \texttt{PLUGIN\_TYPES\_CMO} & External \texttt{.cmo} files \\
\hline \multirow{6}{27mm}{\centering{Compilation flags}}
& \texttt{PLUGIN\_BFLAGS} & Plug-in specific flags for \texttt{ocamlc} \\
& \texttt{PLUGIN\_OFLAGS} & Plug-in specific flags for \texttt{ocamlopt} \\
& \texttt{PLUGIN\_EXTRA\_BYTE} & Additional bytecode files to link against \\
& \texttt{PLUGIN\_EXTRA\_OPT} & Additional native files to link against \\
& \texttt{PLUGIN\_EXTRA\_DIRS} & Additional directories containing source files, relative to the root directory of the plugin (i.e. \texttt{PLUGIN\_DIR}) \\
& \texttt{PLUGIN\_LINK\_BFLAGS} & Plug-in specific flags for linking with
\texttt{ocamlc} \\
& \texttt{PLUGIN\_LINK\_OFLAGS} & Plug-in specific flags for linking with
\texttt{ocamlopt} \\
& \texttt{PLUGIN\_LINK\_GUI\_BFLAGS} & Plug-in specific flags for linking a GUI
with \texttt{ocamlc} \\
& \texttt{PLUGIN\_LINK\_GUI\_OFLAGS} & Plug-in specific flags for linking a GUI
with \texttt{ocamlopt} \\
\hline
\end{tabularx}
\end{center}
\caption{Standard parameters of \texttt{Makefile.dynamic}.}\label{fig:make-parameters}
\end{figure}
\begin{itemize}
\item Variable \texttt{PLUGIN\_NAME}\codeidxdef{PLUGIN\_NAME}
  is the module name of the plug-in.\index{Plug-in!Name}

  \begin{important}
    This name must be capitalized (as is each \caml module name). It
    must be distinct from all other visible modules in the plugin
    directory, or in the \framac kernel.
  \end{important}
\item Variable \texttt{PLUGIN\_DIR}\codeidxdef{PLUGIN\_DIR}
  is the directory containing plug-in source files\index{Plug-in!Directory}. It
  is usually set to \texttt{src/plugins/$plugin$} where $plugin$ is the
  plug-in name.
\item Variable
  \texttt{PLUGIN\_ENABLE}\codeidxdef{PLUGIN\_ENABLE} must be
  set to \yes if the plug-in has to be compiled. It is usually set to
  \texttt{@$plugin$\_ENABLE@} provided by \texttt{configure.in} where $plugin$
  is the plug-in name.
\item Variable
  \texttt{PLUGIN\_HAS\_META}\codeidxdef{PLUGIN\_HAS\_META} must
  be set to \yes if plug-in $plugin$ gets a \texttt{META} file
  describing its packaging. Unless \texttt{PLUGIN\_HAS\_META} is \texttt{yes},
  the following variables are used to generate a suitable \texttt{META}:
  \begin{itemize}
  \item \texttt{PLUGIN\_VERSION}\codeidxdef{PLUGIN\_VERSION}
     short text with $plugin$ version number.
     Default value is \framac's version number.
  \item \texttt{PLUGIN\_REQUIRES}\codeidxdef{PLUGIN\_REQUIRES}
     packages that must be loaded before the $plugin$.
     By default, there is no such package.
  \item \texttt{PLUGIN\_DEPENDENCIES}\codeidxdef{PLUGIN\_DEPENDENCIES}
     other plug-ins that must be loaded before the $plugin$.
     By default, there is no dependency.
  \end{itemize}
  \textbf{Remark:} the $plugin$ package name is defined to be \texttt{frama-c-$plugin$} with 
  lowercased plugin name. You can refer to it directly with \texttt{ocamlfind}.
\item Variables \texttt{PLUGIN\_CMO}\codeidxdef{PLUGIN\_CMO}
  and \texttt{PLUGIN\_CMI}\codeidxdef{PLUGIN\_CMI} are
  respectively \texttt{.cmo} plug-in files and \texttt{.cmi} files without
  corresponding \texttt{.cmo} plug-in files. For each of them, do not write
  their file path nor their file extension: they are automatically added
  (\texttt{\$(PLUGIN\_DIR)/$f$.cm[io]} for a file $f$).
\item Variable
  \texttt{PLUGIN\_TYPES\_CMO}\codeidxdef{PLUGIN\_TYPES\_CMO}
  is the \texttt{.cmo} plug-in files which do not belong to
  \texttt{\$(PLUGIN\_DIR)}. They usually belong to
  \texttt{src/plugins/$plugin$\_types} where $plugin$ is the plug-in name (see
  Section~\ref{adv:static-registration}). Do not 
  write file extension (which is \texttt{.cmo}): it is automatically
  added.\index{Plug-in!Types}
\item Variable
  \texttt{PLUGIN\_GUI\_CMO}\codeidxdef{PLUGIN\_GUI\_CMO} is
  the \texttt{.cmo} plug-in files which have to be linked with the GUI (see
  Section~\ref{adv:gui}). As for variable \texttt{PLUGIN\_CMO}, do not write
  their file path nor their file extension.\index{Plug-in!GUI}
%
\item Variables of the form \texttt{PLUGIN\_*\_FLAGS} are
  plug-in specific flags for \texttt{ocamlc}, \texttt{ocamlopt},
  \texttt{ocamldep} or \texttt{ocamldoc}.
  \codeidxdef{PLUGIN\_BFLAGS}
  \codeidxdef{PLUGIN\_OFLAGS}
  \codeidxdef{PLUGIN\_EXTRA\_BYTE}
  \codeidxdef{PLUGIN\_EXTRA\_OPT}
  \codeidxdef{PLUGIN\_EXTRA\_DIRS}
  \codeidxdef{PLUGIN\_LINK\_OFLAGS}
  \codeidxdef{PLUGIN\_LINK\_GUI\_BFLAGS}
  \codeidxdef{PLUGIN\_LINK\_GUI\_OFLAGS}
  \codeidxdef{PLUGIN\_DEPFLAGS}
  \codeidxdef{PLUGIN\_DOCFLAGS}
\end{itemize}
%%%%
\begin{figure}[htbp]
\begin{center}
\begin{tabularx}{\textwidth}{|c|c|X|c|}
\hline
\textbf{Kind} & \textbf{Name} & \textbf{Specification} & \\
\hline
\hline \multirow{3}{27mm}{\centering{Dependencies}}
& \texttt{PLUGIN\_DEPFLAGS} & Plug-in specific flags for \texttt{ocamldep} &\\
& \texttt{PLUGIN\_GENERATED} & Plug-in files to be generated before running
\texttt{ocamldep} &\\
& \texttt{PLUGIN\_DEPENDS} & Other plug-ins to be compiled before the considered
one & no \\
\hline \multirow{5}{27mm}{\centering{Documentation}}
& \texttt{PLUGIN\_DOCFLAGS} & Plug-in specific flags for \texttt{ocamldoc}& \\
& \texttt{PLUGIN\_UNDOC} & Source files with no provided documentation &\\
& \texttt{PLUGIN\_TYPES\_TODOC} & Additional source files to document &\\
& \texttt{PLUGIN\_INTRO} & Text file to append to the plug-in introduction & \\
& \texttt{PLUGIN\_HAS\_EXT\_DOC} & Whether the plug-in has an external
pdf manual &\\
\hline \multirow{7}{27mm}{\centering{Testing}}
& \texttt{PLUGIN\_NO\_TESTS} & Whether there is no plug-in specific test
directory & \\
& \texttt{PLUGIN\_TESTS\_DIRS} & Directories containing tests &\\
& \texttt{PLUGIN\_TESTS\_DIRS\_DEFAULT} & tests to be included in the default
test suite (all directories by default) & no \\
& \texttt{PLUGIN\_TESTS\_LIBS} & Specific \texttt{.ml} files used by plug-in
tests &\\
& \texttt{PLUGIN\_NO\_DEFAULT\_TEST} & Whether to include tests
in default test suite. &\\
& \texttt{PLUGIN\_INTERNAL\_TEST} & Whether the test suite of the
plug-in is located in \framac's own tests directory &\\
& \texttt{PLUGIN\_PTESTS\_OPTS} & Plug-in specific options to ptests &\\
\hline \multirow{3}{25mm}{\centering{Distribution}}
& \texttt{PLUGIN\_DISTRIBUTED\_BIN} & Whether to include the plug-in
in binary distribution & no \\
& \texttt{PLUGIN\_DISTRIBUTED} & Whether to include the plug-in in
source distribution & no \\
& \texttt{PLUGIN\_DISTRIB\_EXTERNAL} & Additional files to be included
in the distribution & no \\
\hline
\end{tabularx}
\end{center}
\caption{Special parameters of \texttt{Makefile.dynamic}.}\label{fig:special-parameters}
\end{figure}
\begin{itemize}
\item Variable
  \texttt{PLUGIN\_GENERATED}\codeidxdef{PLUGIN\_GENERATED} is
  files which must be generated before computing plug-in dependencies. In
  particular, this is where \texttt{.ml} files generated by \texttt{ocamlyacc}
  and \texttt{ocamllex} must be placed if needed.
\item Variable
  \texttt{PLUGIN\_DEPENDS}\codeidxdef{PLUGIN\_DEPENDS} is the
  other plug-ins which must be compiled before the considered plug-in. 

  \begin{important}
    Using this variable is deprecated: you should consider to use
    \texttt{PLUGIN\_DEPENDENCIES} instead.
  \end{important}
%
\item Variable
  \texttt{PLUGIN\_UNDOC}\codeidxdef{PLUGIN\_UNDOC} is the
  source files for which no documentation is provided. Do not write their file
  path which is automatically set to
  \texttt{\$(PLUGIN\_DIR)}.\index{Plug-in!Documentation}
\item Variable \texttt{PLUGIN\_TYPES\_TODOC}%
  \codeidxdef{PLUGIN\_TYPES\_TODOC} is the additional source
  files to document with the plug-in. They usually belong to
  \texttt{src/plugins/$plugin$\_types} where $plugin$ is the plug-in name (see
  Section~\ref{adv:static-registration}).
  \index{Plug-in!Documentation}\index{Plug-in!Types}
\item Variable
  \texttt{PLUGIN\_INTRO}\codeidxdef{PLUGIN\_INTRO} is the
  text file to append to the plug-in documentation introduction. Usually this
  file is \texttt{doc/code/intro\_$plugin$.txt} for a plug-in $plugin$. It can
  contain any text understood by \texttt{ocamldoc}.\index{Plug-in!Documentation}
\item Variable
 \texttt{PLUGIN\_HAS\_EXT\_DOC}%
 \codeidxdef{PLUGIN\_HAS\_EXT\_DOC} is set to yes if
 the plug-in has its own reference manual. It is supposed to be a
 \texttt{pdf} file generated by \texttt{make} in directory
 \texttt{doc/\$(PLUGIN\_NAME)}\index{Plug-in!Documentation}
%
\item Variable
  \texttt{PLUGIN\_NO\_TEST}\codeidxdef{PLUGIN\_NO\_TEST} must
  be set to \yes if there is no specific test directory for the
  plug-in.\index{Plug-in!Test}
\item Variable
  \texttt{PLUGIN\_TESTS\_DIRS}\codeidxdef{PLUGIN\_TESTS\_DIRS}
  is the directories containing plug-in tests. Its default value is
  \texttt{tests/\$(notdir \$(PLUGIN\_DIR))}).\index{Plug-in!Test}
\item Variable
  \texttt{PLUGIN\_TESTS\_LIB}\codeidxdef{PLUGIN\_TESTS\_LIB}
  contains the \texttt{.ml} plug-in specific files used by plug-in
  tests.\index{Plug-in!Test} Note that if you wish to have these files
  compiled before launching the tests (as opposed to e.g. compile them via
  an \verb|EXECNOW| directive in a test file), you should add the 
  corresponding \verb|.cmo| and \verb|.cmxs| files as pre-requisite
  of the \verb|$(PLUGIN_DIR)/tests/ptests_config| %$
  target.
\item Variable
  \texttt{PLUGIN\_NO\_DEFAULT\_TEST}%
  \codeidxdef{PLUGIN\_NO\_DEFAULT\_TEST} indicates
  whether the test directory of the plug-in should be considered when
  running \framac default test suite. When set to a non-empty value,
  the plug-in tests are run only through \texttt{make
    \$(PLUGIN\_NAME)\_tests}.\index{Plug-in!Test}
\item Variable
  \texttt{PLUGIN\_INTERNAL\_TEST}%
  \codeidxdef{PLUGIN\_INTERNAL\_TEST} indicates
  whether the tests of the plug-in are included in \framac's own tests
  directory. When set to a non-empty value, the tests are searched
  there. When unset, tests are assumed to be in the \texttt{tests}
  directory of the plugin main directory itself. Having the tests of a
  plugin inside \framac's own \texttt{tests} suite is
  deprecated. Plugins should be self-contained.
\item Variable
  \texttt{PLUGIN\_PTESTS\_OPTS}%
  \codeidxdef{PLUGIN\_PTESTS\_OPTS} allows to give specific options to
  \texttt{ptests} when running the tests. It comes in addition to
  \texttt{PTESTS\_OPTS} 
  (see~\ref{make:sections}\textsection\ref{make:sec:tests}). 
  For instance, 
  \verb|PLUGIN_PTESTS_OPTS:=-j 1| will deactivate parallelization of tests in case
  the plugin does not support concurrent runs.
%
\item Variable
\texttt{PLUGIN\_DISTRIB\_BIN}%
\codeidxdef{PLUGIN\_DISTRIB\_BIN}
indicates whether the plug-in should be included in a binary distribution.%
\index{Plug-in!Distribution}
\item Variable
\texttt{PLUGIN\_DISTRIBUTED}%
\codeidxdef{PLUGIN\_DISTRIBUTED}
indicates whether the plug-in should be included in a source distribution.%
\index{Plug-in!Distribution}
\item Variable
\texttt{PLUGIN\_DISTRIB\_EXTERNAL}%
\codeidxdef{PLUGIN\_DISTRIB\_EXTERNAL}
is the list of files that should be included within the source
distribution for this plug-in. They will be put at their proper place for a
release.\index{Plug-in!Distribution}
\end{itemize}
As previously said, the above variables are set before including
\texttt{Makefile.dynamic} in order to customize its behavior. They
must not be use anywhere else in the Makefile. In order to deal with this
issue, for each plug-in $p$, \texttt{Makefile.dynamic} provides some variables
which may be used after its inclusion defining $p$. These variables are listed
in Figure~\ref{fig:make-defvars}. For each variable of the form
\texttt{$p$\_VAR}, its behavior is exactly equivalent to the value of the
parameter \texttt{PLUGIN\_VAR} for the plug-in $p$\footnote{Variables of the
  form \texttt{$p$\_*CMX} have no \texttt{PLUGIN\_*CMX} counterpart but their
  meanings should be easy to understand.}.
\begin{figure}[htbp]
\begin{center}
\begin{tabularx}{\textwidth}{|c|c|X|}
\hline
\textbf{Kind} & \textbf{Name}\footnotemark & \textbf{Remarks}\\
\hline
\hline \multirow{1}{35mm}{\centering{Usual information}}
& \texttt{$plugin$\_DIR}\index{plugin\_DIR@\texttt{$plugin$\_DIR}|bfit} & \\
\hline \multirow{5}{35mm}{\centering{Source files}}
& \texttt{$plugin$\_CMO}\index{plugin\_CMO@\texttt{$plugin$\_CMO}|bfit} & \\
& \texttt{$plugin$\_CMI}\index{plugin\_CMI@\texttt{$plugin$\_CMI}|bfit} & \\
& \texttt{$plugin$\_CMX}\index{plugin\_CMI@\texttt{$plugin$\_CMI}|bfit} & \\
& \texttt{$plugin$\_TYPES\_CMO}%
\index{plugin\_TYPES\_CMO@\texttt{$plugin$\_TYPES\_CMO}|bfit} & \\
& \texttt{$plugin$\_TYPES\_CMX}%
\index{plugin\_TYPES\_CMX@\texttt{$plugin$\_TYPES\_CMX}|bfit} & \\
\hline \multirow{10}{35mm}{\centering{Targets}}
& \texttt{$plugin$\_TARGET\_CMO}%
\index{plugin\_TARGET_CMO@\texttt{$plugin$\_TARGET\_CMO}|bfit} & \\
& \texttt{$plugin$\_TARGET\_CMX}%
\index{plugin\_TARGET\_CMX@\texttt{$plugin$\_TARGET\_CMX}|bfit} & \\
& \texttt{$plugin$\_TARGET\_CMA}%
\index{plugin\_TARGET\_CMA@\texttt{$plugin$\_TARGET\_CMA}|bfit} 
& Empty if plug-in does not have external dependencies
\\
& \texttt{$plugin$\_TARGET\_CMXA}%
\index{plugin\_TARGET\_CMXA@\texttt{$plugin$\_TARGET\_CMXA}|bfit} 
& Empty if plug-in does not have external dependencies
\\
& \texttt{$plugin$\_TARGET\_CMXS}%
\index{plugin\_TARGET\_CMXS@\texttt{$plugin$\_TARGET\_CMXS}|bfit}
& Empty if plugin is not dynamic
\\
& \texttt{$plugin$\_TARGET\_GUI\_CMO}%
\index{plugin\_TARGET\_GUI\_CMO@\texttt{$plugin$\_TARGET\_GUI\_CMO}|bfit}
& \rdelim\}{5}{\hsize}[\parbox{\hsize-7mm}{Empty if there is no plugin-specific GUI code}]
\\
& \texttt{$plugin$\_TARGET\_GUI\_CMX}%
\index{plugin\_TARGET\_GUI\_CMX@\texttt{$plugin$\_TARGET\_GUI\_CMX}|bfit}
&
\\
& \texttt{$plugin$\_TARGET\_GUI\_CMA}%
\index{plugin\_TARGET\_GUI\_CMA@\texttt{$plugin$\_TARGET\_GUI\_CMA}|bfit}
& \\
& \texttt{$plugin$\_TARGET\_GUI\_CMXA}%
\index{plugin\_TARGET\_GUI\_CMXA@\texttt{$plugin$\_TARGET\_GUI\_CMXA}|bfit}
& \\
& \texttt{$plugin$\_TARGET\_GUI\_CMXS}%
\index{plugin\_TARGET\_GUI\_CMXS@\texttt{$plugin$\_TARGET\_GUI\_CMXS}|bfit}
& \\
\hline \multirow{6}{35mm}{\centering{Compilation flags}}
& \texttt{$plugin$\_BFLAGS}%
\index{plugin\_BFLAGS@\texttt{$plugin$\_BFLAGS}|bfit} & \\
& \texttt{$plugin$\_OFLAGS}%
\index{plugin\_OFLAGS@\texttt{$plugin$\_OFLAGS}|bfit} & \\
& \texttt{$plugin$\_LINK\_BFLAGS}%
\index{plugin\_LINK\_BFLAGS@\texttt{$plugin$\_LINK\_BFLAGS}|bfit} & \\
& \texttt{$plugin$\_LINK\_OFLAGS}%
\index{plugin\_LINK\_OFLAGS@\texttt{$plugin$\_LINK\_OFLAGS}|bfit} & \\
& \texttt{$plugin$\_LINK\_GUI\_BFLAGS}%
\index{plugin\_LINK\_GUI\_BFLAGS@\texttt{$plugin$\_LINK\_GUI\_BFLAGS}|bfit} & \\
& \texttt{$plugin$\_LINK\_GUI\_OFLAGS}%
\index{plugin\_GUI\_OFLAGS@\texttt{$plugin$\_GUI\_OFLAGS}|bfit} & \\
\hline \multirow{2}{35mm}{\centering{Dependencies}}
& \texttt{$plugin$\_DEPFLAGS}%
\index{plugin\_DEPFLAGS@\texttt{$plugin$\_DEPFLAGS}|bfit} & \\
& \texttt{$plugin$\_GENERATED}%
\index{plugin\_GENERATED@\texttt{$plugin$\_GENERATED}|bfit} & \\
\hline \multirow{2}{35mm}{\centering{Documentation}}
& \texttt{$plugin$\_DOCFLAGS}%
\index{plugin\_DOCFLAGS@\texttt{$plugin$\_DOCFLAGS}|bfit} & \\
& \texttt{$plugin$\_TYPES\_TODOC}%
\index{plugin\_TYPES\_TODOC@\texttt{$plugin$\_TYPES\_TODOC}|bfit} & \\
\hline \multirow{2}{35mm}{\centering{Testing}}
& \texttt{$plugin$\_TESTS\_DIRS}%
\index{plugin\_TESTS\_DIRS@\texttt{$plugin$\_TESTS\_DIRS}|bfit} & \\
& \texttt{$plugin$\_TESTS\_LIB}%
\index{plugin\_TESTS\_LIB@\texttt{$plugin$\_TESTS\_LIB}|bfit} & \\
\hline
\end{tabularx}
\end{center}
\caption{Variables defined by \texttt{Makefile.dynamic}.}\label{fig:make-defvars}
\end{figure}
\footnotetext{$plugin$ is the module name of the considered plug-in (\emph{i.e.}
  as set by
  \texttt{\$(PLUGIN\_NAME)})\codeidx{PLUGIN\_NAME}.}

\subsection{\texttt{.Makefile.user}}
\codeidx{.Makefile.user}

The following variables can be set inside \texttt{.Makefile.user}:
\begin{itemize}
\item \texttt{FRAMAC\_PARALLEL}: the contents of this variable are passed
to \texttt{ptests} when it is called by \framac's Makefile. This variable
can be used to override the default value of 4.

\emph{Example value:} \texttt{-j 6}

\item \texttt{OCAML\_ANNOT\_OPTION}: this variable of the Makefile
can be overridden in \texttt{.Makefile.user}.
By default, it is set to \texttt{-annot -bin-annot}. Users of
\textsf{Merlin}\index{Merlin} do not need \texttt{-annot}.

\emph{Example value:} \texttt{-bin-annot}

\item \texttt{FRAMAC\_USER\_FLAGS}: the contents of this variable are
passed to \texttt{ocamlc} and \texttt{ocamlopt}, \emph{after} the standard
configuration options coming from the Makefile. It can be used to tweak
the warnings emitted by OCaml, and whether they are emitted as errors.

\emph{Example value:} \texttt{-warn-error -26-27} (do not consider warnings
26 and 27 as fatal errors)

\item \texttt{FRAMAC\_USER\_MERLIN\_FLAGS}: the contents of this variable are
passed to a directive \texttt{FLG}, inside the \texttt{.merlin} file generated
for \framac. (Through  the \texttt{merlin} target of the
main \texttt{Makefile}.) It can be used to tailor \textsf{Merlin}\index{Merlin}
to your needs. See \texttt{ocamlmerlin -help} for the list of flags.


\end{itemize}

\subsection{\texttt{Makefile.dynamic}}
\label{make:dynamic}

\todo

\section{Ptests}\label{sec:ptests}
\subsection{Pre-defined macros for tests commands}\label{sec:ptests-macros}
Ptests pre-defines a certain number of macros for each test about to be run.
Figure~\ref{fig:ptests-macros} gives their definition.
\begin{longtable}{|p{4.5cm}|p{10cm}|}
\hline
Name & Expansion \\
\hline
\verb|frama-c| & path to \framac executable \\
\hline
\verb|PTEST_CONFIG| & either the empty string or \verb|_| followed by the
name of the current alternative configuration 
(see section~\ref{ptests:alternative}). \\
\hline
\verb|PTEST_DIR| & current test suite directory \\
\hline
\verb|PTEST_RESULT| & current result directory \\
\hline
\verb|PTEST_FILE| & path to the current test file \\
\hline
\verb|PTEST_NAME| & basename of the current test file (without suffix) \\
\hline
\verb|PTEST_NUMBER| & rank of current test in the test file. There are in
fact two independent numbering schemes: one for \verb|EXECNOW| commands and
one for regular tests (if more than one \verb|OPT|).\\
\hline
\verb|PTEST_RESULT| & shorthand alias to \verb|@PTEST_DIR@/result@PTEST_CONFIG@|
(the result directory dedicated to the tested configuration)\\
\hline
\verb|PTEST_ORACLE| & basename of the current oracle file (macro only usable in FILTER directives)\\
\hline
\verb|PTEST_MODULE| & current list of module defined by the \verb|MODULE| directive\\
\hline
\verb|PTEST_LIBS| & current list of module defined by the \verb|LIBS| directive\\
\hline
\verb|PTEST_PLUGIN| & current list of plugins defined by the \verb|PLUGIN| directive\\
\hline
\verb|PTEST_SCRIPT| & current list of plugins defined by the \verb|SCRIPT| directive\\
\hline
\verb|PTEST_DEFAULT_OPTIONS| & the default option list: \verb|-journal-disable| \verb|-check| \verb|-no-autoload-plugins|\\
\hline
\verb|PTEST_LOAD_MODULE| & the \verb|-load-module| option related to the \verb|MODULE| directive\\
\hline
\verb|PTEST_LOAD_LIBS| & the \verb|-load-module| option related to the \verb|LIBS| directive\\
\hline
\verb|PTEST_LOAD_PLUGIN| & the \verb|-load-module| option related to the \verb|PLUGIN| directive\\
\hline
\verb|PTEST_LOAD_SCRIPT| & the \verb|-load-script| option related to the \verb|SCRIPT| directive\\
\hline
\verb|PTEST_LOAD_OPTIONS| & shorthand alias to \spverb|@PTEST_LOAD_PLUGIN@ @PTEST_LOAD_LIBS@ @PTEST_LOAD_MODULE@ @PTEST_LOAD_SCRIPT@|\\
\hline
\verb|PTEST_OPTIONS| & the current list of options related to \verb|OPT| and \verb|STDOPT| directives (for \verb|CMD| directives)\\
\hline
\verb|frama-c| & shortcut defined as follows: \spverb|@frama-c-exe@ @PTEST_DEFAULT_OPTIONS@ @PTEST_LOAD_OPTIONS@|\\
\hline
\verb|frama-c-cmd| & shortcut defined as follows: \spverb|@frama-c-exe@ @PTEST_DEFAULT_OPTIONS@|\\
\hline
\verb|frama-c-exe| & set to the value of the \verb|TOPLEVEL_PATH| variable from \verb|./tests/ptests_config| file\\
\hline
\caption{Predefined macros for ptests}\label{fig:ptests-macros}
\end{longtable}

\section{Profiling with Landmarks} \label{refman:landmarks}\codeidxdef{Landmarks}

{\em Landmarks}\footnote{\url{https://github.com/LexiFi/landmarks}} is a
library for ``quick and dirty'' profiling of OCaml programs. It allows the
insertion of annotations in the code to enable profiling of specific parts of
it, but also an automatic mode, in which every function call is instrumented.
The Frama-C \texttt{configure} file is setup to enable usage of this library
when it is available (the usual way to install it is via the \texttt{landmarks}
opam package).

For quick usage of the library:
\begin{itemize}
\item ensure that the \texttt{configure} script detected it
  (there should be a line \texttt{checking for Landmarks... found});
\item enable instrumentation {\em when compiling Frama-C's files}, that is,
  when running \texttt{make}, by setting the environment variable
  \verb+OCAML_LANDMARKS+. For instance, to enable automatic instrumentation
  of every Frama-C function (note: this increases compilation time of Frama-C),
  run:

\begin{lstlisting}
    OCAML_LANDMARKS=auto make
\end{lstlisting}

\item enable instrumentation {\em during execution} of Frama-C, again using
  \verb+OCAML_LANDMARKS+. Note that the \texttt{auto} parameter here is
  implicit if you enabled it on the previous step.
  For instance, run:

  \begin{lstlisting}
    OCAML_LANDMARKS= bin/frama-c [files] [options]
  \end{lstlisting}
\end{itemize}

Commonly used options include \verb+output=landmarks.log+ to output the result
to a file instead of \texttt{stderr}.

To instrument a single file: add \verb+[@@@landmark "auto"]+ at the beginning
of the file.

To instrument a single function: add \verb+[@landmark]+ after the \texttt{let},
e.g.:

\begin{lstlisting}
  let[@landmark] add_visitor vis =
\end{lstlisting}

Check \url{https://github.com/LexiFi/landmarks} for its documentation.

\textbf{Note:} if you intend to use \texttt{ocamlprof} (via \texttt{ocamlcp} or
\texttt{ocamloptp}), which does not support ppx extensions, and you have
Landmarks installed, you need to explicitly disable Landmarks during Frama-C's
configure: \verb+./configure --disable-landmarks+.

%%%%%%%%%%%%%%%%%%%%%%%%%%%%%%%%%%%%%%%%%%%%%%%%%%%%%%%%%%%%%%%%%%%%%%%%%%%%%%%
%%% Local Variables:
%%% TeX-master: "developer.tex"
%%% ispell-local-dictionary: "english"
%%% End:
