\chapter{Marquage générique}\label{sec-mark}

\section{Objectif}

On souhaite pouvoir associer une information à certains éléments d'un PDG,
propager cette information dans les dépendances de ces éléments,
et calculer ce qu'il faut propager dans les autres PDG de l'application
pour avoir un traitement interprocédural cohérent. \\

Par exemple, l'information peut être un simple booléen qui indique si l'élément
correspondant est visible ou non. Pour qu'un élément soit visible, il faut que
toutes ses dépendances le soit aussi.
On voit dans cet exemple qu'on s'intéresse dans un premier temps à une
propagation arrière.\\

Comme un PDG est associé à une fonction, la propagation s'arrête aux entrées
de la fonction. Par ailleurs, les appels de fonction sont traités localement,
sans descendre dans la fonction appelée.
Pour gérer la propagation interprocédurale, les fonctions élémentaires de
marquage doivent fournir l'information à propager aussi bien dans les fonctions
appelées que dans les appelants.

\section{Marquage générique}

Le marquage pouvant être utilisé pour différents besoins,
les marques peuvent être définies par l'utilisateur.
Il lui suffit de définir comment en faire l'union.

On utilise deux types d'union~:
\begin{itemize}
  \item l'une permet de faire une simple union de deux marques,
  \item l'autre permet d'ajouter une marque à un élément de programme~:
    elle calcule donc une nouvelle marque à partir de l'ancienne,
    et de la marque à ajouter et fournit aussi la marque à propager dans les
    dépendances.
\end{itemize}

\section{Propagation arrière}

\subsection{Marquage intraprocédural}

La propagation arrière s'effectue très simplement en suivant les dépendances,
en ajoutant la marque propagée à la marque existante,
et en s'arrêtant quand la marque à propager rendue par la fonction de marquage
est $m_{\bot}$.\\

En cours de propagation, lorsque l'on traite un élément correspondant à la
sortie d'un appel de fonction, ou à une entrée de la fonction courante,
la marque à propager ainsi que l'identifiant de l'élément sont stockés.
A l'issu du marquage, on a donc~:
\begin{itemize}
  \item un ensemble de couple (entrée de la fonction, marque à propager),
  \item une liste des appels de fonction, avec pour chacun un ensemble de
    couples (sortie de l'appel, marque à propager).
\end{itemize}

C'est cette information qui permettra de faire la propagation interprocédurale.

\subsection{Propagation interprocédurale}

\subsubsection{Méthode automatique}

Le module de marquage permet de faire automatiquement
la propagation interprocédurale si on le souhaite.
Il est utilisé de cette façon dans la détection de code mort par exemple.\\

L'utilisateur peut, s'il le souhaite,
donner des fonctions de transformation des marques~:
\begin{itemize}
  \item une qui est utilisée pour transformer les marques des entrées d'une
    fonction avant la propagation dans les appelants,
  \item une autre qui est utilisée pour transformer les marques des sorties
    d'un appel de fonction avant propagation dans la fonction appelée.
\end{itemize}
Dans les cas les plus simples, ces fonctions ne font rien, et rendent
simplement la marque qui leur est passée.

\subsubsection{Méthode manuelle}

Dans certains cas, il est souhaitable de gérer cette propagation à la main.
C'est par exemple le cas dans le \slicing{} où une fonction source est associée
à plusieurs marquages différents.\\

Des fonctions aident néanmoins à retrouver les éléments à marquer dans les
autres fonctions à partir du résultat fourni par le marquage intraprocédural.
Ces fonctions traduisent les informations~:
\begin{itemize}
  \item (entrée de la fonction, marque à propager) en un ensemble d'éléments à
    marquer dans une~- ou toutes les~- fonction(s) appelante(s),
  \item (appel, \{(sortie de l'appel, marque à propager)\})
    en un ensemble d'éléments à marquer dans la fonction appelée.
\end{itemize}
Les éléments rendu sont les nouveaux points de départ de marquages
intraprocéduraux.
L'utilisateur peut, comme lors de la propagation automatique,
donner des fonctions de transformation des marques aux traductions.

\subsubsection{Données non définies}

Lorsque les points de départ marquage initial sont des éléments du PDG,
toutes les propagations s'effectueront a priori sur des éléments existant.
Mais ce n'est pas forcement le cas lorsqu'il s'agit de marquer les données
qui n'interviennent pas dans les calculs. Par exemple~:

\begin{exemple}
  \begin{verbatim}
int X1, X2;

void f (int x) { /*@ assert X1 >= 0; */ return x + 1; }

void f2 (void)  { X2 = f (2); }

void main (void) { X1 = 0; X2 = 0; f2 (); }
  \end{verbatim}

  Ici, si on s'intéresse à l'assertion de \verbtt{f}, il faut sélectionner
  \verbtt{X1} à l'entrée de \verbtt{f}, propager cette information à travers
  \verbtt{f2} pour enfin marquer l'affectation à \verbtt{X1} dans le \verbtt{main}
  alors que cette donnée n'apparaît pas dans les PDG de \verbtt{f} et \verbtt{f2}.
\end{exemple}

On voit donc que dans les informations fournies et calculées
pour gérer l'interprocédural, il peut y avoir des éléments ne correspondant
pas à des noeuds de PDG.

\subsubsection{Terminaison}

La terminaison du processus dépend de la définition des marques.
C'est donc à l'utilisateur de s'assurer que celle-ci permet la stabilisation du
marquage.

\section{Propagation avant}

La propagation avant n'est pas effectuée automatiquement à l'heure actuelle.

La propagation intraprocédurale ne pose pas de problèmes particuliers,
car elle est très semblable au calcul en arrière~: il suffit simplement de
parcourir les liens de dépendance dans l'autre sens.\\

La partie interprocédurale semble un peu plus délicate,
car il n'y a aucun point de repère dans le PDG
pour identifier les endroits à partir desquels il faut propager
(hormis le noeud correspondant au \verbtt{return}
pour la propagation vers les appelants et les arguments explicites pour
la propagation dans les appels de fonction).

Des fonctions supplémentaires sont donc fournies pour~:
\begin{itemize}
  \item trouver les noeuds d'entrées d'une fonction qui doivent être
    sélectionnés à partir des noeuds sélectionnés dans l'appelant,
  \item trouver les noeuds de sortie d'un appel de fonction
    qui doivent être
    sélectionnés à partir des noeuds sélectionnés dans la fonction appelée.
\end{itemize}

En cas d'utilisation intensive de ces fonctionnalités,
il serait sans doute intéressant de mémoriser les liens entre les
différents PDG (cela est également vrai pour la propagation interprocédurale
arrière...).\\
