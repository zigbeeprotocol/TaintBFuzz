\chapter{Aide à l'analyse d'impact}\label{sec-impact}

Le document \cite{baudinImpact} spécifie un outil d'aide à l'analyse d'impact
après avoir analysé les besoins dans ce domaine.
L'exploitation du graphe de dépendances est présenté comme étant 
un composant important de cet outil.
Nous reprenons ici les éléments de spécification 
décrits dans le paragraphe 5.3 de ce document,
auxquels nous ajoutons quelques nouveaux critères.\\


\section{Définition d'ensembles}\label{sec-ensembles}\label{sec-defR}

\newcommand{\remitem}[1]{\begin{itemize}\item #1 \end{itemize}}
\newcommand{\warnitem}[1]{\begin{itemize}\item[$\blacktriangle$] #1
                          \end{itemize}}
\newcommand{\question}[1]{\begin{itemize}\item[{\bf ?}] #1 \end{itemize}}
\newcommand{\smilitem}[1]{\begin{itemize}\item[$\smiley$] #1 \end{itemize}}
\newcommand{\frownitem}[1]{\begin{itemize}\item[$\frownie$] #1 \end{itemize}}


Il s'agit de définir des sous-ensembles d'instructions
répondant à différents critères.

Donnons tout d'abord quelques notations :
\begin{itemize}
\item S : un ensemble d'instructions,
\item L : un point de programme,
\item V : une zone mémoire (le V se réfère à {\bf V}ariable,
mais il peut s'agir d'une donnée quelconque comme un champ de structure ou
un élément de tableau),
\end{itemize}

Les ensembles sont tous relatifs à un point de programme $L$,
et peuvent se classer en deux catégories~:
\begin{itemize}
\item ceux qui contiennent des instructions situées {\bf avant} $L$~: 
ils portent un indice 0,
\item et ceux qui contiennent des instructions situées {\bf après} $L$~:
ils portent un indice 1.\\
\end{itemize}

\begin{description}
\item[$R_0(S,L)$]  : sous-ensemble d'instructions de S depuis lesquelles 
L est accessible.
\warnitem{mais L ne postdomine pas forcement toutes les instructions de cet ensemble.} 

\item[$R_{1}(S,L)$] : sous-ensemble d'instructions de S 
qui sont (éventuellement) accessibles depuis L.

\item[$R_{L0}(S,L)$] : accessibilité à un point du code.
\remitem{il s'agit du sous-ensemble des instructions de $R_0(S,L)$
qui conditionnent le passage en L,
c'est-à-dire les instructions de branchement 
et les dépendances de ces conditions de branchement.
}

%\item[$R_{L1}(S,L)$] : ?

\item[$R_{V0}(S,L,V)$] : contenu d'une zone mémoire en un point du code.
\remitem{il s'agit du critère traditionnel de \slicing,
c'est-à-dire que cet ensemble contient les instructions de $S$ qui 
ont une influence sur la valeur de $V$ en $L$.
}

\item[$R_{V1}(S,L,V)$] : utilisation d'une zone mémoire.
\remitem{il s'agit du sous-ensemble des instructions de $R_{1}(S,L)$
influencées par la valeur qu'à la zone mémoire $V$ a en $L$.}

\item[$R_{DV0}(S,L,V)$] : définition de la valeur d'une variable à un point de
programme.
\remitem{il s'agit de l'ensemble des instructions qui {\bf définissent}
tout ou partie de la valeur de $V$ en $L$. Ces instructions 
sont donc nécessairement des affectations ou des appels de fonction.

On note que~: $R_{DV0}(S,L,V) \subseteq R_{V0}(S,L,V)$
}
\warnitem{
Il serait probablement utile d'avoir un élément spécial 
qui indique qu'il existe 
un ou plusieurs chemin menant à $L$ pour le(s)quel(s) 
$V$ n'est pas entièrement défini.
}

\item[$R_{DV1}(S,L,V)$] : modification de la valeur d'une variable.
\remitem{il s'agit de trouver les instructions $I_i$ accessibles depuis L
qui modifient la valeur de $V$,
et telles qu'il existe un chemin entre $L$ et $I_i$
le long duquel $V$ n'est pas modifiée.
Cela signifie que ces instructions sont les premières à modifier la valeur
de $V$ sur les chemins partant de $L$.}
\remitem{Il n'est pas sûr que cet ensemble soit très utile.
L'ensemble $R_{P1}(S,L,V)$ ci-dessous en sans doute plus intéressant.
}

\item[$R_{P0}(S,L,V)$] : 
portée arrière de la valeur d'une variable à un point de programme.
\remitem{il s'agit d'un sous-ensemble de $R_{L0}(S,L)$
contenant les instructions $I_i$ pour lesquelles la valeur de $V$
n'a été modifiée sur aucun chemin entre le point qui précède $I_i$ et $L$.
C'est-à-dire que la valeur de $V$ en $L$ est la même qu'avant chacune de ces
instructions, et qu'elle n'est pas modifiée entre les deux.
}

\item[$R_{P1}(S,L,V)$] : 
portée de la valeur d'une variable à un point de programme~.
\remitem{il s'agit d'un sous-ensemble de $R_{1}(S,L)$
contenant les instructions $I_i$ pour lesquelles la valeur de $V$ 
n'a été modifiée sur aucun chemin entre $L$ et le point de programme 
qui précède $I_i$.
}

\item[$R_{PV0}(S,L,V)$] : utilisation arrière d'une zone mémoire dans sa portée.
\remitem{il s'agit du sous-ensemble des instructions $R_{P0}(S,L,V)$
qui sont influencées, directement ou non, par $V$. 
}

\item[$R_{PV1}(S,L,V)$] : utilisation d'une zone mémoire dans sa portée.
\remitem{il s'agit du sous-ensemble 
         des instructions influencées, directement ou non, 
         par la valeur de $V$ en $L$
         qui sont dans la portée de cette valeur.

         On a donc : $R_{PV1}(S,L,V) = R_{V1}(S,L,V) \cap R_{P1}(S,L,V)$
        }

\end{description}

\begin{exemple1}
On montre ici différents ensembles relatifs au point $L$
et à la variable $v$~:
\end{exemple1}

\begin{center}
\begin{scriptsize}
\begin{tabular}{|l||c|c|c|c|c|c||c|c|c|c|c| }
\hline
S & $R_{0}$ & $R_{L0}$ & $R_{V0}$ & $R_{DV0}$ & $R_{P0}$ & $R_{VP0}$
            & $R_{1}$ & $R_{V1}$ & $R_{DV1}$ & $R_{P1}$ & $R_{VP1}$ \\
\hline
\verb!!                      &   &   &   &   &   &   &   &   &   &   &  \\
\verb!v = a; !               & X & X & X & X &   &   &   &   &   &   &  \\
\verb!x = b; !               & X &   & X &   &   &   &   &   &   &   &  \\
\verb!z = c; !               & X &   &   &   &   &   &   &   &   &   &  \\
\verb!if (c1>v) { !          & X & X & X &   &   &   &   &   &   &   &  \\
\verb!  z++;    !            &   &   &   &   &   &   &   &   &   &   &  \\
\verb!  }       !            &   &   &   &   &   &   &   &   &   &   &  \\
\verb!else {    !            &   &   &   &   &   &   &   &   &   &   &  \\
\verb!  if (c2) !            & X & X & X &   &   &   &   &   &   &   &  \\
\verb!    v += x; !          & X & X & X & X &   &   &   &   &   &   &  \\
\verb!    if (c3) { !        & X & X & X &   & X &   &   &   &   &   &  \\
\verb!      z += v; !        & X & X &   &   & X & X &   &   &   &   &  \\
\hline
{\tt\bf L:}
  \verb!    y = v; !         &   &   &   &   &   &   & X & X &   & X & X \\
\verb!       z++;   !        &   &   &   &   &   &   & X &   &   & X &   \\
\verb!       v++; !          &   &   &   &   &   &   & X & X &   & X & X \\
\verb!       } !             &   &   &   &   &   &   &   &   &   &   &   \\
\verb!  z += 2*y; !          &   &   &   &   &   &   & X & X &   &   &   \\
\verb!  }      !             &   &   &   &   &   &   &   &   &   &   &   \\
\verb!z += x;  !             &   &   &   &   &   &   & X & X &   &   &   \\
\verb!v = 0; !               &   &   &   &   &   &   &   &   & X &   &   \\
\verb!!                      &   &   &   &   &   &   &   &   &   &   &   \\
\hline
\end{tabular}\end{scriptsize}\end{center}
\begin{exemple2}
\end{exemple2}

La construction des ensembles ci-dessus exploite le CFG (le flot de contrôle)
et le PDG (les dépendances).\\

La construction des ensembles ci-dessous exploite en plus la sémantique du
programme car elle utilise une contrainte $C(...v_i...)$ portant
sur les valeurs des données en $L$~:

\begin{description}
\item[$R_{C0}(S,L,C(...v_i...))$] :
accessibilité contrainte à un point.
\remitem{il s'agit de déterminer l'ensemble des instructions
qui, si elles sont présentes, rendent fausse la contrainte $C$ au point $L$.
En fait, il s'agit de déplacer $C$ pour essayer de couper des branches.
La condition de branchement sera alors éventuellement
remplacée par un \verbtt{assert}.
}
\smilitem{Dans la séquence \verbtt{int x = 0; x=x-1; L:} pour laquelle
on a une contrainte $x \ge 0$, il ne s'agit pas de supprimer l'instruction
\verbtt{x=x-1;} mais bien de dire que cette branche est impossible...}

\item[$R_{C1}(S,L,C(...v_i...))$] : accessibilité contrainte depuis un point.
\remitem{il s'agit de déterminer le sous-ensemble des instructions de
$R_{1}(S,L)$ qui ne peuvent pas être atteintes si l'état en L est tel que 
$C$ est satisfaite.\\


On notera que certaines instructions n'appartenant ni à $R_{L0}$, ni à $R_{1}$
peuvent également ne pas être atteintes du fait de la contrainte.\\

}
\end{description}

\section{Exploitation du graphe pour le calcul d'ensembles}

La plupart des ensembles définis en \ref{sec-ensembles}
peuvent être calculés simplement en exploitant le flot de contrôle
et les fonction du graphe de dépendances présentées en \$\ref{sec-find}.
