\chapter{Introduction}

\section{Objectif}

L'objectif initial a été de réaliser un module de \slicing
dans le cadre d'un outil généraliste d'analyse de programme (voir
\cite{ppcSlicing}).
Or, la plupart des réductions à effectuer se basent sur l'analyse des
dépendances entre les données du programme.
En effet, si l'utilisateur demande la sélection
d'une instruction \verb!return x;!, il va falloir retrouver ce qui permet de
calculer cette valeur de \verbtt{x} dans les instructions qui précèdent.\\

Il s'agit donc de calculer le \indexdef{graphe de dépendances} d'une fonction
(appelé \indexdef{PDG}~: {\it Program Dependence Graph} dans la littérature)
c'est-à-dire de représenter finement les liens de
dépendances entre les différentes instructions qui la composent.
Le résultat ce calcul est un graphe dans lequel les sommets
représentent les instructions,
éventuellement décomposées en plusieurs noeuds représentant
le calcul d'informations élémentaires.\\

Dans le cadre de l'outil de \slicing,
l'intérêt de ce calcul préalable est de pouvoir
travailler en plusieurs passes lors de l'application des
requêtes de réduction sans avoir à
refaire ce calcul qui peut être lourd (alias, dépendances partielles, etc). En
effet, même si l'on souhaite à terme calculer des réductions qui utilisent
davantage la sémantique du programme, les réductions à l'aide des dépendances
peuvent simplifier le problème.\\

Par la suite, il a été jugé intéressant de considérer ce calcul
comme un module à part entière car il peut avoir d'autres
utilités comme étudier la
propagation du flot d'information pour des analyses de sécurité, par exemple.

\section{Spécifications du PDG}\label{sec-flot}

Le PDG que l'on souhaite calculer comporte plusieurs types de dépendances~:

\subsection{Dépendances sur la valeur des données}

Les \indextxtdef{dépendances sur la valeur}{dépendance!valeur}
d'une donnée sont les plus intuitives.

\begin{exemple}
  \begin{tabular}{m{3.5cm}m{\linewidth-4.3cm}}
\begin{clisting}
x = a + b;
\end{clisting}
&
Ici, \verbtt{x} dépend de
\verbtt{a} et \verbtt{b} car la valeur de \verbtt{x} après cette instruction dépend
des valeurs de \verbtt{a} et \verbtt{b} avant.
\end{tabular}
\end{exemple}

La question se pose néanmoins de définir la granularité à laquelle
on s'intéresse aux données.
L'utilisation d'autres analyses et structures de données de \ppc
conduit à choisir la même précision
(pour plus de détail, voir par exemple l'analyse de valeurs de \ppc).

\subsection{Dépendances de calcul d'adresse}

Pour les affectations, la valeur qui est écrite en mémoire dépend de la partie
droite, mais le choix de l'adresse à laquelle on l'écrit peut également dépendre
de variables qui apparaissent dans la partie gauche.

\begin{exemple}
\begin{tabular}{m{3.5cm}m{\linewidth-4.3cm}}
\begin{clisting}
*p = x;
\end{clisting}
&
Ici, la valeur de la donnée modifiée dépend de \verbtt{x},
mais le choix de la case dans laquelle on la range dépend de \verbtt{p}.
En effet, si $p$ pointe sur $a$ ou $b$, c'est soit  \verbtt{a} soit \verbtt{b} qui
va être modifié. 
\end{tabular}
\end{exemple}

On parle alors de \indextxtdef{dépendances sur l'adresse}{dépendance!adresse}.

\subsection{Dépendances de contrôle}

Lorsqu'une donnée peut-être modifiée par plusieurs chemins d'exécution,
elle dépend des conditions qui permettent de choisir le chemin.

\begin{exemple}
\begin{tabular}{m{3.5cm}m{\linewidth-4.3cm}}
\begin{clisting}
if (c) 
  x = a;
L :
\end{clisting}
& 
La valeur de \verbtt{x} en \verbtt{L} dépend de \verbtt{a}, mais aussi de \verbtt{c}.\\
\end{tabular}
\end{exemple}

Il s'agit d'une \indextxtdef{dépendance de contrôle}{dépendance!contrôle}.

\subsection{Dépendances sur les déclarations}

Lorsque des variables sont utilisées dans une instruction, celle-ci dépend de
leurs déclarations, car si on veut la compiler, il faut que les variables
existent.

Les déclarations des variables lues (partie droite d'une affectation) sont
considérées comme participant au calcul de la valeur.
Les déclarations des
variables utilisées pour déterminer la case affectée (partie gauche d'une
affectation) sont considérées comme des dépendances sur l'adresse.

\begin{exemple}
\begin{tabular}{m{3.5cm}m{\linewidth-4.3cm}}
\begin{clisting}
/* 1 */ int x;
/* 2 */ int y;
...
/* i */ x = 3;
/* j */ y = 4;
...
/* n */ x = y;
\end{clisting}
&
L'instruction (n) a une dépendance d'adresse sur la déclaration de x (1),
et des dépendances de donnée sur l'affectation de y (j)
et sa déclaration (2). Dans ce cas, on aurait pu se passer de de cette dernière
dépendance car (j) dépend déjà de (2), mais ce n'est pas forcement le cas
en présence d'alias.
\end{tabular}
\end{exemple}

\subsection{Résumé}

On voit donc qu'on distingue trois types de dépendances~:
\begin{itemize}
\item les calculs de valeurs,
\item les calculs d'adresses,
 \item le contrôle.
\end{itemize}


\section{État de l'art}\label{sec-lart}

Voyons tout d'abord ce que dit la littérature sur ce sujet
afin de voir les solutions qui peuvent répondre à notre besoin,
et les points à modifier.

\subsection{Origine}

Les graphes de dépendances ont principalement été étudiés dans le cadre du
\slicing{}, mais ils sont aussi utilisés dans les travaux sur la compilation.

\subsection{Graphes de dépendances}

\cite{Ottenstein84}, puis \cite{Ferrante87}
introduisent la notion de PDG ({\it Program Dependence Graph}).
Un tel graphe est utilisé pour représenter les différentes dépendances
entre les instructions d'un programme.
Ils l'exploitent pour calculer les instructions
qui influencent la valeur des variables en un point.\\

Cette représentation, initialement intraprocédurale, a été étendue
à l'analyse interprocédurale dans \cite{horwitz88interprocedural} où elle
porte le nom de SDG ({\it System Dependance Graph}).
Elle est maintenant, à quelques variantes près,
quasiment universellement utilisée.

\subsection{Exploitation du graphe}

Lorsque l'on utilise le graphe de dépendance pour faire du \slicing,
le calcul se résume à un problème d'accessibilité à un noeud car comme le dit
Susan Horwitz~:

\begin{definition}{slicing selon \cite{horwitz88interprocedural}}
A slice of a program with respect to program point p and variable x
consists of a set of statements of the program that might affect
the value of x at p.
\end{definition}

c'est-à-dire qu'il faut garder toutes les instructions correspondant à des
noeuds du graphe pour lesquels il existe un chemin vers le noeud représentant le
calcul de {\tt x} en {\tt p}.
Mais le problème est que ce noeud n'existe que si {\tt x} est défini
par l'instruction située en {\tt p}.

Nous verrons que cette limitation peut être levée si l'on garde
(ou recalcule) les structures de données utilisées
lors de la construction du graphe.\\

Le traitement des appels de fonction est souvent compliqué par le fait qu'il
est également ramené à un problème d'accessibilité dans un graphe qui,
cette fois, représente toute l'application. Or, dans un tel graphe,
si on ne prend pas de précautions supplémentaires,
on parcourt des chemins impossibles qui entrent dans une fonction
par un appel, et sortent par un autre site d'appel.
Ces chemins existent en effet dans le graphe,
mais pas dans la réalité.

Comme nous nous proposons de traiter ce problème de manière modulaire,
nous devrions échapper à une partie de ce problème.
Mais nous verrons par la suite que l'utilisation d'une analyse d'alias globale
produit néanmoins quelques effets de bord indésirables.

\subsection{Programmes non structurés}

Les premiers algorithmes utilisés fonctionnent
correctement sur des programmes structurés,
mais produisent des résultats erronés en présence de \verbtt{goto}.

Le problème vient du fait que ces instructions ne modifient pas de données~:
il n'y a donc pas de dépendance de donnée~; et il n'y a pas
de dépendance de contrôle non plus car dans le CFG,
une seule branche sort du noeud pour aller vers le point de branchement.

Plusieurs personnes (\cite{Choi94},  \cite{agrawal94slicing},
\cite{harman98new},  \cite{kumar02better} entre autres)
se sont donc intéressées
aux sauts (\verbtt{goto}) qui brisaient la structure du programme.
Ce point, qui nous intéresse tout particulièrement, est présenté en détail
en \S\ref{sec-cdg}.

\subsection{Pointeurs et données structurées}

De nombreux articles s'intéressent aux traitements des données structurées,
et plus encore des pointeurs. Dans le cadre de cette étude,
nous n'avons pas exploré ces recherches étant donné que
nous nous appuyons déjà sur une analyse d'alias précise.


%~~~~~~~~~~~~~~~~~~~~~~~~~~~~~~~~~~~~~~~~~~~~~~~~~~~~~~~~~~~~~~~~~~~~~~~~~~~~~~~
\section{Plan}

Les trois chapitres suivants
exposent comment est calculé notre graphe de dépendances.

Puis, nous verrons au chapitre \ref{sec-find} comment
exploiter les informations calculées et au chapitre
\ref{sec-mark} comment associer des informations aux éléments
et les propager dans le graphe.


%~~~~~~~~~~~~~~~~~~~~~~~~~~~~~~~~~~~~~~~~~~~~~~~~~~~~~~~~~~~~~~~~~~~~~~~~~~~~~~~
