%% LyX 2.2.2 created this file.  For more info, see http://www.lyx.org/.
%% Do not edit unless you really know what you are doing.
\documentclass[english,dvipsnames]{article}
\usepackage[T1]{fontenc}
\usepackage[latin9]{inputenc}
\usepackage{color}
\usepackage{babel}
\usepackage[a4paper]{geometry}
\usepackage[unicode=true,pdfusetitle,
 bookmarks=true,bookmarksnumbered=false,bookmarksopen=false,
 breaklinks=false,pdfborder={0 0 1},backref=false,colorlinks=false]
 {hyperref}
\usepackage{breakurl}

\makeatletter
%%%%%%%%%%%%%%%%%%%%%%%%%%%%%% User specified LaTeX commands.
\usepackage{tikz}
\usetikzlibrary{positioning,fit,calc,shapes,arrows,backgrounds}

\usepackage{ragged2e}

\makeatother

\usepackage{listings}
\lstset{language={[ANSI]C},
basicstyle={\footnotesize},
tabsize=4,
frame=single,
mathescape=true,
moredelim={**[is][\color{red}]{@}{@}},
keywordstyle={\color{RoyalBlue}},
basicstyle={\small\ttfamily},
commentstyle={\color{webgreen}\ttfamily},
stringstyle={\rmfamily},
numbers=none,
showstringspaces=false,
breaklines=true,
frameround=ftff}
\renewcommand{\lstlistingname}{Listing}

\begin{document}


\newgeometry{left=2cm,right=2cm}
\begin{figure}
\begin{centering}
\resizebox{\textwidth}{!}{
\def\d{0.6}  % vertical distance between nodes.
\def\g{2.25} % horizontal gap between the two functions.

\begin{tikzpicture}[]

\tikzstyle{state}=[circle, align=center,draw,very thin, inner sep = 0.5mm]
\tikzstyle{edge}=[->,>=latex, thin]
\tikzstyle{point}=[font=\footnotesize\itshape]
\tikzstyle{legend}=[align=left, anchor=west,font=\small]
\tikzstyle{dataflow}=[color=purple]
\tikzstyle{stmt}=[color=teal]
\tikzstyle{dom}=[color=magenta]
\tikzstyle{computefun}=[color=orange]

\coordinate (Init) at (-1,0);

\node[point] (Caller) at ($ (Init) + (0, 2.5 * \d) $ ) {caller};
\node[point] (Called) at ($ (Init) + (\g, 2.5 * \d) $ ) {called};

\node[state] (S00) at (Init) {.};
\node[state] (S0) at ($ (S00) - (0, 2 * \d) $) {.};
\node[state] (S1) at ($ (S0) + (\g, - 2 * \d) $) {.};
\node[state] (S11) at ($ (S1) - (0, 2 * \d) $) {.};
\node[state] (S20) at ($ (S11) - (0, 4 * \d) $) {.};
\node[state] (S2) at ($ (S20) - (0, 2 * \d) $) {.};
\node[state] (S30) at ($ (S2) - (0, 2 * \d) $) {.};
\node[state] (S3) at ($ (S30) - (0, 2 * \d) $) {.};
\node[state] (S4) at ($ (S3) - (0, 2 * \d) $) {.};
\node[state] (S50) at ($ (S4) - (\g, 2 * \d) $) {.};
\node[state] (S5) at ($ (S50) - (0, 2 * \d) $) {.};
\node[state] (S6) at ($ (S5) - (0, 2 * \d) $) {.};
\node[state] (S7) at ($ (S6) - (0, 2 * \d) $) {.};

\draw[edge, stmt] (S00) -- (S0);
\draw[edge, stmt] (S0) -- (S1);
\draw[edge, dataflow] (S1) -- (S11);
\draw[edge, dataflow] (S11) -- (S20);
\draw[edge, dataflow] (S20) -- (S2);
\draw[edge, dataflow] (S2) -- (S30);
\draw[edge, stmt] (S30) -- (S3);
\draw[edge, stmt] (S3) -- (S4);
\draw[edge, stmt] (S4) -- (S50);
\draw[edge, stmt] (S50) -- (S5);
\draw[edge, stmt] (S5) -- (S6);
\draw[edge, stmt] (S6) -- (S7);
\draw[edge, stmt] (S0) -- (S50);


\draw[edge] ($ (S00) + (0, \d) $) -- (S00);
\draw[edge] (S7) -- ($ (S7) - (0, \d) $);

\node[anchor=east, point] (P00) at ($ (S00) - (0.2, 0) $)
     {State before the call site};
\node[anchor=east, point] (P0) at ($ (S0) - (0.2, 0) $) {$ S_0 $};
\node[anchor=west, point] (P1) at ($ (S1) + (0.2, 0) $) {$ S_1 $};
\node[anchor=west, point] (P2) at ($ (S20) + (0.2, 0) $)
     {State before the return statement of f};
\node[anchor=west, point] (P4) at ($ (S4) + (0.2, 0) $) {$ S_n $};
\node[anchor=east, point] (P5) at ($ (S50) - (0.2, 0) $) {``$S_0 \sqcap S_n$''};
\node[anchor=east, point] (P7) at ($ (S7) - (0.2, 0) $)
     {State after the call site};

\node[legend] (L00) at ($ (S00) + (2.5, -\d)$)     {1.  evaluation of the arguments $\mathit{exprs}$};
\node[legend, dom] (L0) at ($ (S0) + (2.5, -\d)$)   {2.  $\mathtt{start\_call}(S_0) = \mathrm{Compute}(S_1)$};
\node[legend] (L1) at ($ (S1) + (1,- \d)$)         {3.  $\mathtt{enter\_scope}(f_{locals})$};
\node[legend] (L11) at ($ (S11) + (1, -2 * \d)$)   {4.  analysis of f};
\node[legend] (L20) at ($ (S20) + (1, -\d)$)       {5.  $\mathtt{enter\_scope}(RET)$};
\node[legend] (L2) at ($ (S2) + (1, -\d)$)         {6.  $\mathit{RET} = \mathit{return\_expr} ;$};
\node[legend] (L30) at ($ (S30) + (1, -\d)$)       {7.  evaluation of the formal arguments $\mathit{args}$};
\node[legend] (L3) at ($ (S3) + (1, -\d)$)         {8.  $\mathtt{leave\_scope}(f_{\mathit{formals}} \cup f_{\mathit{locals}})$};
\node[legend, dom] (L4) at ($ (S4) + (1, -\d)$)     {9.  $\mathtt{finalize\_call}(S_0,S_n)$};
\node[legend] (L50) at ($ (S50) + (2.5, -\d)$)     {10. reduction of the concrete arguments,\\ assuming $\mathit{exprs} = \mathit{args}$ at the end of $f$};
\node[legend] (L5) at ($ (S5) + (2.5, -\d)$)       {11. $v = \mathit{RET} ;$};
\node[legend] (L6) at ($ (S6) + (2.5, -\d)$)       {12. $\mathtt{leave\_scope}(RET)$};

\draw[<->, >=latex, very thin] (L30.east) to[bend left] (L50.east);


\draw[edge] (S0) to[bend right = 80]
node[legend, dom, midway, anchor=east]
{$\mathtt{start\_call}(S_0) = \mathrm{Result}(S_n)$ \\
$\mathtt{approximate\_call}(S_0) = S_n$}
(S4);

\draw[edge, computefun] (S1) to[bend right = 26]
node[legend, midway, sloped, anchor=north]{
$  \begin{array}{c}
\color{black}   \mathtt{mem\_exec} \\
\color{magenta} \mathtt{compute\_using\_spec}
\end{array}  $} (S30);

% Legend

\coordinate (N1) at ($ (Init) - (8, 0) $);
\coordinate (N2) at ($ (N1) - (0, 1 * \d) $);
\coordinate (N3) at ($ (N2) - (0, 0.5 * \d) $);
\coordinate (N4) at ($ (N3) - (0, 1 * \d) $);
\coordinate (N5) at ($ (N4) - (0, 0.5 * \d) $);
\coordinate (N6) at ($ (N5) - (0, 1 * \d) $);

\draw[edge, stmt] (N1) -- node[legend,point]{in transfer\_stmt} (N2);
\draw[edge, dataflow] (N3) -- node[legend,point]{in partitioned\_dataflow} (N4);
\draw[edge, computefun] (N5) -- node[legend,point]{in compute\_function} (N6);

\end{tikzpicture}
}
\par\end{centering}
\vspace{1cm}
\caption{Interpretation of the call\label{fig:call-interpretation}}
\end{figure}
\restoregeometry

\begin{lstlisting}[caption={Call to a function $f$},language=C,float,numbers=left,stepnumber=5]
int f (args) {
  [...]
  return return_expr;
}

int main () {
  ...
  v = f (exprs);
  ...
}

\end{lstlisting}


\section*{Interpretation of a Function Call within Eva}

\subsubsection*{Step by Step}

Figure~\ref{fig:call-interpretation} outlines each stage of the
interpretation of a function call. The magenta lines highlight the
functions provided by an abstract domain and dedicated to function
calls. The others computations are performed by the standard transfer
functions for the evaluation of expressions and interpretation of
statements. The teal edges refer to actions written in \texttt{transfer\_stmt.ml},
while the purple edges refer to actions written in \texttt{partitioned\_dataflow.ml}.
The left line gathers the computations carried out at the call site,
while the right line gathers the computations carried out in the body
of the called function.

The stages of the interpretation of a function call are as follows:
\begin{enumerate}
\item The concrete arguments $\mathit{exprs}$ are evaluated at the call
site.
\item $\mathtt{start\_call}$ builds the state at the entry point of function
$f$. This steps includes the entry in scope of the formal arguments
$f_{\mathit{formals}}=\mathit{args}$, and their instantiation with
the value of the concrete arguments (evaluated at step~1).
\item The local variables of $f$ enter in scope.
\item Standard dataflow analysis of $f$ up to the state inferred before
the return statement of $f$.
\item The special variable $\mathit{RET}$ enters in scope.
\item The special variable $\mathit{RET}$ is assigned to the return value,
through a standard assignment.
\item The formal arguments $\mathit{args}$ of the function are evaluated,
and the resulting values $v_{\mathit{args}}$ are stored.
\item The formal and local variables leave the scope.
\item $\mathtt{finalize\_call}$ merges the state at the call site and the
state at the end of the called function.
\item At the call site, the concrete arguments are reduced to the values
$v_{\mathit{args}}$, provided that $\mathit{args}=\mathit{exprs}$
at the end of the function~$f$. This condition holds if the values
of $\mathit{args}$ and $\mathit{exprs}$ have not been modified by
$f$.
\item The assignment of the return value of $f$, through a regular assignment
of the value of $\mathit{RET}$.
\item The special variable $\mathit{RET}$ leaves the scope.
\end{enumerate}


\end{document}
