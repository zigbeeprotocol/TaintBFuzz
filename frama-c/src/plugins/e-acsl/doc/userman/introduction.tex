\chapter{Introduction}

\framac~\cite{userman,fac15} is a modular analysis framework for the C
programming language which supports the ACSL specification
language~\cite{acsl}. This manual documents the \eacsl plug-in of \framac,
version \eacslpluginversion. The \eacsl version you are using is indicated by the
command \texttt{frama-c -e-acsl-version}\optionidx{-}{e-acsl-version}. \eacsl
automatically translates an annotated C program into another program that fails
at runtime if an annotation is violated. If no annotation is violated, the
behavior of the new program is exactly the same as the one of the original
program.

\eacsl translation brings several benefits. First, it allows a user to monitor
\C code and perform what is usually referred to as ``runtime assertion
checking''~\cite{runtime-assertion-checking}\footnote{In our context, ``runtime
  annotation checking'' would be more precise.}. This is the primary goal of
\eacsl. Indirectly, in combination with the \rte plug-in~\cite{rte} of \framac,
this
usage allows the user to detect undefined behaviors in its \C code. Second, it
allows to combine \framac and its existing analyzers with other \C analyzers
that do not natively understand the \acsl specification language. Third, the
possibility to detect invalid annotations during a concrete execution may be
very helpful while writing a correct specification of a given program,
\emph{e.g.} for later program proving.  Finally, an executable specification
makes it possible to check assertions that cannot be verified statically and
thus to establish a link between runtime monitoring and static analysis tools
such as \Eva~\cite{eva}\index{Eva} or \wpplugin~\cite{wp}\index{Wp}.

Annotations used by the plug-in must be written in the \eacsl specification
language~\cite{eacsl,sac13} -- a subset of \acsl. \eacsl plug-in is still in a
preliminary state: some parts of the \eacsl specification language are not yet
supported. Annotations supported by the plugin are described in a separate
document~\cite{eacsl-implem}. It is worth noting that the annotations that aim
to be dynamically verified are not necessarily hand-written, but may be
automatically generated instead. That is for instance the case when checking the
absence of undefined behaviors in combination with RTE, as mentionned in the
previous paragraph. Using \eacsl this way is therefore a fully automatic
process. Many usages, including automatic usages, are described in companion
research papers~\cite{rv13tutorial,rvcubes17tool,signoles18hdr}.

The \eacsl plug-in is installed with \framac, but this manual does \emph{not}
explain how to install \framac.  For installation instructions please refer to
the \texttt{INSTALL}\footnote{
  \url{https://git.frama-c.com/pub/frama-c/blob/master/INSTALL.md}}
file in the \framac distribution. \index{Installation} Furthermore, even though
this manual provides examples, it is \emph{not} a full comprehensive tutorial on
\framac or \eacsl.
% You can still refer to any external
% tutorial~\cite{rv13tutorial} for additional examples.
