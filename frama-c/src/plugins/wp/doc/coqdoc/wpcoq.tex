\documentclass[web]{frama-c-book} 
\usepackage{longtable}
\usepackage{pifont} 
\usepackage{coq2latex}

\newcommand{\fcaffiliationen}{CEA-List, Université Paris-Saclay\\ Software Safety and Security Lab\xspace}
\newcommand{\fcaffiliationfr}{CEA-List, Université Paris-Saclay\\ Laboratoire de Sûreté et Sécurité des Logiciels\xspace}


\begin{document}
\coverpage{WP Coq Handbook}
\begin{titlepage}
\includegraphics[height=14mm]{cealistlogo.jpg}
\hfill~
\vfill
\title{WP Coq Handbook}%
{Version 0.7 for Oxygen-20120901+dev}
\author{Patrick Baudin, Lo�c Correnson, Zaynah Dargaye}
\begin{center}
  \fcaffiliationen
\end{center}
\vfill
\begin{flushleft}
  \textcopyright 2010-2012 CEA LIST
\end{flushleft}
\end{titlepage}

\cleardoublepage
\markright{}
\tableofcontents

\chapter*{Introduction}

This document is a reference manual for using \textsf{Coq} proof assistant
with the \textsf{Frama-C} \textsf{WP} plugin.

\chapter{Core Logic}

These modules are related to the internal logic $\cal L$ language of
the \textsf{WP} plug-in, regardless any \textsf{C}-memory model. The
module \textsf{Qedlib} deals with boolean, integer tactics and arrays.
The module \textsf{Vset} deals with a sets and range of integers.

\input{Qedlib}
\input{Vset}

\chapter{C/ACSL Arithmetics}

The modules are related to the \textit{natural} model of integers and
floats introduced by the \textsf{WP} plug-in. Predefined \textsf{ACSL}
operators are also defined in the \textsf{Cmath} module. Bitwise
operators are only \textit{declared} in these modules, but their
properties are the subject of a dedicated chapter and modules.

\input{Cint}
\input{Cfloat}
\input{Cmath}

\chapter{Compounds Symbols}

This chapter introduce symbols that are generated on-the-fly by
\textsf{WP}, especially to deals with \textit{records} and
\textit{arrays} of \textsf{C} and \textsf{ACSL}.

\section{Generated Symbols}

The \textsf{WP} generates a new type \verb+S_R+ for each structure \verb+R+, and
new identifiers \verb+F_f+ for each field \verb+f+ of the compound.

For each record or array type \verb+T+, a predicate \verb+Is<T>+ is
introduced to constrain the types of each sub-component of compound
\verb+T+, typically the bound of integer fields.

The \textsf{WP} generates also an equality symbols for arrays of fixed
size of each dimensions and types.


\chapter{Bitwise Arithmetics}

This chapter entirely axiomatize the bitwise operators of
\textsf{ACSL} over arbitrary integer. The development is divided into
two parts: first, a general axiomatization of bits operations over
\textsf{Z}; second, the definition of \textsf{ACSL} operators in terms
of bits operations.

\input{Bits}
\input{Cbits}

\chapter{Typed Memory Model}

This chapter introduces the symbols introduced by the \textsf{Typed}
memory model. Some of them are statically defined in module
\textsf{Memory} presented in the chapter. But other symbols are
generated on-the-fly by the \textsf{WP} plug-in. They are introduce in
the \textsf{Generated Symbols} section.

\input{Memory}
%% Typed Memory Model
\section{Generated Symbols}

The \textsf{Typed} memory model generates on the fly some symbols
documented here.

For each \textsf{C} variable \texttt{x} that might be aliased, the
following symbols are generated:
\begin{itemize}
\item \verb+G_x_n+ is the base address of \verb+x+. \verb+G+ is replaced by
  \verb+P+ for formal parameters and \verb+L+ for locals. \verb+n+ is
  the \textsf{CIL} variable identifier.
\item \verb+G_x_n_linked+ is the predicate defining the allocation
  of the variable inside its scope.
\item \verb+G_x_n_region+ defines the region in which the variable is defined
  (heap, stack frame or local variables).
\end{itemize}

For each struct \verb+R+, the \textsf{Typed} memory model generates a
symbol \verb+Load_R+ that reads field-by-field the associated logic
record. For arrays of type \verb+t+, the \textsf{Typed} memory model
generates a symbol \verb+Array<d>_<t>+ where \verb+d+ represent the
dimensions of the array.

For both records and arrays, the \textsf{Typed} memory also generates
lemmas that asserts monotony properties of load symbols with respect
to updates and equalities between two memory states.



\end{document}

