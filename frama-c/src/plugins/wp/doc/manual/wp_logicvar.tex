\section{Logic variables}\label{sec-funvar}

As previously presented, there is two kinds of memory models.  One
based on variables, as in the Hoare model see section~\ref{sec-hoare}
and the other based on the representation of the heap, as the Store
model (see section~\ref{sec-store}) and the Runtime model (see
section~\ref{sec-runtime}).

The second kind of memory model is more expressive since it can handle
 dereferencing of pointers, and addresses of variables. 
The cost is quite heavy: the generated formula are largest.

The ideal is to benefit this expressiveness only for the variables
involved in dereferencing of pointers and address management, in
other words the variables such as their addresses are computed. For
the other variables, variables such as their addresses are never
computed (notice that they behave as functional or logical variables)
they can be handled by a model {\it � la} Hoare. This kind of
optimization, with a generous definition of optimization, is quite
current in memory model specifications and implementations.

In the WP, the model {\it � la} Hoare is the Logicvar model and
to make it cooperate with a based on heap memory model, one has to set
the {\tt -wp-logicvar}.

A variable translated in Logicvar has in $\cal{L}$ a logic type then
it is translated as \textsf{ACSL} data (see section~\ref{sec-acsl}).


Let us consider this example: 
\begin{ccode}
typedef struct S { int a; int * p; }; 

struct S s;
struct S r;


/*@ 
    ensures \result == {s \with .a = (int)4 };
*/
struct S ret_struct(void)
{
  struct S * ps; 
  r.a = s.a ; 
  s.a = 4;
  ps = &r;
  return r;
}
\end{ccode}

This example has a semantics only in a based on heap memory model,
because the address of \whyinline{r} is taken. Computing the 
proof obligation associated to the post-condition with the memory model 
Store in the WP plug-in gives:

\begin{whycode}

goal store_full_ret_struct_post_1:
  forall m:data farray.
  forall ta:int farray.
     global(ta)
  -> (ta[X_ps] = 0)
  -> is_fresh(m, ta, X_ps)
  -> is_fresh(
       m[addr(X_r,0)
         <-encode(int_format,decode(int_format,m[addr(X_s,0)]))][addr(X_s,
                                                                    0)
         <-encode(int_format,4)][addr(X_ps,0)
         <-encode(addr_format,addr(X_r,0))], ta_2[X_ps<-1][X_ps<-0], X_ps)
  -> Eqrec_S(
       decode(Cfmt_S,
         access_range(
           m[addr(X_r,0)
             <-encode(int_format,decode(int_format,m[addr(X_s,0)]))][
             addr(X_s,0)
             <-encode(int_format,4)][addr(X_ps,0)
             <-encode(addr_format,addr(X_r,0))],zrange(X_r,0,2))),
       decode(Cfmt_S,
         access_range(
           m[addr(X_r,0)
             <-encode(int_format,decode(int_format,m[addr(X_s,0)]))][
             addr(X_s,0)
             <-encode(int_format,4)][addr(X_ps,0)
             <-encode(addr_format,addr(X_r,0))],zrange(X_s,0,2)))[F_a
         <-encode(int_format,as_int(sint32_format,4))])
\end{whycode}

For each variable, a location is created into the heap.  The local
variables trigger two local allocations for their locations.  The three
assignments take place in the same $\cal{L}$ array, the store.
Unless expressing the address taken of \whyinline{r} and the address
assignment \whyinline{ps= &r;}, this is uninteresting to use the Store
model. Moreover, this assignment is heavy noise for the automatic
provers. Then, using Logicvar, the WP plug-in generates the formula:

\begin{whycode}
goal store_ret_struct_post_1:
  forall m:data farray.
  forall s:data farray.
  Eqrec_S(
    decode(Cfmt_S,
      access_range(
        m[addr(X_r,0)<-encode(int_format,decode(int_format,s[F_a]))],
        zrange(X_r,0,2))),
    s[F_a<-encode(int_format,4)][F_a
      <-encode(int_format,as_int(sint32_format,4))])
\end{whycode}

Since the assignment \whyinline{ps= &r;} is {\it mute} for the post-condition, 
the WP plug-in ignores it. Hence, only \whyinline{r} is represented into the 
memory, the other variables are handled as logic variables.


