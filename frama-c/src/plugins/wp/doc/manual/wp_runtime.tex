%% Chapter WP Models

\section{Runtime}\label{sec-runtime}

\subsection{Description and limitations}

This model is a very low level one, where the memory is just a big array of bits
indexed by addresses. The values, represented by bit vectors 
(\whyinline{bits}),
are stored in memory zones (\whyinline{rt_zone})
represented by a starting address, and a size (number of bits).
The  \whyinline{rt_load(m, z)} operation returns the bits stored in the zone
\whyinline{z}
of the memory \whyinline{m}, and \whyinline{rt_store (m, a, b)} returns a memory similar to
\whyinline{m} except that the bits \whyinline{b} have been stored at the address
\whyinline{a}.

Operations \whyinline{rt_to_bits} and \whyinline{rt_from_bits} specify
how to interpret the bit vectors into typed values.
In simple cases, we don't need to go into representation details,
but it can be necessary when the C program uses some casts
and other architecture dependent features
such as endianness for instance.

This model also deals with memory management of variables 
(see \whyinline{rt_valloc} and \whyinline{rt_vfree})
to be able to prove the validity of pointers,
but the dynamic allocations are not handled yet.

\subsection{Examples}

\subsubsection{Simple variable assignment}

Let's begin with the simple assignment example used in Hoare presentation.
To check this assertion after this statement:
\begin{ccode}
  x = y+1;
  //@ assert x > 0;
\end{ccode}
the proof obligation with the Runtime model is a lot more complex:
\begin{whycode}
  forall ma:memalloc.
  forall mb:memory.
  let mb=
    rt_store(mb,rt_vaddr(ma,X_x),
      rt_to_bits(sint32_format,
        (rt_from_bits(rt_load(mb,rt_vzone(ma,X_y)),sint32_format)+1))) in
  (0 < rt_from_bits(rt_load(mb,rt_vzone(ma,X_x)),sint32_format))
\end{whycode}
First of all, we see that the memory state is represented by two variables:
$ma$ is used for allocation information, and $mb$ for the values (bits).
The last line is the translated initial property which say that if we read
(\whyinline{
rt_load}) in $mb$ the bits stored in the zone where $x$ is allocated
(\whyinline{
rt_vzone(ma,X_x)}), and if we interpret (\whyinline{rt_from_bits})
those bits as a signed 32 bit integer (which is the type of $x$),
then we should get a positive value.
The other lines represent the bit memory update (\whyinline{rt_store(mb,...)})
due to the assignment.

Trying to complete the proof, the Runtime model requires us to prove
that \cinline{(y+1)} still fits in the \whyinline{sint32_format} which hasn't
been requested by the Hoare model because of its abstraction level.
This condition is implied by the RTE generated annotations
that should be check separately (see \verb!-rte! option).

This example might look a little frightening because of the size of the proof
obligation compared to the size of the code, and that is precisely why 
it is not advisable to use the Runtime model when its low level features are not
required. But we will see in \S\ref{sec-funvar} that in some simple cases,
such as this one, dramatic simplifications can be done.


\subsubsection{Simple variable allocation}

Runtime model defines the operation \whyinline{rt_valloc} to represent
a modification of the allocation information. So checking the assertion
\begin{ccode}
  int x;
  //@ assert \valid (&x);
\end{ccode}
gives the following the proof obligation:
\begin{whycode}
  forall ma:memalloc.
  let ma= rt_valloc(ma,X_x,32) in
  rt_valid(ma, rt_vzone(ma,X_x))
\end{whycode}
which is trivially true according to Runtime properties.

Conversely, the operation \whyinline{rt_vfree} is used to represent
the deallocation. It makes possible to check this kind of property:
\begin{ccode}
  { int x; p = &x; }
  //@ assert ! \valid (p);
  \end{ccode}
That gives the following proof obligation:
\begin{whycode}
  forall ma:memalloc.
  forall mb:memory.
  let ma1 = rt_valloc(ma,X_x,32) in
  let mb= rt_store(mb,rt_vaddr(ma1,X_p),
             rt_to_bits(rt_addr_format,rt_vaddr(ma1,X_x_0_1))) in
  let ma2= rt_vfree(ma1,X_x_0_1) in
  (not rt_valid(ma2,
      rt_zone(rt_from_bits(rt_load(mb,rt_vzone(ma2,X_p)),rt_addr_format),32)))
\end{whycode}
but could be simplified, using Runtime rules, to:
\begin{whycode}
  forall ma:memalloc.
  let ma1 = rt_valloc(ma0,X_x,32) in
  let ma2= rt_vfree(ma1,X_x) in
  not rt_valid (ma2, rt_zone(rt_vaddr(ma1,X_x), 32))
\end{whycode}
which is true because, of course, a zone is not valid after having been freed.
