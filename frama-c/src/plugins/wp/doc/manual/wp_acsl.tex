%% Chapter WP Models

\section{ACSL} \label{sec-acsl}

This section describes, firstly, the model of \textsf{ACSL} datum (or
logical values) such as reals and integers, the logical arrays and
records, the logical pointers and sets. Secondly, the translation
of the predicates and axiomatics of \textsf{ACSL} to $\cal{L}$ is 
presented.


\subsection{ACSL values}

\subsubsection{Values of Basic types}

 The basic types are the C arithmetic types and the logic integers and reals. 
 
 All integer types are translated into the type $\mathbb{Z}$ of $\cal{L}$.
In the same flavor, the object of the logic type \cinline{real} of 
 \textsf{ACSL} becomes object of type $\mathbb{R}$ of $\cal{L}$.

 That is not sufficient for the translation of numerical values:
 they have to be bound in a convenient domain. 
 To express that domain constraint, the notion of {\it format} is necessary.  
 A \whyinline{format} informs about the \textsf{ACSL} type of the
 value. The table in the figure~\ref{fig:type_format} gathers some
 primitive formats.

 The format associated to a C arithmetic type handles the signess and
 size information of the type. The predicate
 \whyinline{is_in_format} expresses the domain constraint.

 For example: \cinline{unsigned char i;} becomes in $\cal{L}$ 
 \whyinline{forall i:Z. is_in_format(uint8_format,i)} which provide the
 information that $0 \le i < 256$.

\TODO{Pointers and Formats have been completely modified}

 \begin{figure}[ht!]
   \begin{center}
     
     \begin{tabular}{|l|r|r|}
       \hline 
       \textsf{ACSL}: $\tau$ & $\cal{L}$: $\overline{\tau}$ & $\cal{L}$ format \\
       \hline
       \cinline{signed char/short/int/long int} & 
       $\mathbb{Z}$ & \whyinline{sint8/16/32/64_format} \\
       \hline
       \cinline{unsigned char/short/int/long int} & 
       $\mathbb{Z}$ & \whyinline{uint8/16/32/64_format} \\
       \hline
       \cinline{float,double} &$\mathbb{R}$ & 
       \whyinline{float16/32/64/96/128_format} \\
       \hline
       integer &  $\mathbb{Z}$  &  \whyinline{int_format} \\
       \hline 
       real & $\mathbb{R}$ & \whyinline{real_format} \\
       \hline 
       $\tau~*$ & pointer  & \whyinline{pointer_format} \\
       \hline 
       $\tau~$ array & $\overline{\tau}$ farray &
       $\overline{\tau}$ \whyinline{array_dim_format} \\
       \hline 
       struct S $\{ \ldots\tau_n~f_n; \ldots \}$ &
       S  & \whyinline{S_format} \\
       \hline 
     \end{tabular}

\end{center}
\caption{Typed representation of objects}
\label{fig:type_format}
\end{figure} 


\subsubsection{Logic pointers}\label{ssec-acsl-ptr}

  The logic pointers of \textsf{ACSL} are translated into an opaque
  type \whyinline{pointer} which is untyped. 

  All memory models use the logic arithmetic of pointers defined in
  figure~\ref{fig:arithptr}, except for the shift operations which are 
  memory model dependent.


\begin{figure}[ht!]
\begin{center}

  \begin{tabular}{|l|r|r|}
    \hline
    ACSL & $\cal{L}$ functional & $\cal{L}$ relational \\
    \hline 
    \cinline{NULL} & \whyinline{null} & \whyinline{is_null} \\
    \cinline{p == q}& \whyinline{equal_pointer(p,q)} & \whyinline{p=q} \\
    \hline 
    \cinline{p < q} & \whyinline{lt_pointer(p,q)} & 
    \whyinline{lt_pointer_rel(p,q)} \\
    \hline 
    \cinline{p <= q} & \whyinline{le_pointer(p,q)} & 
    \whyinline{le_pointer_rel(p,q)} \\
   \hline 
    \cinline{p - q} & \whyinline{minus_pointer(p,q)} & $\emptyset$ \\
   \hline
   \cinline{p + i} & \multicolumn{2}{c|}{memory model dependent} \\
\hline
  \end{tabular}

\end{center}
\caption{Translation of the arithmetic of pointers}
\label{fig:arithptr}
\end{figure}
  

 As all the pointer arithmetic is done via the logic pointer, 
the  memory model has to provide coercion between a logic pointer
 and its location type. 
   

\subsubsection{Arrays}

   An \textsf{ACSL} array of type $\tau$ is represented by an infinite
   array of type $\overline{\tau}$ in $\cal{L}$
   (see figure~\ref{fig:type_format} for interpretation of primitive types). 


   

   The classical operations over \textsf{ACSL} array are translated as 
 suspected: 
 \begin{description}
   \item{access} to a value at the index \cinline{i} of an \textsf{ACSL} array
     \cinline{t} :
     \cinline{t[i]} becomes \whyinline{access(t,i)} in $\cal{L}$,
     which also can be written as \whyinline{t[i]};
 
   \item{functional update} of the value indexed by \cinline{i} of an
     \textsf{ACSL} array \cinline{t} by the new value \cinline{v}:
     \cinline{\{t with [i] = v\}} 
     becomes \whyinline{update (t, i, x)} in $\cal{L}$, which also can
     be written \whyinline{t[i <- x ]}.
 \end{description}

   This is possible to access or update a set of indexes in a logic array:
   \begin{description}
     \item{\whyinline{get_range_index(t,s)}}
       returns the set of values of \whyinline{t} of the indexes of
       the set \whyinline{s}, when \whyinline{t} is an array;

       \item{\whyinline{set_range_index(t,s)}} is an array such as all
         elements \whyinline{set_range_index(t,s)[i]}= \whyinline{t[i]} 
         when \whyinline{i} $\notin$ \whyinline{s} holds.
   \end{description}



\subsubsection{Logic records and unions }\label{ssec-acsl-rec} 
 
 \begin{description}
   \item {\cinline{struct S {...}}} becomes \whyinline{type S}.
   \item{\cinline{union U{...}}} becomes \whyinline{type U}.
   \item{For each field \cinline{tf f;} of a structure or an union S}
     two functions are generated in $\cal{L}$:
     \begin{itemize}
       \item \whyinline{get_field_f_S:S ->}$\overline{tf}$ returns the value 
         of the field \cinline{f} of a record or a union of type \cinline{S} 
         \item \whyinline{set_field_f_S:S,}$\overline{tf}$\whyinline{
           -> S} returns the updated record or union at the field
           \cinline{f}
     \end{itemize}
     and an axiom : 
     \begin{itemize}
       \item\whyinline{axiom GetSetSame_f_S : forall r:S. forall
         v:}$\overline{tf}$\whyinline{.}
         \whyinline{get_field_f_S(set_field_f_S(r,v)) = r}
     \end{itemize}
    \item{For each disjoint pair of fields \cinline{...; tg g;...;tf
        f;...} of a same structure definition S} an axiom is defined as follows:
      \begin{itemize}
      \item\whyinline{axiom GetSetOther_f_g_S : forall r:S. forall
        v:}$\overline{tg}$\whyinline{.}
        \whyinline{get_field_f_S(set_field_g_S(r,v)) = get_field_f_S(r)}
      \end{itemize} 
    \item{\cinline{s.f}} becomes \whyinline{get_field_f_S(s)}. 
    \item{\cinline{s with .f = v}} becomes \whyinline{set_field_f_S(s,v)}. 
 \end{description}

 Consider the following example: %(tests/wp_acsl/record.c): 
 \begin{ccode}
   struct S {int a; int b;};

   struct S s1,s2; 
   
   /*@
   ensures {s1 \with .a = (int)1}.a == 1;  
   */
   void f (void) {}
 \end{ccode}

 The WP computation will give you this $\cal{L}$ specification:
\begin{whycode}
type S
logic get_a_S: S -> int 
logic set_a_S: S , int -> S 
logic get_b_S: S -> int 
logic set_b_S: S , int -> S 
(* Definition of the good properties of the field a*)

axiom GetSetSame_a:
forall r:S.forall v:int.
get_a_S(set_a_S(r,v))=
 v


(* Definition of the commutativity of the get field a over the set field b*)

axiom GetSetOther_a_b:
forall r:S.forall v:int.
get_a_S(set_b_S(r,v))=
 get_a_S(r)



(* Definition of the good properties of the field b*)

axiom GetSetSame_b:
forall r:S.forall v:int.
get_b_S(set_b_S(r,v))=
 v


(* Definition of the commutativity of the get field b over the set field a*)

axiom GetSetOther_b_a:
forall r:S.forall v:int.
get_b_S(set_a_S(r,v))=
 get_b_S(r)
\end{whycode}

 

 \subsubsection{The equality of logical values, \whyinline{v == v'}}\label{ssec-acsl-eq}
 
 Alike for the representation of logic values, the relations of 
 equality of logic values are directed by types. Hence, 
 the equality of logic values of basic type; integer, reals and pointers,
 is the equality of $\cal{L}$, \whyinline{=}. 
 
 Concerning the equality of objects of more complex type; arrays, records 
 and union, the \textsf{ACSL} model generates a equality 
 relation for each type. \par
 
 Two records of same type \cinline{S} are equals if their fields are equal. 
 
 Consider the following example:
 \begin{ccode}
   typedef struct T {float f;};
   
   typedef struct S { int a; int * p; struct T t;};
 \end{ccode}
 
 If the WP computation needs to deal with object of type
 \cinline{S},then an axiom in the environment over equality for object
 of type \cinline{S} is defined \whyinline{Eqrec_S}:
 
 \begin{whycode}
Equality for Struct 'S'
tests/wp_acsl/record.c:20: typedef struct S { ... }
logic Eqrec_S: S,S -> prop

Equality for Struct 'T'
tests/wp_acsl/record.c:18: typedef struct T { ... }
logic Eqrec_T: T,T -> prop

Equality for Struct 'S'
tests/wp_acsl/record.c:20: Axiomatic Definition
axiom EqrecDef_S:
  forall a:S.
  forall b:S.
      Eqrec_S(a, b)
  <-> (    (get_a_S(a) = get_a_S(b))
       and (get_p_S(a) = get_p_S(b))
       and Eqrec_T(get_t_S(a), get_t_S(b)))

Equality for Struct 'T'
tests/wp_acsl/record.c:18: Axiomatic Definition
axiom EqrecDef_T:
  forall a:T.
  forall b:T.
      Eqrec_T(a, b)
  <-> (get_f_T(a) = get_f_T(b))

Symmetry of Equality for Struct 'S'
tests/wp_acsl/record.c:20: Axiomatic Definition
axiom EqrecSym_S: forall a:S.
  Eqrec_S(a, a)

Symmetry of Equality for Struct 'T'
tests/wp_acsl/record.c:18: Axiomatic Definition
axiom EqrecSym_T: forall a:T.
  Eqrec_T(a, a)

Symmetry of Equality for Struct 'S'
tests/wp_acsl/record.c:20: Axiomatic Definition
axiom EqrecTrans_S:
  forall a:S.
  forall b:S.
  forall c:S.
     Eqrec_S(a, b)
  -> Eqrec_S(b, c)
  -> Eqrec_S(a, c)

Symmetry of Equality for Struct 'T'
tests/wp_acsl/record.c:18: Axiomatic Definition
axiom EqrecTrans_T:
  forall a:T.
  forall b:T.
  forall c:T.
     Eqrec_T(a, b)
  -> Eqrec_T(b, c)
  -> Eqrec_T(a, c)


   \end{whycode}

   As an object of type \cinline{S} has a field of type \cinline{T},
   the equality over objects of type \cinline{T} is also defined.  Then
   two objects of type \cinline{S} are equals iff the values their
   field are equal two by two. \par

   The same mechanism defines the equality on objects of an union type. \par


   Two arrays of the same type $\tau$ and the same length are equals iff 
   their elements are equals one by one according to the relation 
   of equality of objects of type $\tau$. 
  
   Consider: 
   \begin{ccode}
     typedef struct S { ...}; 
     struct S tab[10];
   \end{ccode}
   

   The generated equality relation of object of type \cinline{struct S tab[10]}
   is : 
   \begin{whycode}
Equality for S[10]
Logic array
logic Eqarr_S_10: S farray,S farray -> prop

Equality for S[10]
Axiomatic Definition
axiom EqarrDef_S_10:
  forall a:S farray.
  forall b:S farray.
      Eqarr_S_10(a, b)
  <-> (forall i:int. 0 <= i -> i < 10 -> Eqrec_S(a[i], b[i]))


   \end{whycode}

 \subsubsection{Set}

  The logic \textsf{ACSL} $\tau$ sets are specified into
  $\overline{\tau}$ \whyinline{set} with all traditional constructors,
  functions and properties.


 \subsubsection{Logic cast}

 Logic functions \whyinline{as_int} and  \whyinline{as_float}
 are used to represent the type conversions between 
 basic arithmetic types that map to the same type of  $\cal{L}$.
 They also use the {\it format} presented in \ref{fig:type_format}.
 At the moment, these functions are underspecified.
 For \whyinline{as_int} for instance, we only assume that :
\begin{whycode}
forall f: int format. forall x:int. is_in_format (f, x) -> as_int (f, x) = x
forall f: int format. forall x:int. is_in_format (f, (as_int (f, x)))
\end{whycode}

Two other logic functions (\whyinline{truncate_real_to_int} and
\whyinline{real_of_int}) are used to transform $\mathbb{R}$ to and from $\mathbb{Z}$.

Other casts involving pointers are memory model dependent.

\subsection{\textsf{ACSL} predicates and axiomatics} 

\TODO{To be complete}
