%% Chapter WP Models

\section{Hoare}\label{sec-hoare}

\subsection{Description and limitations}

The Hoare model is the traditional one where
the memory state is a mapping between the C program 
variables and their logical values. 
It has been extended a little to be able to deal which structures, arrays
and even with pointers as long as they are not used to make an indirect 
access to
the memory because Hoare model doesn't have any idea of what is an address
in the memory. It means that the C dereferencing (*) operation is not handled 
neither in the program, nor in the properties.

Another limitation about Hoare model is that it can provide only a limited
support about pointer validity (\whyinline{valid} ACSL predicate).
This is because memory allocation is not represented at this abstract level.

\subsection{Operations}

The Hoare model mainly uses logical operations from ACSL. 
It only has to define few more
operations on pointers because even if it doesn't deal with indirect access,
it can still work on address computations~:

\begin{description}
  \item \whyinline{m0_addr (X_x)} : represent the value of the address of the
    variable x;
  \item \whyinline{shift_field (p,f)} : 
    represent \cinline{\&(p->f)} where p
    holds the address of a structure;
  \item \whyinline{shift_ufield (p,f)} :  
    represent \cinline{\&(p->f)} where p
    holds the address of a union;
  \item \whyinline{shift_index (p,i)} : represent \cinline{\&((*p)[i])}
    where p holds the address of an array;
  \item \whyinline{shift_pointer (p,i)} : represent \cinline{(p+i)} 
    where p
    can have any type. Notice that this is different from the previous
    operation, because the result of this one has the same type than p.
\end{description}

\subsection{Examples}

Let's take some elementary transformation examples to better understand to
generated proof obligations.

\subsubsection{Simple variable assignment}

To check this assertion after the assignment:
\begin{ccode}
  x = y+1;
  //@ assert x > 0;
\end{ccode}
the proof obligation to satisfy before the statement would be :
\begin{whycode}
  let x= (y+1) in (0 < x))
\end{whycode}
So this first example shows that for Hoare model, 
the WP of a variable assignment is a variable substitution. 

\subsubsection{Pointer assignment}

In Hoare model, a pointer is not different from a simple variable,
but computation on pointers uses the model specific operations defined before.
For instance, to check:
\begin{ccode}
  p++;
  //@ assert p == &(t[3]);
\end{ccode}
the proof obligation is: 
\begin{whycode}
let p= shift_pointer(p,1) in (p = shift_index(m0_addr(X_t),3))
\end{whycode}
which should be equivalent to:
\begin{whycode}
p = shift_index(m0_addr(X_t),2)
\end{whycode}

% \TODO: check that we have the rules in hoare.why...

\subsubsection{Structure field assignment}

To check:
\begin{ccode}
  s.a = x;
  //@ assert s.a > 0 && s.b > 0;
\end{ccode}
the proof obligation is:
\begin{whycode}
  let s= s[F_a->{x}] in (0 < {s[F_a]}) and (0 < {s[F_b]})
\end{whycode}
where we see the structure update for the field $a$.
When applying the access/update rules, this can be simplified to :
\begin{whycode}
  (0 < x) and (0 < {s[F_b]})
\end{whycode}

\subsubsection{Array element assignment}

To check:
\begin{ccode}
  t[0] = x;
  //@ assert t[i] > 0;
\end{ccode}
the proof obligation is:
\begin{whycode}
  let t= t[0->x] in (0 < t[i])
\end{whycode}
meaning that we want $0 < t[i]$ after an update of $t$. Because of access/update
rules, this is equivalent to:
\begin{whycode}
  if (i == 0) then (0 < x) else (0 < t[i])
\end{whycode}

