\chapter{WP Simplifier}
\label{wp-simplifier}

The logical language $\cal{L}$ used to build proof obligations is now
equipped with built-in simplifications. This allows for proof
obligations to be simplified \emph{before} being sent to external
provers, and sometimes to be reduced to trivial goals.

This chapter is dedicated to the description of the simplifier and how to
use it with the \textsf{WP} plug-in. It also presents how combinatorial
explosion of path exploration is now tackled down thanks to
\emph{passive form} transformation and automated sub-terms
\emph{factorization}~\cite{FlanaganSaxe2001,Leino2003}. This also
leads to more compact and (somehow) more readable proof obligations,
with less memory, less disk usage and lower external prover time
overhead, compared to \textsf{WP} versions \verb+0.6+ and lower.

\section{Logic Normalizations}

The new logic language $\cal{L}$ is naturally equipped with term
normalization and maximal sub-term sharing. It is only used with new
memory models, not with the standard ones.

Maximal sub-term sharing is responsible for the introduction of
let-bindings whenever a sub-expression appears several times in the
generated proof obligations. The occupied memory and disk usage of WP
is also reduced compared to other models.

The normalization rules can not be turned off, and are responsible for
local simplifications. Although modest, they allow a proof
obligation to be trivially discharged.

\begin{description}
\item[Logic] normalization by commutativity and associativity;
  absorption and neutral elements; elimination of redundant facts;
  propagation of negations (Morgan laws); simplification of
  conditionals.
\item[Arithmetic] normalization by commutativity and associativity;
  absorption and neutral elements; factorization with linear forms;
  constant folding; normalization of linear equalities and
  inequalities.
\item[Array] elimination of consecutive accesses and updates.
\item[Record] elimination of consecutive accesses and updates;\\
  simplification of structural equalities and inequalities.
\end{description}

\section{Simplifier Engine (Qed)}

Built on top of our normalizing logic language $\cal{L}$, we have a
simplifier engine named \textsf{Qed}\cite{qed}. The simplifier engine is used by
the \textsf{WP} plug-in to simplify the generated proof contexts and proof
obligations. The basic feature of \textsf{Qed} is to manage a knowledge
base $\Gamma$. It is possible to add new facts (hypotheses) to
$\Gamma$, and to simplify (rewrite) a term of a property with respect
to $\Gamma$.

By default, the only rewriting performed by \textsf{Qed} is the
propagation of equality classes by normalization. The \textsf{Qed}
engine can be enriched by means of plug-ins to perform more dedicated
simplifications. For instance, we have developed a simplifier plug-in
for array and record theories, and a prototype for linear inequalities.

\textsf{WP} uses the simplification engine to simplify proof contexts
by recursively combining basic laws involving the simplifier
engine. Each law is applied with respect to a local knowledge base
$\Gamma$ (initially empty).

\newcommand{\QED}[3]{#1\models\,#2\,\triangleright\,#3}
\newcommand{\LAW}[2]{\begin{array}{c}{#1}\\\hline{#2}\end{array}}
\newcommand{\AND}{\quad\quad\quad}

Adding a new fact $H$ to $\Gamma$ is denoted by $\Gamma\oplus H$ ;
rewriting a term of predicate $e$ into $e'$ with respect to $\Gamma$
is denoted by $\Gamma\models e\triangleright e'$.

\paragraph{Inference Law.} 
A hypothesis is simplified and added to the knowledge base to
simplify the goal.
\[\LAW%
 {\QED{\Gamma}{H}{H'}\AND\QED{\Gamma\oplus H'}{G}{G'}}%
 {\QED{\Gamma}{(H\rightarrow G)}{(H'\rightarrow G')}}
\]

\paragraph{Conjunction Law.}
Each side of a conjunction is simplified with the added knowledge of
the other side. This law scales up to the conjunction of $n$ facts,
and simplifications can be performed incrementally.
\[\LAW%
 {\QED{\Gamma\oplus B}{A}{A'}\AND\QED{\Gamma\oplus A}{B}{B'}}%
 {\QED{\Gamma}{(A\wedge B)}{(A'\wedge B')}}
\]

\paragraph{Conditional Law.}
The conditional expression is simplified, before simplifying each
branch under the appropriate hypothesis.
\[\LAW
{\QED{\Gamma}{H}{H'}\AND
 \QED{\Gamma\oplus H'}{A}{A'}\AND
 \QED{\Gamma\oplus \neg H'}{B}{B'}}
{\QED{\Gamma}{(H\,?A:B)}{(H'\,?A':B')}}
\]

Inside the \textsf{WP} plug-in, the proof contexts are only built in
terms of conjunctions, conditional and inference rules. Hence, these
laws are sufficient to perform proof context simplifications.
Technically, simplification has a quadratic complexity in the width
and depth of the proof formula. Options will be added to control the
risk of combinatorial explosion. In practice, simplification is
delayed until submission of the proof obligation to external provers,
that have similar complexity. Since we account on simplification for
enhancing prover efficiency, we expect it to be worth the cost.

The power of the simplification process depends on the simplification plug-ins
loaded in the \textsf{Qed} engine, and will be the purpose of further
developments. Chapter~\ref{wp-builtins} provides additional informations on the
implemented simplifications and the supported \textsf{ACSL} built-in operations.

\section{Efficient WP Computation}

During the \emph{Weakest Precondition} calculus, proof obligations are
constructed backwardly for each program instruction. Conditional
statements are of particular interest, since they introduce a fork in
the generated proof contexts.

More precisely, consider a conditional statement \texttt{if $(e)$ $A$
  else $B$}. Let $W_A$ be the weakest precondition calculus from block
$A$, and $W_B$ the one from block $B$. Provided the translation of expression
$e$ in the current memory model leads to assumption $E$, the naive
weakest precondition of the conditional is: $(E \, ? W_A : W_B)$.

With this formula, the \emph{weakest preconditions} of the program
after the conditional are duplicated inside $W_A$ and $W_B$. Moreover, these
common postconditions have been transformed by the effects of $A$ and
$B$.  Then, the factorization of common sub-terms of logic language
$\cal{L}$ is \emph{not} capable of avoiding the duplication. In the presence of
successive conditionals, proof obligations generated become twice as
big at each conditional statement.

To tackle this problem, the solution is to put the program in
\emph{passive form}~\cite{FlanaganSaxe2001,Leino2003}. Each variable
of the program is assigned a different logic variable in each
branch. The different variables are joined at conditionals into new
fresh variables and equality conditions.

In practice, the passive form transformation is done during the
\emph{weakest precondition} calculus, together with the translation of
\textsf{C} and \textsf{ACSL} by the memory model. Hence, a translation
map $\sigma$ is maintained at each program point from memory model
variables to $\cal{L}$ logic variables.

Joining maps $\sigma_1$ and $\sigma_2$ from the branches of a
conditional leads to a new map $\sigma$ assigning a new logic variable
$x$ to memory variable $m$ whenever $\sigma_1(m)$ and $\sigma_2(m)$
are different. This join also produces the two sets of equalities $H_1$
and $H_2$ associated to this variable renaming. Hence $\sigma(m)=\sigma_1(m)$ below
is member of $H_1$ and $\sigma(m)=\sigma_2(m)$ is member of $H_2$.

Now, if $W$ is the postcondition of the conditional program below,
$W_A$ and $W_B$ can always be decomposed into: $W_A=W^0_A\wedge W$ and
$W_B=W^0_B\wedge W$. Finally, the weakest precondition of the
conditional is:
\[ (\,E \, ?\; H_1\wedge W^0_A \;:\; H_2\wedge W^0_B\;) \wedge W \]

This form actually factorizes the common postcondition to $A$ and $B$,
which makes the \emph{weakest precondition} calculus linear in the
number of program statements.

\section{Structure of JSON Scripts}

The proof scripts generated by using the Interactive Proof Editor (Section~\ref{wp-proof-editor})
are saved in the \textsf{Frama-C/WP} session as \textsf{JSON} files. This section defines
the format of session scripts precisely.

\textsf{JSON} scripts files are located in the \texttt{<session>/wp/scripts} sub-directory
of the \textsf{Frama-C} session directory (set with command line option \texttt{-session})
or in the \texttt{<wpsession>/scripts} sub-directory of the \textsf{Frama-C/WP} session
directory (set with command line option \texttt{-wp-session}).

The proof scripts for goal named \texttt{<goal>} is stored in a single file \texttt{<goal>.json} ; the
format of the \texttt{<goal>} name is identical to the one used by the \textsf{Frama-C/Report} plug-in
for its \textsf{JSON} output.

The content of each script file is an array ofs alternatives, each alternative being a record with
two possible structures, that might represent a \textit{prover} attempt or the application of a
\textit{tactic}:

\begin{align*}
    \mathit{wp.script} &::= \mathtt{[} \mathit{wp.alternative} , \ldots \mathtt{]} \\
    \mathit{wp.alternative} &::= \mathit{wp.prover} \;|\; \mathit{wp.tactic}
\end{align*}

\paragraph{Prover Attempts} The proof script records previous attempts to discharge a proof obligation
with external provers (also called decision procedures).
Each attempt is represented by a \textsf{JSON} record with the following fields:

\begin{align*}
    \mathit{wp.prover} &::=
    \left\{
    \begin{array}{rl}
        \mathtt{"prover"}  :& \mathit{string}, \\
        \mathtt{"verdict"} :& \mathit{wp.verdict}, \\ 
        \mathtt{"time"}    :& \mathit{number}? \\ 
        \mathtt{"steps"}   :& \mathit{number}? \\ 
        \mathtt{"depth"}   :& \mathit{number}? \\ 
    \end{array}
    \right\} \\
    \mathit{wp.verdict} &::=
    \begin{array}[t]{l}
    \mathtt{"none"} ~|~ 
    \mathtt{"valid"} ~|~ 
    \mathtt{"unknown"} ~|~ \\
    \mathtt{"timeout"} ~|~ 
    \mathtt{"stepout"} ~|~ 
    \mathtt{"invalid"} ~|~
    \mathtt{"failed"}
    \end{array}
\end{align*}

Remark that, since the proof scripts do not record the proof obligation that was exercised, there is no
guarantee that replying the script on a (potentially new) proof obligation would issue the same result.
Hence, prover attempts \emph{must} not be used as a cache, but can serve as a hint for choosing among
several alternatives.

\paragraph{Proof Tactic} Applying a tactic is represented by a \textsf{JSON} record with the following
fields:

\begin{align*}
    \mathit{wp.tactic} &::=
    \left\{
    \begin{array}{rl}
        \mathtt{"tactic"}   :& \mathit{string}, \\
        \mathtt{"header"}   :& \mathit{string} ? \\ 
        \mathtt{"select"}   :& \mathit{wp.selection}, \\ 
        \mathtt{"params"}   :& \mathtt{\{}\; \mathit{string}:\mathit{wp.param} ,\ldots \;\mathtt{\}} ? \\ 
        \mathtt{"children"} :& \mathtt{\{}\; \mathit{string}:\mathit{wp.script}, \ldots \;\mathtt{\}} ? \\ 
    \end{array}
    \right\}
\end{align*}

The \verb"header" field is just a descriptive title and has no special meaning. The \verb"tactic"
identifies the tactic, as it has been registered in the \textsf{Frama-C/WP} internal API.
The \verb"selection" is an opaque encoding of the target of the tactic inside the proof obligation
structure. The \verb"params" array stores the value of tactic parameters, if any.
The \verb"children" stores the proof scripts associated with each sub-goal generated by applying
the specified tactic on a proof obligation.

\paragraph{Tactic Parameters} Values of tactical parameters are encoded with the
following \textsf{JSON} format, depending on the internal type of the parameter 
value:

\begin{align*}
    \mathit{wp.param} ::=\;&
    \mathit{bool} ~|~ \mathit{number} ~|~ \mathit{string} \\
    |\;&\mathit{wp.named} ~|~ \mathit{wp.selection} \\
    \mathit{wp.named} ::= &\mathtt{null} ~|~
    \left\{
    \begin{array}{rl}
        \mathtt{"id"} :& \mathit{string}, \\
        \mathtt{"title"} :& \mathit{string}? \\
        \mathtt{"description"} :& \mathit{string}?
    \end{array}
    \right\}
\end{align*}

Named items (\textit{wp.named}) corresponds, for instance, to searched lemmas.
Selections corresponds to terms or expressions selected by the user \emph{via}
the graphical user interface, like the target of the tactic, as described below.

\paragraph{Tactic Selection} The target of the tactic
is identified by a complex structure
encoding on which part of the proof obligation it shall be applied. Since the
exact structure of the proof obligation may vary from time to time, this structure
allows for searching in the proof obligation a \emph{pattern} that ressemble the
original target that was originally selected by the user during an interactive
session with the \textsf{Frama-C/WP} graphical user interface.

Such \emph{patterns} are encoded as follows:

\newcommand{\dash}{\rule[0.5ex]{1ex}{1pt}}

\begin{align*}
    \mathit{wp.selection}
    ::=\;& \mathtt{null} \\
    |\;\;& \mathtt{\{}\; \mathtt{"select":"clause\dash{}goal"},\;
                         \mathit{wp.pattern} \;\mathtt{\}} \\
    |\;\;& \mathtt{\{}\; \mathtt{"select":"inside\dash{}goal"},\;
                         \mathit{wp.occur},\; \mathit{wp.pattern} \;\mathtt{\}} \\
    |\;\;& \mathtt{\{}\; \mathtt{"select":"clause\dash{}step"},\;
                         \mathit{wp.at},\; \mathit{wp.kind},\;
                         \mathit{wp.pattern} \;\mathtt{\}} \\
    |\;\;& \mathtt{\{}\; \mathtt{"select":"inside\dash{}step"},\;
                         \mathit{wp.at},\; \mathit{wp.kind},\;
                         \mathit{wp.occur},\; \mathit{wp.pattern} \;\mathtt{\}} \\
    |\;\;& \mathtt{\{}\; \mathtt{"select":"range"},\;
                         \mathtt{"min"}:\mathit{number},\;
                         \mathtt{"max"}:\mathit{number} \;\mathtt{\}} \\                         
    |\;\;& \mathtt{\{}\; \mathtt{"select":"kint"},\;
                         \mathtt{"val"}:\mathit{number} \;\mathtt{\}} \\
    |\;\;& \mathtt{\{}\; \mathtt{"select":"compose"},\;
                         \mathtt{"id"}:\mathit{string},\;
                         \mathtt{"args"}:[\mathit{wp.selection},\ldots]
                         \;\mathtt{\}} \\
% --
    \mathit{wp.pattern} ::=\;& \mathtt{"pattern"} : \mathit{string} \\
    \mathit{wp.occur}   ::=\;& \mathtt{"occur"} : \mathit{number} \\
    \mathit{wp.at}      ::=\;& \mathtt{"at"} : \mathit{number} \\
    \mathit{wp.kind}    ::=\;& \mathtt{"kind"} : \mathtt{"have"|"type"|"init"|"branch"|"either"|"state"} \\
\end{align*}

The various components of selection patterns have the following meaning: \textit{wp.pattern} encodes
a term or predicate pattern; \textit{wp.occur} designates which occurrence of the pattern to select,
in case of there is many of them inside the selected goal or hypothesis; \textit{wp.at} designates the
order of the selected hypothesis inside the proof obligation (the pattern is searched around this value),
and \textit{wp.kind} its kind.

The \textit{wp.pattern} is an simple string that encodes the head of the structure of the 
selected term or formulæ inside the designated goal or hypothesis. It is the concatenation of
the first 32 head nodes of width-first traversal of the selected term, each node being represented by
the following string:
\begin{itemize}
    \item constant are represented by their value,
    \item free variables by their base-name prefixed with \verb'$',
    \item bound variables by their de-Bruijn index prefixed with \verb'#',
    \item true and false by \verb"T" and \verb"F", and quantifiers by \verb"\F" and \verb"\E",
    \item operators by \verb"&,|,!,~,+,-,*,/,%<,>,=,>=,/,?"
    \item array operations by \verb"[]", \verb"[.]" and \verb"[=]"
    \item record operations by \verb".fd" and \verb"{fd,...}"
    \item function calls by their name
\end{itemize}

For instance, the term $(x_1 \leq x_2+1)$ will be represented by the pattern \verb|"<=$x+$x1"|.

\paragraph{Alternative Ordering} When several alternatives are available for
discharging a proof obligation, the \texttt{script} and \texttt{tip} provers
of \textsf{Frama-C/WP} choose which one to apply first according to the
following heuristic:
\begin{enumerate}
    \item try internal prover \texttt{"Qed"} with a \texttt{"valid"} verdict;
    \item try any SMT prover with a \texttt{"valid"} verdict;
    \item try the \texttt{"Coq"} proof assistant with a \texttt{"valid"} verdict;
    \item try any Tactic alternative;
    \item finally try the remaining Prover alternatives.
\end{enumerate}

Inside the same level of priority, alternatives are kept in their original
order.
