\chapter{Weakest Preconditions Calculus}
\label{wp-calculus}

\begin{center}
\color{orange}\it For the beta-1 release of Carbon, the final version of this
chapter is not yet available. Complete documentation will be available
for final release of Carbon.
\end{center}

This chapter explains how the WP plug-in computes proof obligations
from ACSL properties.
It is not mandatory to read it in order to use the plug-in,
but it can help to understand the generated formulas, either to be convinced
that they actually means what we want, or more important, 
when the automatic theorem provers fail to
prove them, so that the proof have to be manually completed.

\section{Control Flow Graph}

The elementary WP computation works on one function at a time.
First of all, a control flow graph of the function is computed.
We won't go into the details of how this is done. 
To understand the following, it is sufficient to know that
the nodes of the graph represent the statements, and the edges
are the program points between those statements.

\section{Annotation selection}

The elementary computation is done for one behavior at a time.
It can be a function behavior, or a behavior of a statement specification.
The default (unnamed) behavior of the function and all the default behavior of
the statements are considered as different behaviors.

The first step is to select from the annotations of the function
the properties that will be used during the computation.
Properties can be selected as {\bf goal}, {\bf hypothesis}, or {\bf both}.
Some of them can also be marked as {\bf cut} properties.
Only the properties from the selected behavior are considered,
but if it is a named one, 
some of the properties from the default behavior are also used as hypotheses.

Let's now see how each ACSL property is transformed~- most of the time to a
predicate~- and how it is mapped to a program
point. The result of this step is an annotation table A. 

If the user ask to compute on a specific property 
(through the GUI or with the {\tt -wp-prop} option)
it doesn't change anything to the hypotheses selection
in the following description,
but when a property should be selected as goal (or both), 
and that it is not the asked one,
it is not selected (or only selected as hypothesis) 

\subsection{Function Specifications}

The predicates of the  {\bf requires} and  {\bf assumes} properties
are attached as hypotheses to the function entry point,
except in the case of the main function where 
the {\bf requires} are selected as {\it both}.

The predicates of the  {\bf ensures} (respectively {\bf exits}) properties
are attached as goal
to the function output (resp. exit) point.

The {\bf assigns} property
is attached as a goal to the function output point.
It is translated into a predicate by the memory model.

The properties about {\bf complete} and {\bf disjoint behaviors}
are only added to the default behavior annotations.
The predicates are built according to the ACSL manual, using the
{\bf assumes} part of the behaviors, and are considered as goal
at the function entry point.

\subsection{Statement specification}

If the statement behavior is the target of the current selection,
the post-condition and assigns properties are selected as goals,
the {\bf assumes} is selected as hypothesis, and the {\bf requires}
as both goal and hypothesis. The {\bf ensures} and {\bf exits}
properties are selected as goals. The {\bf ensures} predicates
are attached to the statement output point,
and the {\bf exits} predicates, to the edges that go from the statement
to the exit node of the control flow graph.

If the statement behavior is not the selected one,
its properties are used as hypotheses.

When used as a hypothesis, the statement assigns property is different
than other annotations because it is not used as a predicate.
It is stored at the statement output point along with a label to identify the
statement entry point. It will be used to skip the statement
(see \S\ref{def-wp-assign-hyp}).

The properties about {\bf complete} and {\bf disjoint behaviors} are processed
like for the function specification, except that they are  attached to the statement entry point.

\subsection{Loop properties}

The {\bf loop invariant} is an invariant attached to the edge that goes 
out of the loop node. 
All the {\bf invariant} predicates are selected as both hypothesis and goal
{\bf cut} properties.

The {\bf loop variant} gives two goals :
\begin{itemize}
  \item the predicate which says that the variant is positive
    has to be considered as a goal on the ingoing edges to the loop node,
    and then can be considered as hypothesis on the outgoing edges;
  \item the predicate which says that the expression decrease :
    (\verb!v < \at (L, v)! where L is first program point of the loop)
    as to be proved at the loop back-edge.
\end{itemize}

The {\bf loop assigns} goal is translated into a predicate, like the function or
statement assigns. It is attached as a goal to the back-edge of the loop.
The {\bf loop assigns} hypothesis is attached to the loop head.

\subsection{Other properties}

The {\bf assert} predicate are selected as both goals and hypotheses.
Having several predicates at the same program point is similar 
than having their conjunction.
But it has to be noticed that the assertions coming from the source code
are internally {\sc not} at the same program point. 
This means that the order of the assertions is important because
the assertions that come first can be used to prove the ones that come later.\\

The properties of the {\bf called function specification}
are used as hypothesis,
except the precondition which is selected as both.

%-------------------------------------------------------------------------------
\section{Computation}

The engine takes the previously built
annotation table A that maps the program points to the selected annotations,
and compute a result R for each program point, i.e. the edges of the control 
flow graph.
The result R(X) at the program point X is a list of proof obligations
that ensures that the goals seen so far (the one that are at X or after)
are valid.
So after having determined the result at each program point,
the validation of proof obligations obtained at the entry program point 
of the function (\(R(X_0)\))
ensures that all the goals of the annotation table are satisfied if
the hypotheses are.

The result at the program point X, is computed from:
\begin{itemize}
  \item the annotations A(X) which is composed of:
    \begin{itemize}
      \item goals \(G_1, ..., G_n\),
      \item hypotheses \(H_1, ... H_m\),
      \item and both \(B_1, ... B_p\),
    \end{itemize}
  \item a goal P which most of the time is the propagation of the results 
    obtained on X successors across the node that follows X
    (but we will see that P can be different on some cases),
  \item some proof obligation PO(X) that can come from previous calculus
    (see \S\ref{def-wp-assign-hyp})
\end{itemize}
and then, \(R(X)\) is given by \(F(X, P)\), which use \(A(X)\) and \(PO(X)\),
to compute the following proof obligation list:
\[ 
F(X, P) =
\left\{\begin{array}{ll}
 H_1 \wedge ... \wedge H_m \Rightarrow  B_i & \textrm{for each } B_i\\
(H_1 \wedge ... \wedge H_m \wedge B_1 \wedge ... \wedge B_p) \Rightarrow G_i 
                                         & \textrm{for each } G_i\\
(H_1 \wedge ... \wedge H_m \wedge B_1 \wedge ... \wedge B_p) \Rightarrow P_i 
                                    & \textrm{for each } P_i \textrm{ in } P\\
(H_1 \wedge ... \wedge H_m \wedge B_1 \wedge ... \wedge B_p) \Rightarrow PO_i
                               & \textrm{for each } PO_i \textrm{ in } PO(X)\\
\end{array}
\right.
\]
Notice that the properties B, selected as {\it both}, are not mixed
with other goals and hypotheses so that they can be
combined in a way 
such that they are not used as hypotheses to prove themselves!\\


Let's now see how \(P\) and \(PO(X)\) are built.
There are two special cases: the cut properties (\S\ref{def-wp-cut})
and the assigns hypothesis (\S\ref{def-wp-assign-hyp}). 
All the other cases are handled the same way (\S\ref{def-wp-def}).

\subsection{Cut properties}\label{def-wp-cut}

When a program point with a cut property $C$ is reached, 
it means that the result at that point doesn't depend on the result of its
successors. This can be useful to split a proof (not available yet), 
but it is even more useful for loops,
because without it, the calculation would not terminate.

If the cut property is selected as a goal, it is propagated, so we have 
\(P = C\). Otherwise, there is nothing to propagate, so \(P = true\).

As a hypothesis, $C$ is used to prove \(WP(X)\). 
So there is a proof obligation:
\[ P' \; = \; C \Rightarrow WP(X) \]
$P'$ cannot be integrated in \(R(X)\) because in might depend on
\(R(X)\) if X is in a loop.
So $P'$ has either to be added as a proof obligation
by closing the free variables and adding it to \(PO(X_0)\),
or proved separately as a goal at X program point to consider
in a second pass of computation.
The later possibility can be interesting in order to be able to use
information from the statements before X, in addition to the hypothesis \(C\),
to prove \(WP(X)\). 


\subsection{Assigns hypothesis}\label{def-wp-assign-hyp}

An assigns hypothesis specify an over-approximation of the modified data
between two program points \(X_1\) and \(X_2\). The property is stored as
an annotation of \(X_2\), along with the label of \(X_1\).
It is used nearly like a cut property in order to skip the statements
between \(X_1\) and \(X_2\).
\[
R(X_2) = F (X, true) 
\quad\textrm{ and }\quad 
PO(X_1) = \forall {assigns}, WP(X_2)
\]
That is where the PO proof obligation comes from.

\subsection{Other cases}\label{def-wp-def}

For all other cases, $R(X)$ is computed using $WP(X)$ (see \S\ref{sec-wp}):
\[
R(X) = F (X, WP(X))
\]
and need no proof obligation aside.

%-------------------------------------------------------------------------------
\section{Weakest Precondition}\label{sec-wp}

As explained in \S\ref{wp-intro-calculus},
the WP defines how to compute the property that must be valid before
a statement S to ensure that a property P holds after S.
Because the computation are done on a CFG, it means that the WP has to define
how to compute the property that must be valid on the edges before a node
to ensure that the properties of the edges after the node hold.
So the WP must be defined for each kind of node that can be found in the CFG.

\subsection{Assignment and return}\label{def-wp-asgn}

The WP of an assignment is defined by the memory model as explained in chapter
\ref{wp-models}.
If the memory model is not able to translate the right part,
it is still possible to assign an unknown value to the location
given by the left part. If the model is not able to translate
the left part to a location, the WP result is {\tt false}\\

The return statement is exactly like an assignment 
where the left part is the logical variable {\tt result}.

\subsection{Function Call}\label{def-wp-call}

The WP of a function call does nothing more than translating the called function
specification and substitute the formal parameters to the call arguments
in it.
The properties are used as normal hypotheses, the assigns property
is used as usual to quantify the modified locations.
Notice that the assigns hypothesis is mandatory in this computation, 
so if it in not provided
in the specification, it has to be considered to be {\it everything}.

\subsection{Goto}\label{def-wp-goto}

The {\tt goto} statements (and all the {\tt break}, {\tt continue}, etc)
do nothing except adding edges to the control flow graph,
so the WP of such a statement node is the identity.

\subsection{Test and switch}\label{def-wp-if}

A test node carries the expression to be tested, and has two outgoing edges.
To compute the WP in order to ensure that \(P_1\) holds 
when the test is true, and \(P_2\) holds otherwise,
the condition \(c\) has first to be translated by the memory model,
and the WP is then:
\[
c \Rightarrow P1 \; \wedge \; \neg c \Rightarrow P2
\]

Switch nodes are handle in a similar way.

\subsection{Loop}\label{def-wp-loop}

The control flow graph loop node represent a natural loop of the program.
It has one outgoing edge that is called the 'head',
and some ingoing edges that belongs either to the entry edges that come
from outside of the loop, or back-edges that are dominated by the head edge.
There can be other loops than natural ones in the control flow graph. 
In that case, there is no loop node, but the cycles are still there.
To be able to compute on such a graph, the cycles must be broken.
It can be done with an invariant property which is used
as a cut annotation (see \ref{def-wp-cut}).\\

If the annotation table doesn't provide a cut property somewhere in the loop,
and in case of there is a control flow graph loop node,
a default {\bf loop invariant} with the predicate {\it true} is used.
In that case, the WP for the back-edges is then {\it true}.
For the entry edges, it would normally be also {\it true} according to the
application of the cutting rule, but that wouldn't give much information!
Instead of that, a mandatory loop assigns property is used as a hypothesis.
As for the function call, if the annotation table doesn't provide it,
it is considered to be {\it everything}.
This hypothesis is used to quantify the annotation at the loop head,
and then it can be propagated backward from entry edges.\\

For the loops that are not natural ones, invariant properties breaking the
cycles are mandatory.
